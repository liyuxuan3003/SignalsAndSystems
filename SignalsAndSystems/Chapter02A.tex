\section{LTI连续系统的经典解法}

\subsection{微分方程}
通常来说,若单输入--单输出系统,激励为$f(t)$,响应为$y(t)$,则描述LTI连续系统激励与响应之间关系的数学模型是$n$阶常系数线性微分方程,它可以写为以下形式
\begin{BoxDefinition}[连续系统的微分方程]
    LTI连续系统可以由以下$n$阶常系数线性微分方程描述
    \begin{Equation}
        \Sum[i=0][n]a_iy^{(i)}(t)=
        \Sum[j=0][m]b_jf^{(j)}(t)
    \end{Equation}
\end{BoxDefinition}

注意到,非齐次项是以激励$f(t)$及激励$f(t)$的导数的形式出现的,然而$f(t)$导数的存在往往会为我们的计算带来困难,不过幸运的是,通过一些简单的变化我们就可以解决这个问题。

\begin{BoxFormula}[连续系统的简化微分方程]*
    LTI连续系统中,若相对响应$y(t)$引入代换函数$y_1(t)$
    \begin{Equation}
        y(t)=\Sum[j=0][m]b_j y_1^{(j)}(t)
    \end{Equation}
    则微分方程可以简化为
    \begin{Equation}
        \Sum[i=0][n]a_iy_1^{(i)}(t)=f(t)
    \end{Equation}
\end{BoxFormula}
\begin{Proof}
    我们已经知道的是\fancyref{def:连续系统的微分方程}
    \begin{Equation}&[1]
        \Sum[i=0][n]a_iy^{(i)}(t)=
        \Sum[j=0][m]b_jf^{(j)}(t)
    \end{Equation}
    我们希望在不改变左端形式的情况下,将右端变成$f(t)$
    \begin{Equation}&[2]
        \Sum[i=0][n]a_iy_1^{(i)}(t)=f(t)
    \end{Equation}
    现在的问题是,$y(t)$怎么用这里新引入的$y_1(t)$表达呢?

    这要运用\fancyref{ppt:LTI系统的微分特性},在\xrefpeq{2}的基础上求导乘常系数,得到
    \begin{Gather}[8pt]
        \Sum[i=0][n]a_i[b_0y_1^{(i)}(t)]=b_0f(t)\\
        \Sum[i=0][n]a_i[b_1y_1^{(i+1)}(t)]=b_1f^{(1)}(t)\\
        \Sum[i=0][n]a_i[b_2y_1^{(i+2)}(t)]=b_2f^{(2)}(t)\\
        \vdots\\
        \Sum[i=0][n]a_i[b_my_1^{(i+m)}(t)]=b_mf^{(m)}(t)
    \end{Gather}
    将以上各式相加
    \begin{Equation}&[3]
        \Sum[i=0][n]a_i\qty[\Sum[j=0][m]b_jy_1^{(j)}]^{(i)}=\Sum[j=0][m]b_jf^{(j)}(t)
    \end{Equation}
    将\xrefpeq{3}对比\xrefpeq{1},即得
    \begin{Equation}*
        y(t)=\Sum[j=0][m]b_jy_1^{(j)}\qedhere
    \end{Equation}
\end{Proof}
之后,以上这两种形式的微分方程我们都会使用,不过后面这种用的多些,也更为重要。

\subsection{微分方程的经典解}
微分方程的经典解法,在微积分的课程中我们已经很熟悉了,下面简要概况结论。

\begin{BoxTheorem}[常系数线性微分方程解的结构]
    对于常系数线性非齐次微分方程
    \begin{Equation}
        \Sum[i=0][n]a_iy^{(i)}(t)=f(t)
    \end{Equation}
    其解可以表示为两部分
    \begin{Equation}
        y=y_\te{h}(t)+y_\te{p}(t)
    \end{Equation}
    其中,$y_\te{p}(t)$是非齐次方程的特解,$y_\te{h}(t)$是对应齐次方程的通解,也称为齐次解。
\end{BoxTheorem}

\subsubsection{通解部分}

通解$y_\te{h}(t)$的形式与激励$f(t)$无关,仅取决于齐次微分方程,即
\begin{Equation}
    \Sum[i=0][n]a_iy_\te{h}^{(i)}(t)=0
\end{Equation}
常考虑以下的特征根方程,即将$y(t)$的几阶导数换成$\lambda$的几次方
\begin{Equation}
    \Sum[i=0][n]a_i\lambda^i=0
\end{Equation}
通解$y_\te{p}(t)$的具体形式,就与以上关于$\lambda$的特征方程根的情况有关,如\xref{tab:不同特征根对应的通解}。

\begin{Tablex}[不同特征根对应的通解]{XX}{\linewidth}
<特征根$\lambda$&通解$y_\text{h}(t)$应添加什么项?\\>
对于每一个$r$重实根$\lambda$&\xgp[4pt][-10pt]{$\mal{\e^{\lambda t}\Sum[k=0][r-1]C_k t^{k}}$}\\
对于每一个$r$重共轭复根$\lambda=\alpha+\j\beta$&\xgp[6pt][-10pt]{$\mal{\e^{\alpha t}\Sum[k=0][r-1]A_kt^k\cos(\beta t)+B_kt^k\sin(\beta t)}$}\\
\end{Tablex}
    
以二阶微分方程为例
\begin{Equation}
    y_\te{h}''(t)+a_1y_\te{h}'(t)+a_0y_\te{h}(t)=0
\end{Equation}
其对应的特征方程应为
\begin{Equation}
    \lambda^2+a_1\lambda+a_0=0
\end{Equation}
判别式
\begin{Equation}
    \Delta=a_1^2-4a_0
\end{Equation}
当$\Delta=a_1^2-4a_0>0$时,此时特征方程有两个不等的实根$\lambda_1,\lambda_2$
\begin{Equation}
    y(t)=C_1\e^{\lambda_1t}+C_2\e^{\lambda_2t}
\end{Equation}
当$\Delta=a_1^2-4a_0=0$时,此时特征方程有两个相等的实根$\lambda$
\begin{Equation}
    y(t)=[C_1+C_2t]\e^{\lambda t}
\end{Equation}
当$\Delta=a_1^2-4a_0<0$时,此时特征方程有两个共轭的复根$\alpha\pm\j\beta$
\begin{Equation}
    y(t)=[C_1\cos(\beta t)+C_2\sin(\beta t)]\e^{\alpha t}
\end{Equation}
通解$y_\te{h}(t)$中包含待定的常系数,这些常系数暂时还不能确定,需要等求出特解$y_\te{p}(t)$后,将通解和特解相加得到非齐次方程完整的解$y(t)=y_\te{h}(t)+y_\te{p}(t)$,再由$y(t)$的初值条件定出。

\subsubsection{特解部分}

特解$y_\te{p}(t)$则与激励$f(t)$的形式有关,我们目前只能处理特定形式的激励,如\xref{tab:不同激励所对应的特解}。
\begin{Tablex}[不同激励所对应的特解]{XlX}
<激励$f(t)$&条件&特解$y_\te{p}(t)$应满足什么形式?\\>
\xgp[4pt][-10pt]{$\mal{\e^{\lambda t}\Sum[k=0][r-1]R_k't^k}$}&$\lambda$是$s$重实根&\xgp[4pt][-10pt]{$\mal{\e^{\lambda t}x^s\Sum[k=0][r-1]R_kt^k}$}\\
\xgp[4pt][-10pt]{$\mal{\e^{\alpha t}\Sum[k=0][r-1]P_kt^k\cos(\beta t)+Q_kt^k\sin(\beta t)}$}&$\lambda=\alpha\pm\j\beta$是$s$重共轭复根&\xgp[4pt][-10pt]{$\mal{\e^{\alpha t}x^s\Sum[k=0][r-1]P_kt^k\cos(\beta t)+Q_kt^k\sin(\beta t)}$}\\
\end{Tablex}
特解$y_\text{p}(t)$中包含的的待定常系数,可以通过将$y_\text{p}(t)$代回微分方程确定,无关初值条件。

再让我们回顾一下完整解的形式,在信号与系统中
\begin{Equation}
    y(t)=y_\te{h}(t)+y_\te{p}(t)
\end{Equation}
\begin{itemize}
    \item 通解$y_\te{h}(t)$常称为\uwave{自由响应}(Nature Response)或\uwave{瞬态响应}(Transient Response)。
    \item 特解$y_\te{p}(t)$常称为\uwave{强迫响应}(Forced Response)或\uwave{稳态响应}(Steady Response)。
\end{itemize}
由以上分析可见,LTI系统的数学模型,常系数线性微分方程的完整解,由齐次通解和特解组成:特解$y_\te{p}(t)$的形式由激励$f(t)$确定,通解$y_\te{h}(t)$的形式则仅仅依赖系统本身的特性,而与激励$f(t)$无关,但需要注意的是,通解$y_\te{h}(t)$中的待定常系数是却是与激励$f(t)$有关的。

\subsection{关于初值的一个重要讨论}
经典解法好像没有任何问题?让我们来看看我们有什么疏漏。试想我们有一个关于$y(t)$的二阶微分方程,非齐次项是$f(t)=\cos(t), t>0$或$f(t)=\e^{-t}, t>0$一类的东西。并且,我们已知初值条件$y(0)$和$y'(0)$。这样按照\xref{subsec:微分方程的经典解}中的流程,应该可以很顺利的解出$y(t)$。但问题出在初值上,很明显,严格的说,这里的初值应当是$y(0_{+})$和$y'(0_{+})$,作为微分方程的练习题,直接给出$y(0_{+})$和$y'(0_{+})$并无大碍。但是,作为实际的应用,我们其实只应该知道激励作用前$y(0_{-})$和$y'(0_{-})$情况,而激励实质上$f(t)=\cos(t)\varepsilon(t)$或$f(t)=\e^{-t}\varepsilon(t)$是包含阶跃函数的,因此,激励作用前后$y(t)$的变化是不连续的,即$y(0_{+})\neq y(0_{-})$以及$y'(0_{+})\neq y'(0_{-})$。

现在的问题就是,如何通过$y(0_{-}), y'(0_{-})$求出$y(0_{+}), y'(0_{+})$?这主要包含两种方法,我们分别称为\uwave{积分法}和\uwave{待定系数法}。作为技巧性的内容,我们下面通过两道例题体会一下其思想。

\begin{BoxExample}[积分法]
    设描述某LTI系统的微分方程为
    \begin{Equation}&[]
        y''(t)+a_1y'(t)+a_0y(t)=b_1f'(t)+b_0f(t)
    \end{Equation}
    已知初值$y(0_{-}), y'(0_{-})$,并且激励是阶跃函数,即$f(t)=\varepsilon(t)$,现求$y(0_{+}), y'(0_{+})$。
\end{BoxExample}

\begin{Proof}
    在\xrefpeq{}中代入$f(t)=\varepsilon(t)$,得到
    \begin{Equation}&[1]
        y''(t)+a_1y'(t)+a_0y(t)=b_1\dirac(t)+b_0\varepsilon(t)
    \end{Equation}
    比较\xrefpeq{1}的等式两端,$y''(t)$应包含$\dirac(t)$,$y'(t)$应包含$\varepsilon(t)$,因而,$y(t)$连续。

    在\xrefpeq{1}两端作$0_{-}$至$0_{+}$的积分
    \begin{Equation}&[2]
        \qquad
        \Int[\zm][\zp]y''(t)\dd{t}+
        a_1\Int[\zm][\zp]y'(t)\dd{t}+
        a_0\Int[\zm][\zp]y(t)\dd{t}=
        b_1\Int[\zm][\zp]\dirac(t)\dd{t}+
        b_0\Int[\zm][\zp]\varepsilon(t)\dd{t}
        \qquad
    \end{Equation}
    由于$y(t)$连续故积分为零,而$\varepsilon(t)$以及最高包含$\varepsilon(t)$的$y'(t)$虽然在$0$处发生阶跃,但是两者在无穷小的区间$[\zm,\zp]$上的积分为仍然为零,因此,上式可以简化为
    \begin{Equation}
        \Int[\zm][\zp]y''(t)\dd{t}=b_1\Int[\zm][\zp]\dirac(t)
    \end{Equation}
    即
    \begin{Equation}
        y'(\zp)-y'(\zm)=b_1
    \end{Equation}
    而由于$y(t)$是连续的
    \begin{Equation}
        y(\zp)-y(\zm)=0
    \end{Equation}
    这就求得了$y(\zp), y'(\zp)$相对$y(\zm), y'(\zm)$的表达式。
\end{Proof}

\begin{BoxExample}[待定系数法]
    设描述某LTI系统的微分方程为
    \begin{Equation}&[]
        y''(t)+a_1y'(t)+a_0y(t)=b_2f''(t)+b_1f'(t)+b_0f(t)
    \end{Equation}
    已知初值$y(0_{-}), y'(0_{-})$,并且激励是冲激函数,即$f(t)=\dirac(t)$,现求$y(0_{+}), y'(0_{+})$。
\end{BoxExample}

\begin{Proof}
    在\xrefpeq{}中代入$f(t)=\dirac(t)$,得到
    \begin{Equation}&[1]
        y''(t)+a_1y'(t)+a_0y(t)=b_2\dirac''(t)+b_1\dirac'(t)+b_0\dirac(t)
    \end{Equation}
    很明显,$y''(t)$最高包含冲激函数的二阶导数$\dirac''(t)$
    \begin{Equation}&[2]
        y''(t)=A\dirac''(t)+B\dirac'(t)+C\dirac(t)+r_2(t)
    \end{Equation}
    $y'(t)$可以表示为
    \begin{Equation}&[3]
        y'(t)=A\dirac'(t)+B\dirac(t)+r_1(t)\qquad r_1(t)=C\varepsilon(t)+\Int[-\infty][t]r_2(t)\dd{t}
    \end{Equation}
    $y(t)$可以表示为
    \begin{Equation}&[4]
        y(t)=A\dirac(t)+r_0(t)\qquad r_0(t)=B\varepsilon(t)+\Int[-\infty][t]r_1(t)\dd{t}
    \end{Equation}
    将\xrefpeq{2},\xrefpeq{3},\xrefpeq{4}相加,得到
    \begin{Split}
        A\dirac''(t)+(a_1A+B)\dirac'(t)&+(a_0A+a_1B+C)\dirac(t)\\ 
        &+[r_2(t)+a_1r_1(t)+a_0r_0(t)]=b_2\dirac''(t)+b_1\dirac'(t)+b_0\dirac(t)
    \end{Split}
    两端冲激函数的各阶导数应当对应系数相等
    \begin{Equation}
        A=b_2\qquad a_1A+B=b_1\qquad a_0A+a_1B+C=b_0
    \end{Equation}
    写成矩阵形式
    \begin{Equation}
        \begin{pmatrix}
            1&0&0\\
            a_1&1&0\\
            a_0&a_1&1\\
        \end{pmatrix}
        \begin{pmatrix}
            A\\
            B\\
            C\\
        \end{pmatrix}
        =
        \begin{pmatrix}
            b_2\\
            b_1\\
            b_0
        \end{pmatrix}
    \end{Equation}
    系数行列式$D$
    \begin{Equation}
        D=\begin{vmatrix}
            1&0&0\\
            a_1&1&0\\
            a_0&a_1&1\\
        \end{vmatrix}=1
    \end{Equation}
    系数$A$
    \begin{Equation}
        A=D_A/D=\begin{vmatrix}
            b_2&0&0\\
            b_1&1&0\\
            b_0&a_1&1\\
        \end{vmatrix}=b_2
    \end{Equation}
    系数$B$
    \begin{Equation}
        B=D_B/D=\begin{vmatrix}
            1&b_2&0\\
            a_1&b_1&0\\
            a_0&b_0&1\\
        \end{vmatrix}=b_1-a_1b_2
    \end{Equation}
    系数$C$
    \begin{Equation}
        C=D_C/D\begin{vmatrix}
            1&0&b_2\\
            a_1&1&b_1\\
            a_0&a_1&b_0\\
        \end{vmatrix}=b_0-b_1a_1-b_2(a_0-a_1^2)
    \end{Equation}
    至此,$A,B,C$这三个待定系数都是已知的了。

    根据\xrefpeq{2}
    \begin{Equation}
        y''(t)=A\dirac''(t)+B\dirac'(t)+C\dirac(t)+r_2(t)
    \end{Equation}
    两端在$[\zm,\zp]$的区间上积分,注意到$\dirac'(t),\dirac''(t)$的积分都是零
    \begin{Equation}
        y'(\zp)-y'(\zm)=C
    \end{Equation}
    根据\xrefpeq{3}
    \begin{Equation}
        y'(t)=A\dirac'(t)+B\dirac(t)+r_1(t)
    \end{Equation}
    两端在$[\zm,\zp]$的区间上积分,注意到$\dirac'(t)$的积分是零
    \begin{Equation}
        y(\zp)-y(\zm)=B
    \end{Equation}
    这就求得了$y(\zp), y'(\zp)$相对$y(\zm), y'(\zm)$的表达式。
\end{Proof}

积分法和待定系数法是由$\zm$值计算$\zp$值的两种不同方法,积分法过程相对较为简单,但仅能适合较简单的情况,待定系数法过程相对较为复杂,但相对通用。积分法用的比较多些。