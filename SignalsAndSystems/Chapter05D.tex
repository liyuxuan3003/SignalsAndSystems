\section{傅里叶变换的性质}
在本节,我们将详细的探讨并证明傅里叶变换的若干重要性质。

\subsection{傅里叶变换的线性性质}
\begin{BoxProperty}[傅里叶变换的线性性质]
    傅里叶变换的线性性质是指,若$x(t)\Farr X(\j\omega)$且$y(t)\Farr Y(\j\omega)$,则
    \begin{Equation}
        ax(t)+by(t)\Farr aX(\j\omega)+bY(\j\omega)
    \end{Equation}
\end{BoxProperty}

\begin{Proof}
    容易证明。
\end{Proof}

\subsection{傅里叶变换的对偶性质}
\begin{BoxProperty}[傅里叶变换的对偶性质]
    傅里叶变换的对偶性质是指,若$x(t)\Farr X(\j\omega)$,则
    \begin{Equation}
        X(\j t)\Farr 2\pi x(-\omega)
    \end{Equation}
\end{BoxProperty}
\begin{Proof}
    根据\fancyref{fml:傅里叶变换}的逆变换式
    \begin{Equation}
        x(t)=\frac{1}{2\pi}\Int[-\infty][\infty]X(\j\omega)\e^{\j\omega t}\dd{\omega}
    \end{Equation}
    将$t$改换为$-t$
    \begin{Equation}
        x(-t)=\frac{1}{2\pi}\Int[-\infty][\infty]X(\j\omega)\e^{-\j\omega t}\dd{\omega}
    \end{Equation}
    将$t$变为$\omega$,将$\omega$变为$t$
    \begin{Equation}
        x(-\omega)=\frac{1}{2\pi}\Int[-\infty][\infty]X(\j t)\e^{-\j\omega t}\dd{t}
    \end{Equation}
    两边同乘$2\pi$
    \begin{Equation}
        2\pi f(-\omega)=\Int[-\infty][\infty]X(\j t)\e^{-\j\omega t}\dd{t}
    \end{Equation}
    即
    \begin{Equation}*
        F(-\j t)\Farr 2\pi f(-\omega)\qedhere
    \end{Equation}
\end{Proof}
傅里叶变换的对偶性质指出
\begin{Equation}
    x(t)\Farr X(\j\omega)\qquad X(\j t)\Farr 2\pi x(-\omega)
\end{Equation}
傅里叶变换的对偶性质告诉我们,对一个函数的傅里叶变换再做傅里叶变换,将得到原函数翻转的$2\pi$倍。对偶性质表述中的符号的变换可能让我们感到有些困惑,只需要牢记两点,首先所谓时域函数和频域函数$x(t),X(\j\omega)$本质都是关于某个变量的函数,自变量记为什么,自变量代表什么含义并不重要,因此即便以$x(\omega),X(\j t)$的形式出现也不值得我们惊奇。其次,再次强调这里$X(\j\omega)$的$\j$是函数名的一部分,因此在$X(\j t)$中也并没有什么多余的$\j$被添加。

在\xref{exp:冲激函数的傅里叶变换}和\xref{exp:常值函数的傅里叶变换}中,我们曾分别求得了
\begin{Equation}
    \dirac(t)\Farr 1\qquad 1\Farr 2\pi\dirac(\omega)
\end{Equation}
这两个结论都被独立的证明了,而对偶性告诉我们,如果左式成立,则必有右式成立。这就给我们了一个计算傅里叶变换的好思路:若要计算某个函数的傅里叶变换,而这个函数恰好是某个已知傅里叶变换的像函数,那我们就可以利用对偶性质立即得到结果,不必任何计算。\nopagebreak

这个想法可以通过下面这个例子实践一下,例子中涉及的函数我们都很熟悉,正式定义一下。\goodbreak
\begin{BoxDefinition}[门函数]
    门函数$\rect(t)$被定义为\footnote[2]{不同参考书对此有不同的记号和定义,例如\cite{主参考书}中以$g_{\tau}(t)$表示宽度为$\tau$高度为$1$的门函数,它和这里定义的$\rect(t)$的关系是$g_{1}(t)=2\rect(t/2)$和$g_{\tau}(t)=2\rect(t/2\tau)$。这里选取的$\rect(t)$的定义可以使其傅里叶变化具有最为简洁的形式。}
    \begin{Equation}
        \rect(t)=\begin{cases}
            1/2, &|t|\leq 1\\
            0, &|t|>1
        \end{cases}
    \end{Equation}
\end{BoxDefinition}

\begin{BoxDefinition}[采样函数]
    采样函数$\sinc(t)$被定义为\footnote[2]{采样函数$\sinc(t)$有时也被记为$\Sa(t)$,另一项争议是$\sinc(t)=\sin t/t$和$\sinc(t)=\sin\pi t/\pi t$,后者的优点是归一化。}
    \begin{Equation}
        \sinc(t)=\frac{\sin t}{t}
    \end{Equation}
\end{BoxDefinition}

现在我们要指出:门函数和采样函数间具有傅里叶变换的关系!
\begin{BoxExample}[门函数和采样函数的傅里叶变换关系]
    门函数和采样函数间具有傅里叶变换的关系
    \begin{Equation}
        \rect(t)\Farr \sinc(\omega)\qquad
        \sinc(\omega)\Farr 2\pi\rect(t)
    \end{Equation}
\end{BoxExample}

\begin{Proof}
    我们先证明$x(t)=\rect(t)$的傅里叶变换,根据\fancyref{fml:傅里叶变换}
    \begin{Equation}
        X(\j\omega)=\Int[-\infty][\infty]x(t)\e^{-\j\omega t}\dd{t}=\Int[-\infty][\infty]\rect(t)\e^{-\j\omega t}\dd{t}
    \end{Equation}
    这个积分非常容易,依据\fancyref{def:门函数}
    \begin{Equation}
        X(\j\omega)=\frac{1}{2}\Int[-1][+1]\e^{-\j\omega t}\dd{t}
    \end{Equation}
    即
    \begin{Equation}
        X(\j\omega)=\eval{\frac{-\e^{-\j\omega t}}{2\j\omega}}_{-1}^{+1}=\frac{\e^{\j\omega t}-\e^{-\j\omega t}}{2\j\omega}=\frac{\sin(\omega)}{\omega}=\sinc(\omega)
    \end{Equation}
    由此就证明了
    \begin{Equation}
        \rect(t)\Farr\sinc(\omega)
    \end{Equation}
    由\fancyref{ppt:傅里叶变换的对偶性质},我们可以立即得到
    \begin{Equation}
        \sinc(t)\Farr 2\pi \rect(\omega)
    \end{Equation}
    试想,如果这里不运用对偶性质,而是直接计算$\sinc(t)$的傅里叶变换,那会多么麻烦!
\end{Proof}

\begin{Tablex}[门函数和采样函数的傅里叶变换关系]{|Y|Y|}
<>
\xcell<Y>[2ex][0ex]{\includegraphics[scale=0.75]{build/Chapter05C_02a.fig.pdf}}&
\xcell<Y>[2ex][0ex]{\includegraphics[scale=0.75]{build/Chapter05C_02b.fig.pdf}}\\ \hlinelig
\mc{2}[1.5ex][1ex](|c|){$\rect(t)\Farr\sinc(\omega)$}\\ \hlinemid
\xcell<Y>[2ex][0ex]{\includegraphics[scale=0.75]{build/Chapter05C_02c.fig.pdf}}&
\xcell<Y>[2ex][0ex]{\includegraphics[scale=0.75]{build/Chapter05C_02d.fig.pdf}}\\ \hlinelig
\mc{2}[1.5ex][1ex](|c|){$\sinc(t)\Farr 2\pi\rect(\omega)$}\\
\end{Tablex}

\subsection{傅里叶变换的共轭性质}
\begin{BoxProperty}[傅里叶变换的共轭性质]
    傅里叶变换的共轭性质是指,若$x(t)\Farr X(\j\omega)$,则
    \begin{Equation}
        x^{*}(t)\Farr X^{*}(-\j\omega)
    \end{Equation}
\end{BoxProperty}
\begin{Proof}
    根据\fancyref{fml:傅里叶变换}
    \begin{Equation}
        X(\j\omega)=\Int[-\infty][\infty]x(t)\e^{-\j\omega t}\dd{t}
    \end{Equation}
    对上式两端同时取共轭
    \begin{Equation}
        X^{*}(\j\omega)=\qty[\Int[-\infty][\infty]x(t)\e^{-\j\omega t}\dd{t}]^{*}
    \end{Equation}
    将共轭和积分交换顺序
    \begin{Equation}
        X^{*}(\j\omega)=\Int[-\infty][\infty]x^{*}(t)\e^{\j\omega t}\dd{t}
    \end{Equation}
    将$\omega$用$-\omega$代替
    \begin{Equation}
        X^{*}(-\j\omega)=\Int[-\infty][\infty]x^{*}(t)\e^{-\j\omega t}\dd{t}
    \end{Equation}
    这就表明
    \begin{Equation}*
        x^{*}(t)\Farr X^{*}(-\j\omega)\qedhere
    \end{Equation}
\end{Proof}

傅里叶变换的共轭性质指出:共轭的傅里叶变换,等于傅里叶变换的共轭的翻转。

假若$x(t)$是实函数,由于$x(t)=x^{*}(t)$,那么,两者的傅里叶变换$X(\j\omega),X^{*}(-\j\omega)$也应相等
\begin{Equation}
    X(\j\omega)=X^{*}(-\j\omega)
\end{Equation}
即有
\begin{Equation}
    \Re[X(\j\omega)]=\Re[X(-\j\omega)]\qquad \Im[X(\j\omega)]=-\Im[X(-\j\omega)]
\end{Equation}
或者
\begin{Equation}
    |X(\j\omega)|=|X(-\j\omega)|\qquad
    \Arg[X(\j\omega)]=-\Arg[X(-\j\omega)]
\end{Equation}
这就告诉我们,若$x(t)$为实函数且$x(t)\Farr X(\j\omega)$
\begin{itemize}
    \item 若$X(\j\omega)$以实部和虚部的形式表示,实部为$\omega$的偶函数,虚部为$\omega$的奇函数。
    \item 若$X(\j\omega)$以模值和辐角的形式表示,模值为$\omega$的偶函数,辐角为$\omega$的奇函数。
\end{itemize}
这意味着,由于对称性的存在,实函数的频域函数(实部虚部/模值辐角)均只需要画出一半。

\subsection{傅里叶变换的尺度变换}
\begin{BoxProperty}[傅里叶变换的尺度变换]
    傅里叶变换的尺度变换是指,若$x(t)\Farr X(\j\omega)$,则
    \begin{Equation}
        x(at)\Farr\frac{1}{|a|}X\qty(\j\frac{\omega}{a})
    \end{Equation}
    特别的,当$a=-1$时有
    \begin{Equation}
        x(-t)\Farr F(-\j\omega)
    \end{Equation}
\end{BoxProperty}

\begin{Proof}
    根据\fancyref{fml:傅里叶变换}
    \begin{Equation}&[1]
        \F{x(at)}=\Int[-\infty][\infty]x(at)\e^{-\j\omega t}\dd{t}
    \end{Equation}
    现在作变量代换,令$\tau=at$,则有
    \begin{Equation}&[2]
        t=\frac{\tau}{a}\qquad \dd{t}=\frac{\dd{\tau}}{a}
    \end{Equation}
    当$a>0$时,积分限不需要交换
    \begin{Equation}&[3]
        \qquad\quad
        \F{x(at)}=\frac{1}{a}\Int[-\infty][\infty]x(\tau)\e^{-\j\omega \tau/a}\dd{\tau}=\frac{1}{a}\Int[-\infty][\infty]x(\tau)\e^{-\j(\omega/a)\tau}\dd{\tau}=\frac{1}{a}X\qty(\j\frac{\omega}{a})
        \qquad\quad
    \end{Equation}
    当$a<0$时,积分限需要交换,这会带来一个额外的负号
    \begin{Equation}&[4]
        \qquad\quad
        \F{x(at)}=\frac{1}{a}\Int[\infty][-\infty]x(\tau)\e^{-\j\omega \tau/a}\dd{\tau}=-\frac{1}{a}\Int[-\infty][\infty]x(\tau)\e^{-\j(\omega/a)\tau}\dd{\tau}=-\frac{1}{a}X\qty(\j\frac{\omega}{a})
        \qquad\quad
    \end{Equation}
    归纳以上两式的结论,即可得到
    \begin{Equation}
        \F{x(at)}=\frac{1}{|a|}X\qty(\j\frac{\omega}{a})
    \end{Equation}
    即
    \begin{Equation}*
        x(at)\Farr\frac{1}{|a|}X\qty(\j\frac{\omega}{a})\qedhere
    \end{Equation}
\end{Proof}
傅里叶变换的尺度变换性质指出
\begin{itemize}
    \item 时域上的压缩($a>1$),意味着频域上的展宽(同时幅值减小)。
    \item 时域上的展宽($a<1$),意味着频域上的压缩(同时幅值增大)。
\end{itemize}
概括起来,就是,\empx{时域和频域不可同时被压缩}。

傅里叶变换的尺度变换性质的一个直观理解是:$2$倍速声音会更尖锐,$0.5$倍速声音会更低沉。

\subsection{帕斯瓦尔定理}
\begin{BoxTheorem}[帕斯瓦尔定理]
    \uwave{帕斯瓦尔定理}(Parseval's Theorem)是指,若$x(t)\Farr X(\j\omega)$,则
    \begin{Equation}
        \Int[-\infty][\infty]|x(t)|^2=\frac{1}{2\pi}\Int[-\infty][\infty]|X(\j\omega)|^2\dd{\omega}
    \end{Equation}
\end{BoxTheorem}
\begin{Proof}
    将$|x(t)|^2$拆分为$x(t)$和其共轭$x^{*}(t)$的积
    \begin{Equation}
        \Int[-\infty][\infty]|x(t)|^2\dd{t}=
        \Int[-\infty][\infty]x(t)x^{*}(t)\dd{t}
    \end{Equation}
    根据\fancyref{ppt:傅里叶变换的共轭性质},有$x^{*}(t)\Farr X^{*}(-\j\omega)$
    \begin{Equation}
        \Int[-\infty][\infty]|x(t)|^2\dd{t}=\Int[-\infty][\infty]x(t)\qty[\frac{1}{2\pi}\Int[-\infty][\infty]X^{*}(-\j\omega)\e^{\j\omega t}\dd{\omega}]\dd{t}
    \end{Equation}
    将这里的$\omega$代换为$-\omega$
    \begin{Equation}
        \Int[-\infty][\infty]|x(t)|^2\dd{t}=\Int[-\infty][\infty]x(t)\qty[\frac{1}{2\pi}\Int[-\infty][\infty]X^{*}(\j\omega)\e^{-\j\omega t}\dd{\omega}]\dd{t}
    \end{Equation}
    将积分顺序交换一下
    \begin{Equation}
        \Int[-\infty][\infty]|x(t)|^2\dd{t}=\frac{1}{2\pi}\Int[-\infty][\infty]X^{*}(\j\omega)\qty[\Int[-\infty][\infty]x(t)\e^{-\j\omega t}\dd{t}]\dd{\omega}
    \end{Equation}
    这里方括号中的恰就是$x(t)$的傅里叶变换$X(\j\omega)$
    \begin{Equation}
        \Int[-\infty][\infty]|x(t)|^2\dd{t}=\frac{1}{2\pi}\Int[-\infty][\infty]X^{*}(\j\omega)X(\j\omega)\dd{\omega}
    \end{Equation}
    即
    \begin{Equation}*
        \Int[-\infty][\infty]|x(t)|^2\dd{t}=\frac{1}{2\pi}\Int[-\infty][\infty]|X(\j\omega)|^2\dd{\omega}\qedhere
    \end{Equation}
\end{Proof}

\subsection{傅里叶变换的平移性质}
\begin{BoxProperty}[傅里叶变换的平移性质]*
    傅里叶变换的时移性质是指,若$x(t)\Farr X(\j\omega)$,则
    \begin{Equation}
        x(t\pm t_0)\Farr\e^{\pm\j\omega t_0}X(\j\omega)
    \end{Equation}
    傅里叶变换的频移性质是指,若$x(t)\Farr X(\j\omega)$,则
    \begin{Equation}
        x(t)\e^{\mp\j\omega_0t}\Farr X(\j(\omega\pm\omega_0))
    \end{Equation}
\end{BoxProperty}

\begin{Proof}\nopagebreak
    根据\fancyref{fml:傅里叶变换}
    \begin{Equation}&[1]
        x(t)=\frac{1}{2\pi}\Int[-\infty][\infty]X(\j\omega)\e^{\j\omega t}\dd{\omega}
    \end{Equation}\goodbreak
    \subparagraph{时移性质} 在\xrefpeq{1}中以$t-t_0$代换$t$,可得
    \begin{Equation}
        x(t\pm t_0)=\frac{1}{2\pi}\Int[-\infty][\infty]X(\j\omega)\e^{\j\omega(t\pm t_0)}\dd{\omega}
    \end{Equation}
    整理得到
    \begin{Equation}
        x(t\pm t_0)=\frac{1}{2\pi}\Int[-\infty][\infty][X(\j\omega)\e^{\pm\j\omega t_0}]\e^{\j\omega t}\dd{\omega}
    \end{Equation}
    这就证明了时移性质
    \begin{Equation}*
        x(t\pm t_0)\Farr X(\j\omega)\e^{\pm\j\omega t_0}
    \end{Equation}
    \subparagraph{频移性质} 在\xrefpeq{1}中两端同乘$\e^{\mp\j\omega_0t}$
    \begin{Equation}
        x(t)\e^{\mp\j\omega_0t}=\frac{1}{2\pi}\Int[-\infty][\infty]X(\j\omega)\e^{\j(\omega\mp\omega_0)t}\dd{\omega}
    \end{Equation}
    令$w=\omega\mp\omega_0$,则$\omega=w\pm\omega_0$
    \begin{Equation}
        x(t)\e^{\mp\j\omega_0t}=\frac{1}{2\pi}\Int[-\infty][\infty]X(\j(w\pm \omega_0))\e^{\j wt}\dd{w}
    \end{Equation}
    令$w$重新记为$\omega$
    \begin{Equation}
        x(t)\e^{\mp\j\omega_0t}=\frac{1}{2\pi}\Int[-\infty][\infty]X(\j(\omega\pm \omega_0))\e^{\j \omega t}\dd{\omega}
    \end{Equation}
    这就证明了频移性质
    \begin{Equation}*
        x(t)\e^{\mp\j\omega_0t}\Farr X(\j(\omega\pm\omega_0))\qedhere
    \end{Equation}
\end{Proof}

傅里叶变换的平移性质指出:时域上的平移在频域上表现为相移,反之亦然。

\subsection{傅里叶变换的卷积性质}
\begin{BoxProperty}[傅里叶变换的卷积性质]
    傅里叶变换的时域卷积性质是指,若$x_1(t)\Farr X_1(\j\omega)$和$x_2(t)\Farr X_2(\j\omega)$
    \begin{Equation}
        x_1(t)*x_2(t)\Farr X_1(\j\omega)X_2(\j\omega)
    \end{Equation}
    傅里叶变换的频域卷积性质是指,若$x_1(t)\Farr X_1(\j\omega)$和$x_2(t)\Farr X_2(\j\omega)$
    \begin{Equation}
        x_1(t)x_2(t)=\frac{1}{2\pi}X_1(\j\omega)*X_2(\j\omega)
    \end{Equation}
\end{BoxProperty}\goodbreak

\begin{Proof}
    根据\fancyref{def:卷积积分}
    \begin{Equation}&[1]
        x_1(t)*x_2(t)=\Int[-\infty][\infty]x_1(\tau)x_2(t-\tau)\dd{\tau}
    \end{Equation}
    根据\fancyref{fml:傅里叶变换},求其傅里叶变换
    \begin{Equation}&[2]
        \F{x_1(t)*x_2(t)}=\Int[-\infty][\infty]\qty[\Int[-\infty][\infty]x_1(\tau)x_2(t-\tau)\dd{\tau}]\e^{-\j\omega t}\dd{t}
    \end{Equation}
    调整积分顺序
    \begin{Equation}&[3]
        \F{x_1(t)*x_2(t)}=\Int[-\infty][\infty]x_1(\tau)\qty[\Int[-\infty][\infty]x_2(t-\tau)\e^{-\j\omega t}\dd{t}]\dd{\tau}
    \end{Equation}
    根据\fancyref{ppt:傅里叶变换的平移性质}
    \begin{Equation}&[4]
        \Int[-\infty][\infty]x_2(t-\tau)\e^{-\j\omega t}\dd{t}=X_2(\j\omega)\e^{-\j\omega\tau}
    \end{Equation}
    将\xrefpeq{4}代回\xrefpeq{3}
    \begin{Equation}
        \F{x_1(t)*x_2(t)}=\Int[-\infty][\infty]x_1(\tau)X_2(\j\omega)\e^{-\j\omega\tau}\dd{\tau}
    \end{Equation}
    再次应用傅里叶变换的定义
    \begin{Equation}
        \F{x_1(t)*x_2(t)}=X_1(\j\omega)X_2(\j\omega)
    \end{Equation}
    即
    \begin{Equation}
        x_1(t)*x_2(t)\Farr X_1(\j\omega)X_2(\j\omega)
    \end{Equation}
    这就证明了时域卷积性质,类似也可以证明频域卷积性质。
\end{Proof}

傅里叶变换的卷积性质指出:时域上的卷积在频域上表现为乘积,反之亦然。

有了傅里叶变换的卷积性质后,进一步推导傅里叶变换的微分性质和积分性质就容易很多了。

\subsection{傅里叶变换的微分性质}
\begin{BoxProperty}[傅里叶变换的微分性质]
    傅里叶变换的时域微分性质是指,若$x(t)\Farr X(\j\omega)$,则
    \begin{Equation}
        x^{(n)}(t)\Farr X(\j\omega)(\j\omega)^n
    \end{Equation}
    傅里叶变换的频域微分性质是指,若$x(t)\Farr X(\j\omega)$,则
    \begin{Equation}
        x(t)(-\j t)^n\Farr X^{(n)}(\j\omega)
    \end{Equation}
\end{BoxProperty}

\begin{Proof}
    根据\fancyref{ppt:卷积与冲激函数}和\fancyref{ppt:卷积的微分与积分特性}
    \begin{Equation}
        x^{(n)}(t)=x^{(n)}(t)*\dirac(t)=x(t)*\dirac^{(n)}(t)
    \end{Equation}
    根据\fancyref{ppt:傅里叶变换的卷积性质}
    \begin{Equation}
        \F{x^{(n)}(t)}=\F{x(t)*\dirac^{(n)}(t)}=\F{x(t)\vphantom{x^{(n)}(t)}}\F{\dirac^{(n)}(t)}
    \end{Equation}
    或者
    \begin{Equation}
        \F{x^{(n)}(t)}=X(\j\omega)\F{\dirac^{(n)}(t)}
    \end{Equation}
    根据\fancyref{exp:冲激函数的傅里叶变换}
    \begin{Equation}
        \F{x^{(n)}(t)}=X(\j\omega)(\j\omega)^n
    \end{Equation}
    即
    \begin{Equation}*
        x^{(n)}(t)\Farr X(\j\omega)(\j\omega)^n
    \end{Equation}
    这就证明了时域微分性质,类似也可以证明频域微分性质。
\end{Proof}

由此可见,时域上的微分转换,到频域上就变成了很简单的代数运算。

% \begin{Proof}
%     简洁起见,以下证明均以一阶导数为例进行证明。\goodbreak

%     根据\fancyref{fml:傅里叶变换}
%     \begin{Equation}
%         x(t)=\frac{1}{2\pi}\Int[-\infty][\infty]X(\j\omega)\e^{\j\omega t}\dd{\omega}
%     \end{Equation}
%     两端对$t$求导
%     \begin{Equation}
%         x'(t)=\dv{t}\frac{1}{2\pi}\Int[-\infty][\infty]X(\j\omega)\e^{\j\omega t}\dd{\omega}=\frac{1}{2\pi}\Int[-\infty][\infty]X(\j\omega)(\j\omega)\e^{\j\omega t}\dd{\omega}
%     \end{Equation}
%     因此有
%     \begin{Equation}*
%         x'(t)\Farr X(\j\omega)(\j\omega)
%     \end{Equation}
%     根据\fancyref{fml:傅里叶变换}
%     \begin{Equation}
%         X(\omega)=\Int[-\infty][\infty]x(t)\e^{-\j\omega t}\dd{t}
%     \end{Equation}
%     两端对$\omega$求导
%     \begin{Equation}
%         X'(\omega)=\dv{\omega}\Int[-\infty][\infty]x(t)\e^{-\j\omega t}\dd{t}=\Int[-\infty][\infty]x(t)(-\j t)\e^{-\j\omega t}\dd{t}
%     \end{Equation}
%     因此有
%     \begin{Equation}*
%         x(t)(-\j t)\Farr X'(\omega)\qedhere
%     \end{Equation}
% \end{Proof}

\subsection{傅里叶变换的积分性质}
\begin{BoxProperty}[傅里叶变换的积分性质]
    傅里叶变换的时域积分性质是指,若$x(t)\Farr X(\j\omega)$,则
    \begin{Equation}
        x^{(-1)}(t)\Farr X(\j\omega)(\j\omega)^{-1}+\pi X(\j 0)\dirac(\omega)
    \end{Equation}
    傅里叶变换的频域积分性质是指,若$x(t)\Farr X(\j\omega)$,则
    \begin{Equation}
        x(t)(-\j t)^{-1}+\pi x(0)\dirac(t)\Farr X^{-1}(\j\omega)
    \end{Equation}
    这里$x^{(-1)}(t)$和$X^{(-1)}(\j\omega)$的含义是
    \begin{Equation}
        x^{(-1)}(t)=\Int[-\infty][t]x(\tau)\dd{\tau}\qquad
        X^{(-1)}(\omega)=\Int[-\infty][\omega]X(\j w)\dd{w}
    \end{Equation}
\end{BoxProperty}\goodbreak

\begin{Proof}
    根据\fancyref{ppt:卷积与冲激函数}和\fancyref{ppt:卷积的微分与积分特性}
    \begin{Equation}
        x^{(-1)}(t)=x^{(-1)}(t)*\dirac(t)=x(t)*\dirac^{(-1)}(t)
    \end{Equation}
    我们知道,冲激函数的积分是阶跃函数
    \begin{Equation}
        x^{(-1)}(t)=x(t)*u(t)
    \end{Equation}
    根据\fancyref{ppt:傅里叶变换的卷积性质}
    \begin{Equation}
        \F{x^{-1}(t)}=\F{x(t)*u(t)\vphantom{x^{-1}(t)}}=\F{x(t)\vphantom{x^{-1}(t)}}\F{u(t)\vphantom{x^{-1}(t)}}
    \end{Equation}
    或者
    \begin{Equation}
        \F{x^{-1}(t)}=X(\j\omega)\F{u(t)\vphantom{x^{-1}(t)}}
    \end{Equation}
    根据\fancyref{exp:阶跃函数的傅里叶变换}
    \begin{Equation}
        \F{x^{-1}(t)}=X(\j\omega)\qty[\pi\dirac(\omega)+\frac{1}{\j\omega}]
    \end{Equation}
    展开得
    \begin{Equation}
        \F{x^{-1}(t)}=X(\j\omega)(\j\omega)^{-1}+\pi X(\j\omega)\dirac(\omega)
    \end{Equation}
    运用\fancyref{ppt:冲激函数与普通函数的乘积}
    \begin{Equation}
        \F{x^{-1}(t)}=X(\j\omega)(\j\omega)^{-1}+\pi X(\j 0)\dirac(\omega)
    \end{Equation}
    即
    \begin{Equation}*
        x^{(-1)}(t)\Farr X(\j\omega)(\j\omega)^{-1}+\pi X(\j 0)\dirac(\omega)
    \end{Equation}
    这就证明了时域积分性质,类似也可以证明频域积分性质。
\end{Proof}

由此可见,积分性质和微分性质是非常相似的,除了增加了直流冲激项。

\section{傅里叶变换与傅里叶级数}
傅里叶变换最初是从傅里叶级数引出的,适用于非周期信号,但其实,傅里叶变换也可以用于周期信号,这样就可以在统一的框架内考虑周期信号和非周期信号。以上就是本节的目标。

