\section{绪论}
信号与系统由两部分组成:\uwave{信号理论}、\uwave{系统理论}。两者的设计内容都十分广泛
\begin{itemize}
    \item \uwave{信号}(Signal)理论包含:信号分析、信号综合、信号传输、信号处理。
    \item \uwave{系统}(System)理论包含:系统分析、系统综合。
\end{itemize}

在本课程中,我们主要讨论信号分析和系统分析,它们是信号与系统理论的基础。在电路中我们曾学过,电路分析是指给定电路求特性,电路综合是指给定特性反过来设计电路。因此可以预见的是,本课程要研究的就是,\empx{对于给定的系统,在输入信号的作用下产生的输出信号}。而在其中,信号分析侧重于信号的解析表示和性质,系统分析则更侧重于描述系统的功能等。

\begin{itemize}
    \item 输入信号,也称为系统的\uwave{激励}(Excitation)。
    \item 输出信号,也称为系统的\uwave{响应}(Response)。
\end{itemize}

\begin{Figure}[信号与系统]
    \includegraphics{build/Chapter01A_01.fig.pdf}
\end{Figure}

现在的问题是,信号和系统到底是指什么?我们可以将其具象化为电信号和电子线路,但应当指出的是,信号与系统是更为抽象的。信号可以广义地定义为随时间或空间变化的某些物理量,系统则是指若干相互关联相互作用按一定规律组合而成的具有特定功能的整体。系统的观点广泛的应用于物理、化学、生物、经济、社会学等诸多领域中。在分析属性各异的各类系统时,常常会抽去其具体的含义,而抽象化为理想化的模型,将系统中运动变化的各种物理量统称为信号,将系统中各种量物理量所遵循的各异的物理规律统一视为系统对信号的某种作用或变换。因此,许多实际问题,最终都可以抽象为一个信号与系统的模型。信号与系统宏观地研究信号作用于系统时的变换规律,揭示系统的一般特性,而不关心系统内部的具体细节。

信号的概念与系统的概念是紧密相连的,\xref{fig:信号与系统}所示的,是信号与系统的一个基本模型,系统在输入信号(激励)的驱动下对它进行“加工”和“处理”后变换产生输出信号(响应)。


信号,例如在电子线路中,是随时间变化的电压或电流,因此,从数学观点上看,信号是关于独立变量$t$的函数$f(t)$,这称为一维信号。客观的说,我们并不关心信号中的变量到底是时间还是空间或别的什么东西,不过作为时间的情况是最常见的,故以$t$表示。亦有多维信号的例子,例如在相机的光学成像系统中,CMOS图像传感器的信号是分布于二维平面随时间变化的灰度信息(或许也带有颜色信息,这不重要),即$f(x,y,t)$。本课程仅讨论一维信号。

