\section{连续时间傅里叶级数的表示}

\subsection{连续时间傅里叶级数}\setpeq{连续时间傅里叶级数}
正如\xref{chap:信号与系统}中\xref{def:连续信号的周期},若连续信号是周期的,那对于所有的$t$,存在某个正值的$T$,有
\begin{Equation}
    x(t)=x(t+T)
\end{Equation}
在这里,我们将上述$T$的最小非零值称为\uwave{基波周期},而$\omega_0=2\pi/T$则相应称为\uwave{基波频率}。

在本章中我们所关注的复指数信号也是一个周期信号
\begin{Equation}
    x(t)=\e^{\j\omega_0t}
\end{Equation}

而与之呈\uwave{谐波关系}(Harmonically Related)的复指数信号集就是
\begin{Equation}
    \phi_k(t)=\e^{\j k\omega_0t}\qquad k=0,\pm 1,\pm 2,\cdots 
\end{Equation}
这些信号的每一个都有一个基波频率,它们都是$\omega_0$的整数倍$1\omega_0, 2\omega_0, 3\omega_0, \cdots$,因此,它们的周期相应即为$(T/1), (T/2), (T/3), \cdots$,换言之,这些信号对于$T$来说仍然都是周期的(尽管对于$k\geq 2$的信号,$T$并非基波周期),故一个由成谐波关系的复指数线性组合的信号
\begin{Equation}&[1]
    x(t)=\Sum[k=-\infty][\infty]a_k\e^{\j k\omega_0t}
\end{Equation}
仍然是关于$T$的周期信号。其中,$k=0$的项其实就是一个常数,$k=\pm 1$这两项的基波频率都等于$\omega_0$,两者合在一起称为\uwave{基波分量}(Fundamental Component)或\uwave{一次谐波分量}(First Harmonic Component),$k=\pm 2$的频率则是基波分量的两倍,两者合在一起称为\uwave{二次谐波分量}(Second Harmonic Component)。而一般来说,$k=\pm N$就相应称为$N$次谐波分量。\goodbreak

而一个周期信号表示为\xrefpeq{1}的形式,就称为其傅里叶级数表示。
\begin{BoxDefinition}[连续傅里叶级数]
    若一个频率为$\omega_0$的连续周期信号$x(t)$被表示为以下形式
    \begin{Equation}
        x(t)=\Sum[k=-\infty][\infty]a_k\e^{\j k\omega_0t}
    \end{Equation}
    就称之为为$x(t)$的\uwave{傅里叶级数}(Fourier Series)表示。
\end{BoxDefinition}

我们在微积分等课程中更熟悉的那种傅里叶级数,通常是由$\sin(k\omega_0 t)$和$\cos(k\omega_0 t)$表示的,让我们来看,那种形式是如何从\xref{def:傅里叶级数}产生的。事实上,若$x(t)$是实信号,由于$x(t)=x^{*}(t)$
\begin{Equation}
    x(t)=x^{*}(t)=\Sum[k=-\infty][\infty]a_k^{*}\e^{-\j k\omega_0 t}
\end{Equation}
以$-k$代替$k$并不妨碍结果
\begin{Equation}
    x(t)=\Sum[k=-\infty][\infty]a_{-k}^{*}\e^{\j k\omega_0 t}
\end{Equation}
而将其与\xref{def:连续傅里叶级数}的原始定义比较
\begin{Equation}
    x(t)=\Sum[k=-\infty][\infty]a_k\e^{\j k\omega_0t}
\end{Equation}
我们就可以得到
\begin{Equation}
    a_k^{*}=a_{-k}
\end{Equation}
这是一个重要的性质,即,若$x(t)$是实信号,那其傅里叶系数就满足$a_k^{*}=a_{-k}$,或者说
\begin{itemize}
    \item $a_k$的实部是一个关于$k$的偶函数(序列)。
    \item $a_k$的虚部是要给关于$k$的奇函数(序列)。
\end{itemize}
现在,为了导出傅里叶级数的另外一种形式,我们将\xref{def:连续傅里叶级数}的求和重新写成
\begin{Equation}
    x(t)=a_0+\Sum[k=1][\infty][a_k\e^{\j k\omega_0t}+a_{-k}\e^{-\j k\omega_0t}]
\end{Equation}
由于$x(t)$被假定是实信号,有$a_{k}^{*}=a_{-k}$
\begin{Equation}
    x(t)=a_0+\Sum[k=1][\infty][a_k\e^{\j k\omega_0t}+a_k^{*}\e^{-\j k\omega_0t}]
\end{Equation}
由于括号内的两项互为共轭,重新写作
\begin{Equation}
    x(t)=a_0+\Sum[k=1][\infty]2\Re{a_k\e^{\j k\omega_0t}}
\end{Equation}
若$a_k$以极坐标形式呈现$a_k=A_k\e^{j\theta_k}$,这就有
\begin{Equation}
    x(t)=a_0+2\Sum[k=1][\infty]A_k\cos(k\omega_0t+\theta_k)
\end{Equation}
若$a_k$以直接坐标形式呈现$a_k=B_k+\j C_k$,这就有
\begin{Equation}
    x(t)=a_0+2\Sum[k=1][\infty][B_k\cos(k\omega_0t)-C_k\sin(k\omega_0t)]
\end{Equation}
值得强调的是,这两种三角形式的傅里叶级数,与\xref{def:连续傅里叶级数}中复指数形式的傅里叶级数并不完全等价。三角形式仅适用于$x(t)$为实信号的情况,若$x(t)$为复信号,则只能使用复指数形式。实际上,傅里叶级数的三角形式是最普遍被采用的(也是微积分中最初接触的形式),傅里叶最初工作中使用的傅里叶级数就是三角形式的,但是,复指数形式的傅里叶级数对于我们将讨论的问题来说,却是特别方便,今后我们都将毫无例外的采用复指数形式的傅里叶级数。

\subsection{连续时间傅里叶级数的表示}
现在的问题是,若一个周期信号可以表示为傅里叶级数,那傅里叶级数的系数如何确定呢?
\begin{BoxFormula}[连续傅里叶级数的系数]
    若$x(t)$能表示为傅里叶级数的形式
    \begin{Equation}&[a]
        x(t)=\Sum[k=-\infty][\infty]a_k\e^{\j k\omega_0t}
    \end{Equation}
    那么其中的\uwave{傅里叶系数}(Fourier Series Conefficient)$a_k$就相应为(其中$T=2\pi/\omega_0$)
    \begin{Equation}&[b]
        a_k=\frac{1}{T}\Int[T]x(t)\e^{-\j k\omega_0t}\dd{t}
    \end{Equation}
    \xrefpeq{a}称为\uwave{综合}(Synthesis)公式,\xrefpeq{b}称为\uwave{分析}(Analysis)公式。
\end{BoxFormula}

\begin{Proof}
    假设一个给定的周期信号$x(t)$可以表示为傅里叶级数,根据\fancyref{def:连续傅里叶级数}
    \begin{Equation}&[1]
        x(t)=\Sum[k=-\infty][\infty]a_k\e^{\j k\omega_0 t}
    \end{Equation}
    两边各乘$\e^{-\j n\omega_0t}$,其中$n$是任意整数
    \begin{Equation}&[2]
        x(t)\e^{-\j n\omega_0t}=\Sum[k=-\infty][\infty]a_k\e^{\j k\omega_0t}\e^{-\j n\omega_0t}
    \end{Equation}
    两边从$0$到$T=2\pi/\omega_0$对$t$积分,有
    \begin{Equation}&[3]
        \Int[0][T]x(t)\e^{-\j n\omega_0t}\dd{t}=\Int[0][T]\Sum[k=-\infty][\infty]a_k\e^{\j k\omega_0t}\e^{-\j n\omega_0t}\dd{t}
    \end{Equation}
    交换积分和求和的次序
    \begin{Equation}&[4]
        \Int[0][T]x(t)\e^{-\j n\omega_0t}\dd{t}=\Sum[k=-\infty][\infty]a_k\Int[0][T]\e^{\j(k-n)\omega_0t}\dd{t}
    \end{Equation}
    \xrefpeq{4}右端的积分是很容易的,为此,利用欧拉关系可得
    \begin{Equation}&[5]
        \Int[0][T]\e^{\j(k-n)\omega_0t}\dd{t}=\Int[0][T]\cos[(k-n)\omega_0t]+\j\Int[0][T]\sin[(k-n)\omega_0t]\dd{t}
    \end{Equation}
    对于$k\neq n$,由于$\cos[(k-n)\omega_0t]$和$\sin[(k-n)\omega_0t]$都是关于$T$的周期函数,积分值为零。特别的,对于$k=n$,左端$\cos$的为常数$1$,积分后为$T$,右端$\sin$的积分为$0$,因此
    \begin{Equation}&[6]
        \Int[0][T]\e^{\j(k-n)\omega_0t}\dd{t}=
        \begin{cases}
            T,&k=n\\
            0,&k\neq n
        \end{cases}
    \end{Equation}
    将\xrefpeq{6}代回\xrefpeq{4}
    \begin{Equation}
        \Int[0][T]x(t)\e^{-\j n\omega_0t}\dd{t}=Ta_n
    \end{Equation}
    即
    \begin{Equation}
        a_n=\frac{1}{T}\Int[0][T]x(t)\e^{-\j n\omega_0t}\dd{t}
    \end{Equation}
    将$n$改记为$k$,且$[0,T]$上积分也可以换成在任意周期内积分,故
    \begin{Equation}*
        a_k=\frac{1}{T}\Int[T]x(t)\e^{-\j k\omega_0t}\dd{t}\qedhere
    \end{Equation}
\end{Proof}

现在让我们来实践一下上述结论,如\xref{tab:周期性方波的傅里叶系数}所示,考虑周期性方波,其在一个周期内的定义是
\begin{Equation}
    x(t)=
    \begin{cases}
        1,&|t|<T_1\\
        0,&T_1<|t|<T/2\\
    \end{cases}
\end{Equation}
换言之,作为偶函数定义,方波周期为$T$,方波高电平的一半为$T_1$,即占空比是$2T_1/T$。

首先,对于$k=0$,这实质就是在计算占空比
\begin{Equation}
    a_0=\frac{1}{T}\Int[-T/2][T/2]x(t)\dd{t}=\frac{1}{T}\Int[-T_1][T_1]\dd{t}=\frac{2T_1}{T}
\end{Equation}
随后,对于$k\neq 0$,计算得到
\begin{Equation}
    a_k=\frac{1}{T}\Int[-T/2][T/2]x(t)\e^{-\j k\omega_0t}\dd{t}=\frac{1}{T}\Int[-T_1][T_1]\e^{-\j k\omega_0t\dd{t}}=\eval{-\frac{1}{\j k\omega_0T}\e^{-\j k\omega_0t}}_{-T_1}^{T_1}
\end{Equation}
或重新写作
\begin{Equation}
    a_k=\frac{2}{ k\omega_0T}\qty[\frac{\e^{\j k\omega_0T_1}-\e^{-\j k\omega_0T_1}}{2\j}]
\end{Equation}
即,并考虑到$\omega_0=2\pi/T$
\begin{Equation}
    a_k=\frac{2\sin(k\omega_0T_1)}{k\omega_0T}=\frac{\sin[k\pi(2T_1/T)]}{k\pi}
\end{Equation}
综上,对于这里的方波,其傅里叶系数满足
\begin{Equation}
    a_k=
    \begin{cases}
        2T_1/T,&k=0\\
        \sin[k\pi(2T_1/T)]/k\pi,&k=0\\
    \end{cases}
\end{Equation}
特别的,如果$T=4T_1$,即占空比$2T_1/T=0.5$时,此时的方波高电平和低电平各占一半。

\begin{Table}[周期性方波的傅里叶系数]{cc}
<时域$x(t)$&频域$a_k$\\>
\xcell<c>[2ex][0ex]{\includegraphics{build/Chapter04C_01a.fig.pdf}}&
\xcell<c>[2ex][0ex]{\includegraphics{build/Chapter04C_01d.fig.pdf}}\\
$T=4T_1$&$T=4T_1$\\
\xcell<c>[2ex][0ex]{\includegraphics{build/Chapter04C_01b.fig.pdf}}&
\xcell<c>[2ex][0ex]{\includegraphics{build/Chapter04C_01e.fig.pdf}}\\
$T=8T_1$&$T=8T_1$\\
\xcell<c>[2ex][0ex]{\includegraphics{build/Chapter04C_01c.fig.pdf}}&
\xcell<c>[2ex][0ex]{\includegraphics{build/Chapter04C_01f.fig.pdf}}\\
$T=16T_1$&$T=16T_1$\\
\end{Table}

\xref{tab:周期性方波的傅里叶系数}中,分别就$T=4T_1, T=8T_1, T=16T_1$的方波$x(t)$及其傅里叶系数$a_k$进行了绘图。

需要说明的是,这里$a_k$是整数只是一个巧合,通常$a_k$是复数,需要两张图才能表示。