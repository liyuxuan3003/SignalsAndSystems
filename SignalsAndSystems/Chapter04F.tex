\section{离散时间傅里叶级数的表示}
本节将讨论离散时间周期信号的傅里叶级数表示,虽然这一节与\xref{sec:连续时间傅里叶级数的表示}的讨论完全以并行的方式进行,但是它们之间有一些很重要的区别,特别是
\begin{itemize}
    \item 连续时间周期信号的傅里叶级数,是无穷级数。
    \item 离散时间周期信号的傅里叶级数,是有限项级数。
\end{itemize}
因此,离散傅里叶级数并不存在连续傅里叶级数中的收敛问题(狄利克雷定理和吉布斯现象)。

\subsection{离散时间傅里叶级数}\setpeq{离散时间傅里叶级数}
正如\xref{chap:信号与系统}中\xref{def:离散信号的周期},若离散信号是周期的,那对于所有的$n$,存在某个正值的$N$,有
\begin{Equation}
    x(n)=x(n+N)
\end{Equation}
在这里,我们将上述$N$的最小正整数称为基波周期,而$\omega_0=2\pi/N$则相应称为基波频率。

离散信号最大的不同,就表现在离散复指数信号集中
\begin{Equation}
    \phi_k(n)=\e^{\j k\omega_0n}\qquad
    k=0,\pm 1,\pm 2,\cdots
\end{Equation}
离散复指数信号集中实质上只有$N$个信号是不同的,具体而言
\begin{Equation}
    \phi_k(n)=\phi_{k+rN}(n)
\end{Equation}
换言之,当$k$变化一个$N$的整数倍时,就得到了一个完全一样的序列。因为尽管频率变得更高了,但由于离散信号仅考察一些离散的整数点,在被考察的整数点上,和过去完全一样!

现在我们尝试用$\phi_k(n)$的线性组合表示更一遍的周期序列
\begin{Equation}
    x(n)=\Sum[k=0][N-1]a_k\phi_{k}(n)=\Sum[k=0][N-1]a_k\e^{\j k\omega_0 n}
\end{Equation}
但由于前述原因,实际上除了$k=0, 1, 2, \cdots, N-1$,完全也可以取$k=1, 2, 3, \cdots, N$,只要是任意$N$个$k$的连续取值即可,介于$\phi_k(n)$仅在$k$的$N$个连续取值范围内是不同的。为了表示这种一般性,特将求和限表示为$k=\<N>$的形式,代表任意$N$个连续取值,即
\begin{Equation}&[1]
    x(n)=\Sum[k=\<N>]a_k\phi_{k}(n)=\Sum[k=\<N>]a_k\e^{\j k\omega_0 n}
\end{Equation}
这就是离散信号的傅里叶级数表示。

\begin{BoxDefinition}[离散傅里叶级数]
    若一个频率为$\omega_0$的离散周期信号$x(n)$被表示为以下形式
    \begin{Equation}
        x(n)=\Sum[k=\<N>]a_k\e^{\j k\omega_0 n}
    \end{Equation}
    就称之为离散信号$x(n)$的傅里叶级数表示。
\end{BoxDefinition}

\subsection{离散时间傅里叶级数的表示}\setpeq{离散时间傅里叶级数的表示}
假设一个周期序列$x(n)$,其周期为$N$,现在想确定$x(n)$能否表示为傅里叶级数的形式?如果可以,那么这些系数$a_k$是什么?这个问题实质上就是要求得一组线性联立方程的解。

根据\fancyref{def:离散傅里叶级数}
\begin{Equation}&[1]
    x(n)=\Sum[k=\<N>]a_k\e^{\j k\omega_0n}
\end{Equation}
联立方程组如何建立?依次令$x(n)$对$n=0,1,2,\cdots,N-1$取值
\begin{Equation}&[2]
    \begin{cases}
        x(0)=\Sum[k=\<N>]a_k\\
        x(1)=\Sum[k=\<N>]a_k\e^{\j k\omega_0}\\
        x(2)=\Sum[k=\<N>]a_k\e^{\j 2k\omega_0}\\
        x(3)=\Sum[k=\<N>]a_k\e^{\j 3k\omega_0}\\
        \vdots\\
        x(N-1)=\Sum[k=\<N>]a_k\e^{\j (N-1)k\omega_0}
    \end{cases}
\end{Equation}
这样,\xrefpeq{2}就是一个具有$N$个未知系数$a_k, k=\<N>$的$N$个线性方程。从理论上可以证明,这$N$个方程是线性独立的,因此可以利用已知的$x(n)$值求得系数$a_k$。不过在这里,我们并不计划这么做,而是仍采用与连续时间情况并行的方法进行推导,得到$x(n)$的一个闭式。

\begin{BoxFormula}[离散傅里叶级数的系数]
    若$x(n)$能表示为傅里叶级数的形式
    \begin{Equation}&[a]
        x(n)=\Sum[k=\<N>]a_k\e^{\j k\omega_0 n}
    \end{Equation}
    那么其中的傅里叶系数$a_k$就相应为(其中$N=2\pi/\omega_0$)
    \begin{Equation}&[b]
        a_k=\frac{1}{N}\Sum[n=\<N>]x(n)\e^{-\j k\omega_0n}
    \end{Equation}
    \xrefpeq{a}称为综合公式,\xrefpeq{b}称为分析公式。
\end{BoxFormula}

\begin{Proof}
   假设一个给定的周期序列$x(n)$可以表示为傅里叶级数,根据\fancyref{def:离散傅里叶级数}
   \begin{Equation}&[1]
       x(n)=\Sum[k=\<N>]a_k\e^{\j k\omega_0n}
   \end{Equation}
   两边各乘$\e^{-\j r\omega_0n}$,其中$r$是任意整数
   \begin{Equation}&[2]
       x(n)\e^{-\j r\omega_0n}=\Sum[k=\<N>]a_k\e^{\j k\omega_0 n}\e^{-\j r\omega_0n}
   \end{Equation}
   两边对$n$在任意连续$N$项上求和
   \begin{Equation}&[3]
       \Sum[n=\<N>]x(n)\e^{-\j r\omega_0n}=\Sum[n=\<N>]\Sum[k=\<N>]a_k\e^{\j k\omega_0 n}\e^{-\j r\omega_0n}
   \end{Equation}
   交换右边的求和顺序
   \begin{Equation}&[4]
    \Sum[n=\<N>]x(n)\e^{-\j r\omega_0n}=\Sum[k=\<N>]a_k\Sum[n=\<N>]\\e^{\j(k-r)\omega_0n}
   \end{Equation}
   而一个显然的事实是
   \begin{Equation}&[5]
       \Sum[n=\<N>]\e^{\j k\omega_0n}=
       \begin{cases}
           N,&k=0,\pm N,\pm 2N,\cdots\\
           0,&\text{otherwise}
       \end{cases}
   \end{Equation}
   即周期复指数序列的值在一个完整周期内求和的结果为零,除非该复指数是某一常数\footnote{当$k=0$时$\e^{\j k\omega_0n}=1$为常数,而$k=\pm N,\pm 2N,\cdots$的结果同$k=0$。}。

   在\xrefpeq{4}中应用\xrefpeq{5}的结果,\xrefpeq{4}右边内层对$n$的求和为零,除非$(k-r)$为零或者是$N$的整倍数。简便起见,不妨假设$r$的取值在$k$的变化范围内,所以,此处右侧的内层求和,在$k=r$时为$N$,在$k\neq r$时为$0$,因此右侧的求和结果就演变为$Na_r$,于是有
   \begin{Equation}
       \Sum[n=\<N>]x(n)\e^{-\j r\omega_0n}=Na_r
   \end{Equation}
   即
   \begin{Equation}
       a_r=\frac{1}{N}\Sum[n=\<N>]x(n)\e^{-\j r\omega_0n}
   \end{Equation}
   将$r$改记为$k$,得到
   \begin{Equation}*
       a_k=\frac{1}{N}\Sum[n=\<N>]x(n)\e^{-\j k\omega_0n}\qedhere
   \end{Equation}
\end{Proof}

现在让我们回过来看\fancyref{fml:离散傅里叶级数的系数}
\begin{Equation}
    x(n)=\Sum[k=\<N>]a_k\phi_k(n)
\end{Equation}
若$k$依次取$0,1,2\cdots,N-1$
\begin{Equation}
    x(n)=a_0\phi_0(n)+a_1\phi_1(n)+a_2\phi_2(n)+\cdots+a_{N-1}\phi_{N-1}(n)
\end{Equation}
若$k$依次取$1,2,3,\cdots,N$
\begin{Equation}
    x(n)=a_1\phi_1(n)+a_2\phi_2(n)+a_3\phi_3(n)+\cdots+a_{N}\phi_{N}(n)
\end{Equation}
由于$\phi_0(n)=\phi_N(n)$,因此$a_0=a_N$,更一般的说
\begin{Equation}
    a_k=a_{k+N}
\end{Equation}
即$a_k$和$\phi_k(n)$一样都是以$N$为周期重复的。诚然,傅里叶级数中只需要$N$个连续的$a_k$值就可以了,但有时需要将$a_k$视作定义在全部$k$值上的序列,此时$a_k=a_{k+N}$就很重要了。

应当指出的是,对于一个离散信号$x(n)$,离散傅里叶级数总是存在的,没有收敛问题,没有吉布斯现象。这是由于这样一个事实,任何离散时间周期序列$x(n)$完全是由有限个参数,这就是在一个周期内的$N$个序列值。从这个角度看,分析公式只是把这$N$个参数变为一组等效的$N$个傅里叶系数值,综合公式则告诉我们如何利用一个有限项级数来回复原来的序列值。



