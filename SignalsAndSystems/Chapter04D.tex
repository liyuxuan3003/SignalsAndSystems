\section{连续时间傅里叶级数的收敛}

\subsection{狄利克雷条件}
\fancyref{fml:连续傅里叶级数的系数}中的综合和分析公式在18世纪中叶对于欧拉和拉格朗日来说都是熟悉的,然而他们两人都放弃了这条分析途径,而没有去研究这样一个问题:\empx{究竟多大一类的周期信号可以表示成这种形式}?有趣的是,欧拉和拉格朗日都反对上一节中有关方波的例子,因为方波的$x(t)$是不连续,而每个谐波分量却都是连续的,连续的东西如何组成不连续的东西呢?另一方面,傅里叶研究了同一个例子,并认为方波的傅里叶级数表示也是正确的。不过事实上,傅里叶坚持的是任何周期信号都能用傅里叶级数表示。虽然这一点并不完全正确,但是,傅里叶级数的确能用于表示相当广泛的一类周期信号,包括周期性方波信号。

我们会说,我们都已经写出周期性方波信号的傅里叶系数了,为什么又要纠结周期性方波信号能否用傅里叶级数表示?为了理解这一点,让我们回顾一下目前已经完成的工作
\begin{itemize}
    \item \xref{def:连续傅里叶级数}描述了傅里叶级数的数学形式。
    \item \xref{fml:连续傅里叶级数的系数}描述了\textbf{若一个信号能用傅里叶级数表示},则傅里叶系数应当如何计算。
\end{itemize}
我们容易注意到问题出在哪里,前面推导方波的傅里叶系数的前提是我们已知方波可以用傅里叶级数表示,但问题是,如何确定一个信号能否用傅里叶级数表示?这就是本节要研究的问题。这个问题也可以换一个提法,我们总是可以形式的写出信号$x(t)$对应的傅里叶系数,但是由这些系数构成的傅里叶级数未必仍然收敛至$x(t)$本身,如何判断傅里叶级数是否收敛?



\begin{BoxTheorem}[狄利克雷条件]
    若信号$x(t)$满足以下的\uwave{狄利克雷条件}(Dirichlet Condition)
    \begin{enumerate}
        \item 在任何周期内,$x(t)$必须\uwave{绝对可积}(Absolutely Integrable),即
        \begin{Equation}
            \Int[T]|x(t)|\dd{t}<\infty
        \end{Equation}
        \item 在任意周期内,$x(t)$仅存在有限个极大值和极小值。
        \item 在任意周期内,$x(t)$仅存在有限个不连续点,且在不连续点的两端函数为有限值。
    \end{enumerate}
    那么$x(t)$的傅里叶级数,在连续点收敛至$x(t)$本身,在间断点收敛至左右极限的均值。
\end{BoxTheorem}

\begin{Proof}
    证明从略。
\end{Proof}

在这里,我们不讨论狄利克雷条件在分析上的证明,而是重点考虑一下不满足其条件的例子。\goodbreak

不满足狄利克雷第一条件可以举例如下,如\xref{fig:不满足第一条件},$1/t$在$[0,1]$上的积分并不收敛
\begin{Equation}
    x(t)=\frac{1}{t}\qquad 0<t\leq 1
\end{Equation}

不满足狄利克雷第二条件可以举例如下,如\xref{fig:不满足第二条件},其会出现无穷振荡
\begin{Equation}
    x(t)=\sin(\frac{2\pi}{t})\qquad 0<t\leq 1
\end{Equation}
不满足狄利克雷第三条件可以举例如下,试想函数$x(t)$的周期$T=1$,在每个周期的$[0,1/2]$区间内为$1$,在$[1/2,1/4]$内为$1/2$,在$[1/4,1/8]$内为$1/4$,在$[1/8,1/16]$内为$1/8$,依次类推,即$x(t)=1/2^n, t\in[1/2^n,1/2^{n+1}]$,很明显$x(t)$每个周期内的积分不超过$1$满足第一条件,不存在极值点满足第二条件,但是不连续点的数目却是无穷多个,并不满足第三条件。

\begin{Figure}[不满足狄利克雷条件的函数]
    \begin{FigureSub}[不满足第一条件]
        \includegraphics[scale=0.85]{build/Chapter04D_02a.fig.pdf}
    \end{FigureSub}
    \hspace{0.5cm}
    \begin{FigureSub}[不满足第二条件]
        \includegraphics[scale=0.85]{build/Chapter04D_02b.fig.pdf}
    \end{FigureSub}
\end{Figure}

由此可见,不满足狄利克雷条件的信号,通常都术语比较反常的信号,在实际场合中不会出现。因此,傅里叶级数的收敛问题对于信号与系统将讨论的问题不具有特别重要的意义
\begin{itemize}
    \item 对于一个连续函数,傅里叶级数收敛且在每一点收敛至原来的信号$x(t)$。
    \item 对于一个存在有限个间断点的函数,在连续点上,傅里叶级数仍然收敛至$x(t)$,在孤立的间断点上,傅里叶级数收敛于间断点左右极限的平均值,无论原先间断点的取值或是否定义。由于只是在一些孤立点上存在差异,因此积分意义下,两个信号完全是一致的。
\end{itemize}

\subsection{吉布斯现象}
为了进一步理解对一个不连续点的周期信号,其傅里叶级数是如何收敛的,我们还是回到方波的例子。1898年。美国物理学家迈克尔逊(Albert Michelson)制作了一台谐波分析仪。该仪器可以计算任何周期信号$x(t)$的傅里叶级数截断至$\k=\pm N$后的近似,其中$N$可达到80
\begin{Equation}
    x_N(t)=\Sum[k=-N][N]a_k\e^{\j k\omega_0 t}
\end{Equation}
迈克尔逊\footnote{没错,这位迈克尔逊和“迈克尔逊干涉仪”和“迈克尔逊--莫雷零实验”中的迈克尔逊是同一个人。}用了很多函数来测试他的仪器,结果显示$x_N(t)$和$x(t)$非常一致。然而当他测试到方波信号时,他得到了一个令他吃惊的重要结果!于是他根据这一结果而怀疑他的仪器是否有不完善的地方。他将这一问题写了一封信给著名的数学物理学家吉布斯(Josiah Gibbs)。

迈克尔逊观察到的有趣现象如\xref{tab:方波的吉布斯现象}所示,在不连续点附近,部分和$x_N(t)$呈现的起伏,而且这个起伏的峰值大小似乎不随$N$的增加而下降!吉布斯证明:情况确实是这样,而且也应该是这样,并不是仪器的设计问题。具体而言,若不连续点处的跳变的高度是$1$,则部分和所呈现的峰值最大值约是$1.09$,即在间断点两侧均有$9\%$的超量,并且无论$N$取多大,这个$9\%$的超量不变。这就是\uwave{吉布斯现象}(Gibbs Phenomenon),其含义是,不连续周期信号$x(t)$的傅里叶级数的截断近似$x_N(t)$在接近不连续点时,将呈现高频起伏和$9\%$与$N$无关超量。
\begin{Table}[方波的吉布斯现象]{cc}
    <\mc{2}(c){方波的傅里叶级数的近似$x_N(t)$}\\>
    \xcell<c>[2ex][0ex]{\includegraphics{build/Chapter04D_01a.fig.pdf}}&
    \xcell<c>[2ex][0ex]{\includegraphics{build/Chapter04D_01b.fig.pdf}}\\
    \xcell<c>[2ex][0ex]{\includegraphics{build/Chapter04D_01c.fig.pdf}}&
    \xcell<c>[2ex][0ex]{\includegraphics{build/Chapter04D_01d.fig.pdf}}\\
    \xcell<c>[2ex][0ex]{\includegraphics{build/Chapter04D_01e.fig.pdf}}&
    \xcell<c>[2ex][0ex]{\includegraphics{build/Chapter04D_01f.fig.pdf}}\\
\end{Table}

关于吉布斯现象,首先要认识到的是其与狄利克雷条件之间并不矛盾。尽管存在与$N$无关超量,但是随着$N$的增大,部分和$x_N(t)$的起伏和超量会向不连续处压缩。故部分和$x_N(t)$在连续点仍然会收敛于$x(t)$。关于为何此处存在恒定$9\%$的超量,以下列出了定性解释\cite{zhihu:吉布斯}\cite{enwiki:吉布斯}
\begin{enumerate}
    \item 从部分和$x_N(t)$中$N\to\infty$依狄利克雷条件收敛于$x(t)$来看,首先,狄利克雷定理是正确的,确实收敛。但问题在于,\empx{当存在间断点时,部分和仅逐点收敛而不一致收敛}。依照$\varepsilon$--$N$定义,对于取定的误差$\varepsilon$,如果部分和$x_N(t)$对于每一个$t$的取值,都能在相同的$N$使其于原信号$x(t)$的误差小于$\varepsilon$,那就称为一致收敛,否则就是逐点收敛。该例中,越靠近间断使误差小于$\varepsilon$所需的$N$就越大,无论$N$取多大都无法同时让$x_N(t)$上的每个点与$x(t)$的误差小于$\varepsilon$。因此,无论$N$取多大$x_N(t)$在间断点附近都存在超量。
    \item 从傅里叶系数$a_k$的观点看,方波的傅里叶系数\footnote{这里指的是那种,振幅在$[-1,1]$间且占空比为$0.5$的方波,这在微积分中研究过。}是$1,(1/3),(1/5),(1/7),\cdots$,如果将其视为一个常数项级数,我们会注意到它并不绝对收敛,因为它的衰减速度很慢,和调和级数一致而调和级数并不收敛。事实上,\empx{当存在间断点时,傅里叶系数的衰减速度将很慢,且作为常数项级数并不绝对收敛}。衰减很慢就意味着,每一项都很“重要”,而取部分和$x_N(t)$就相当于截断$N$之后的项,截断了“重要”的项自然就得不到完美的结果。
\end{enumerate}
简而言之,吉布斯现象的本质:当存在间断点,部分和不一致收敛,傅里叶系数不绝对收敛。