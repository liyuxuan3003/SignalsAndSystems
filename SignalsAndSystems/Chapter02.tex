\chapter{连续系统的时域分析}

本章将研究线性时不变(LTI)连续系统的时域分析方法。即对于给定激励,根据描述系统响应与激励之间关系的微分方程求得其响应的方法。由于分析在时域内进行,故称为时域分析。

本章将在用经典法求解微分方程的基础上,讨论其与基于零输入响应和零状态响应的求解方法的差异。随后,引入冲激响应的概念后,我们发现,零状态响应可以表示为冲激响应与激励的卷积,使得零状态响应的求解得到了显著的简化,具有重要意义。因此,卷积积分是时域求解零状态响应的重要方法,卷积积分和冲激响应的的引入使得LTI系统的分析更加简洁。

\section{LTI连续系统的响应}

\subsection{微分方程的经典解}

\section{LTI连续系统的分解特性}
当然,经典解法将$y(t)$拆分为齐次通解$y_\te{h}(t)$和特解$y_\te{p}(t)$的作法是正确且可行的,但是作为信号与系统,有时,我们更愿意将$y(t)$拆分为零输入响应$y_\te{zi}(t)$和零状态响应$y_\te{zs}(t)$来看。

\begin{BoxTheorem}[常系数线性微分方程的分解]
    对于常系数线性非齐次微分方程
    \begin{Equation}
        \Sum[i=0][n]a_if^{(i)}(t)=f(t)
    \end{Equation}
    其解可以表示为两部分
    \begin{Equation}
        y=y_\te{zi}(t)+y_\te{zs}(t)
    \end{Equation}
    其中,$y_\te{zi}(t)$是零输入响应,无激励,有$\zm$初值,且$\zp$初值与$\zm$初值相同
    \begin{Equation}
        \Sum[i=0][n]a_iy_\te{zi}^{(i)}(t)=0\qquad
        y_\te{zi}^{(i)}(\zm)=y_\te{zi}^{(i)}(\zp)
    \end{Equation}
    其中,$y_\te{zs}(t)$是零状态响应,有激励,无$\zm$初值,但$\zp$初值与$\zm$初值不同
    \begin{Equation}
        \Sum[i=0][n]a_iy_\te{zs}^{(i)}(t)=f(t)\qquad
        y_\te{zs}^{(i)}(\zm)=0
    \end{Equation}
\end{BoxTheorem}
我们可能会觉得$y_\te{h}(t),y_\te{p}(t)$与这里的$y_\te{zi}(t),y_\te{zs}(t)$很像,但其实两者是很不同的
\begin{itemize}
    \item $y_\te{h}(t), y_\te{p}(t)$是在解同一个非齐次方程,只不过按照微分方程的求解方法,我们将求解过程分为了两步,$y_\te{h}(t)$给出对应齐次方程的通解,保留待定系数,$y_\te{p}(t)$给出非齐次方程的任意一个特解,而两者相加得到$y(t)$后,再用初值条件定出$y_\te{h}(t)$中的待定系数。
    \item $y_\te{zi}(t), y_\te{zs}(t)$是在解两个独立的方程
    \begin{itemize}
        \item 零输入响应$y_\te{zi}(t)$是在解一个齐次但初值非零(适用$y(t)$的初值)的方程。
        \item 零状态响应$y_\te{zs}(t)$是在解一个非齐次但初值为零的方程。
    \end{itemize}
\end{itemize}
我们之后将主要采用零输入响应$y_\te{zi}(t)$与零状态响应$y_\te{zs}(t)$的这种方法求解。