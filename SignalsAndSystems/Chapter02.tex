\chapter{连续系统的时域分析}

本章将研究线性时不变(LTI)连续系统的时域分析方法。即对于给定激励,根据描述系统响应与激励之间关系的微分方程求得其响应的方法。由于分析在时域内进行,故称为时域分析。

本章将在用经典法求解微分方程的基础上,讨论其与基于零输入响应和零状态响应的求解方法的差异。随后,引入冲激响应的概念后,我们发现,零状态响应可以表示为冲激响应与激励的卷积,使得零状态响应的求解得到了显著的简化,具有重要意义。因此,卷积积分是时域求解零状态响应的重要方法,卷积积分和冲激响应的的引入使得LTI系统的分析更加简洁。

\section{LTI连续系统的经典解法}

\subsection{微分方程}
通常来说,若单输入--单输出系统,激励为$f(t)$,响应为$y(t)$,则描述LTI连续系统激励与响应之间关系的数学模型是$n$阶常系数线性微分方程,它可以写为以下形式
\begin{BoxDefinition}[连续系统的微分方程]
    LTI连续系统可以由以下$n$阶常系数线性微分方程描述
    \begin{Equation}
        \Sum[i=0][n]a_iy^{(i)}(t)=
        \Sum[j=0][m]b_jf^{(j)}(t)
    \end{Equation}
\end{BoxDefinition}

注意到,非齐次项是以激励$f(t)$及激励$f(t)$的导数的形式出现的,然而$f(t)$导数的存在往往会为我们的计算带来困难,不过幸运的是,通过一些简单的变化我们就可以解决这个问题。

\begin{BoxFormula}[连续系统的简化微分方程]*
    LTI连续系统中,若相对响应$y(t)$引入代换函数$y_1(t)$
    \begin{Equation}
        y(t)=\Sum[j=0][m]b_j y_1^{(j)}(t)
    \end{Equation}
    则微分方程可以简化为
    \begin{Equation}
        \Sum[i=0][n]a_iy_1^{(i)}(t)=f(t)
    \end{Equation}
\end{BoxFormula}
\begin{Proof}
    我们已经知道的是\fancyref{def:连续系统的微分方程}
    \begin{Equation}&[1]
        \Sum[i=0][n]a_iy^{(i)}(t)=
        \Sum[j=0][m]b_jf^{(j)}(t)
    \end{Equation}
    我们希望在不改变左端形式的情况下,将右端变成$f(t)$
    \begin{Equation}&[2]
        \Sum[i=0][n]a_iy_1^{(i)}(t)=f(t)
    \end{Equation}
    现在的问题是,$y(t)$怎么用这里新引入的$y_1(t)$表达呢?

    这要运用\fancyref{ppt:LTI系统的微分特性},在\xrefpeq{2}的基础上求导乘常系数,得到
    \begin{Gather}[8pt]
        \Sum[i=0][n]a_i[b_0y_1^{(i)}(t)]=b_0f(t)\\
        \Sum[i=0][n]a_i[b_1y_1^{(i+1)}(t)]=b_1f^{(1)}(t)\\
        \Sum[i=0][n]a_i[b_2y_1^{(i+2)}(t)]=b_2f^{(2)}(t)\\
        \vdots\\
        \Sum[i=0][n]a_i[b_my_1^{(i+m)}(t)]=b_mf^{(m)}(t)
    \end{Gather}
    将以上各式相加
    \begin{Equation}&[3]
        \Sum[i=0][n]a_i\qty[\Sum[j=0][m]b_jy_1^{(j)}]^{(i)}=\Sum[j=0][m]b_jf^{(j)}(t)
    \end{Equation}
    将\xrefpeq{3}对比\xrefpeq{1},即得
    \begin{Equation}*
        y(t)=\Sum[j=0][m]b_jy_1^{(j)}\qedhere
    \end{Equation}
\end{Proof}
之后,以上这两种形式的微分方程我们都会使用,不过后面这种用的多些,也更为重要。

\subsection{微分方程的经典解}
微分方程的经典解法,在微积分的课程中我们已经很熟悉了,下面简要概况结论。

\begin{BoxTheorem}[常系数线性微分方程解的结构]
    对于常系数线性非齐次微分方程
    \begin{Equation}
        \Sum[i=0][n]a_iy^{(i)}(t)=f(t)
    \end{Equation}
    其解可以表示为两部分
    \begin{Equation}
        y(t)=y_\te{h}(t)+y_\te{p}(t)
    \end{Equation}
    其中,$y_\te{p}(t)$是非齐次方程的特解,$y_\te{h}(t)$是对应齐次方程的通解,也称为齐次解。
\end{BoxTheorem}

\subsubsection{通解部分}

通解$y_\te{h}(t)$的形式与激励$f(t)$无关,仅取决于齐次微分方程,即
\begin{Equation}
    \Sum[i=0][n]a_iy_\te{h}^{(i)}(t)=0
\end{Equation}
常考虑以下的特征根方程,即将$y(t)$的几阶导数换成$\lambda$的几次方
\begin{Equation}
    \Sum[i=0][n]a_i\lambda^i=0
\end{Equation}
通解$y_\te{h}(t)$的具体形式,就与以上关于$\lambda$的特征方程根的情况有关,如\xref{tab:微分方程不同特征根对应的通解}。

\begin{Tablex}[微分方程不同特征根对应的通解]{XX}{\linewidth}
<特征根$\lambda$&通解$y_\text{h}(t)$应添加什么项?\\>
对于每一个$r$重实根$\lambda$&\xgp[4pt][-10pt]{$\mal{\e^{\lambda t}\Sum[k=0][r-1]C_k t^{k}}$}\\
对于每一个$r$重共轭复根$\lambda=\alpha\pm\j\beta$&\xgp[6pt][-10pt]{$\mal{\e^{\alpha t}\Sum[k=0][r-1]A_kt^k\cos(\beta t)+B_kt^k\sin(\beta t)}$}\\
\end{Tablex}
    
以二阶微分方程为例
\begin{Equation}
    y_\te{h}''(t)+a_1y_\te{h}'(t)+a_0y_\te{h}(t)=0
\end{Equation}
其对应的特征方程应为
\begin{Equation}
    \lambda^2+a_1\lambda+a_0=0
\end{Equation}
判别式
\begin{Equation}
    \Delta=a_1^2-4a_0
\end{Equation}
当$\Delta=a_1^2-4a_0>0$时,此时特征方程有两个不等的实根$\lambda_1,\lambda_2$
\begin{Equation}
    y(t)=C_1\e^{\lambda_1t}+C_2\e^{\lambda_2t}
\end{Equation}
当$\Delta=a_1^2-4a_0=0$时,此时特征方程有两个相等的实根$\lambda$
\begin{Equation}
    y(t)=[C_1+C_2t]\e^{\lambda t}
\end{Equation}
当$\Delta=a_1^2-4a_0<0$时,此时特征方程有两个共轭的复根$\alpha\pm\j\beta$
\begin{Equation}
    y(t)=[C_1\cos(\beta t)+C_2\sin(\beta t)]\e^{\alpha t}
\end{Equation}
通解$y_\te{h}(t)$中包含待定的常系数,这些常系数暂时还不能确定,需要等求出特解$y_\te{p}(t)$后,将通解和特解相加得到非齐次方程完整的解$y(t)=y_\te{h}(t)+y_\te{p}(t)$,再由$y(t)$的初值条件定出。

\subsubsection{特解部分}

特解$y_\te{p}(t)$则与激励$f(t)$的形式有关,我们目前只能处理特定形式的激励,如\xref{tab:微分方程不同激励所对应的特解}。
\begin{Tablex}[微分方程不同激励所对应的特解]{XlX}
<激励$f(t)$&条件&特解$y_\te{p}(t)$应满足什么形式?\\>
\xgp[4pt][-10pt]{$\mal{\e^{\lambda t}\Sum[k=0][r-1]R_k't^k}$}&$\lambda$是$s$重实根&\xgp[4pt][-10pt]{$\mal{\e^{\lambda t}t^s\Sum[k=0][r-1]R_kt^k}$}\\
\xgp[4pt][-10pt]{$\mal{\e^{\alpha t}\Sum[k=0][r-1]P_k't^k\cos(\beta t)+Q_k't^k\sin(\beta t)}$}&$\lambda=\alpha\pm\j\beta$是$s$重共轭复根&\xgp[4pt][-10pt]{$\mal{\e^{\alpha t}t^s\Sum[k=0][r-1]P_kt^k\cos(\beta t)+Q_kt^k\sin(\beta t)}$}\\
\end{Tablex}
特解$y_\text{p}(t)$中包含的的待定常系数,可以通过将$y_\text{p}(t)$代回微分方程确定,无关初值条件。

再让我们回顾一下完整解的形式,在信号与系统中
\begin{Equation}
    y(t)=y_\te{h}(t)+y_\te{p}(t)
\end{Equation}
\begin{itemize}
    \item 通解$y_\te{h}(t)$常称为\uwave{自由响应}(Nature Response)或\uwave{瞬态响应}(Transient Response)。
    \item 特解$y_\te{p}(t)$常称为\uwave{强迫响应}(Forced Response)或\uwave{稳态响应}(Steady Response)。
\end{itemize}
由以上分析可见,LTI系统的数学模型,常系数线性微分方程的完整解,由齐次通解和特解组成:特解$y_\te{p}(t)$的形式由激励$f(t)$确定,通解$y_\te{h}(t)$的形式则仅仅依赖系统本身的特性,而与激励$f(t)$无关,但需要注意的是,通解$y_\te{h}(t)$中的待定常系数是却是与激励$f(t)$有关的。

\subsection{关于初值的一个重要讨论}
经典解法好像没有任何问题?让我们来看看我们有什么疏漏。试想我们有一个关于$y(t)$的二阶微分方程,非齐次项是$f(t)=\cos(t), t>0$或$f(t)=\e^{-t}, t>0$一类的东西。并且,我们已知初值条件$y(0)$和$y'(0)$。这样按照\xref{subsec:微分方程的经典解}中的流程,应该可以很顺利的解出$y(t)$。但问题出在初值上,很明显,严格的说,这里的初值应当是$y(0_{+})$和$y'(0_{+})$,作为微分方程的练习题,直接给出$y(0_{+})$和$y'(0_{+})$并无大碍。但是,作为实际的应用,我们其实只应该知道激励作用前$y(0_{-})$和$y'(0_{-})$情况,而激励实质上$f(t)=\cos(t)\varepsilon(t)$或$f(t)=\e^{-t}\varepsilon(t)$是包含阶跃函数的,因此,激励作用前后$y(t)$的变化是不连续的,即$y(0_{+})\neq y(0_{-})$以及$y'(0_{+})\neq y'(0_{-})$。

现在的问题就是,如何通过$y(0_{-}), y'(0_{-})$求出$y(0_{+}), y'(0_{+})$?这主要包含两种方法,我们分别称为\uwave{积分法}和\uwave{待定系数法}。作为技巧性的内容,我们下面通过两道例题体会一下其思想。

\begin{BoxExample}[积分法]
    设描述某LTI系统的微分方程为
    \begin{Equation}&[]
        y''(t)+a_1y'(t)+a_0y(t)=b_1f'(t)+b_0f(t)
    \end{Equation}
    已知初值$y(0_{-}), y'(0_{-})$,并且激励是阶跃函数,即$f(t)=\varepsilon(t)$,现求$y(0_{+}), y'(0_{+})$。
\end{BoxExample}

\begin{Proof}
    在\xrefpeq{}中代入$f(t)=\varepsilon(t)$,得到
    \begin{Equation}&[1]
        y''(t)+a_1y'(t)+a_0y(t)=b_1\dirac(t)+b_0\varepsilon(t)
    \end{Equation}
    比较\xrefpeq{1}的等式两端,$y''(t)$应包含$\dirac(t)$,$y'(t)$应包含$\varepsilon(t)$,因而,$y(t)$连续。

    在\xrefpeq{1}两端作$0_{-}$至$0_{+}$的积分
    \begin{Equation}&[2]
        \qquad
        \Int[\zm][\zp]y''(t)\dd{t}+
        a_1\Int[\zm][\zp]y'(t)\dd{t}+
        a_0\Int[\zm][\zp]y(t)\dd{t}=
        b_1\Int[\zm][\zp]\dirac(t)\dd{t}+
        b_0\Int[\zm][\zp]\varepsilon(t)\dd{t}
        \qquad
    \end{Equation}
    由于$y(t)$连续故积分为零,而$\varepsilon(t)$以及最高包含$\varepsilon(t)$的$y'(t)$虽然在$0$处发生阶跃,但是两者在无穷小的区间$[\zm,\zp]$上的积分为仍然为零,因此,上式可以简化为
    \begin{Equation}
        \Int[\zm][\zp]y''(t)\dd{t}=b_1\Int[\zm][\zp]\dirac(t)
    \end{Equation}
    即
    \begin{Equation}
        y'(\zp)-y'(\zm)=b_1
    \end{Equation}
    而由于$y(t)$是连续的
    \begin{Equation}
        y(\zp)-y(\zm)=0
    \end{Equation}
    这就求得了$y(\zp), y'(\zp)$相对$y(\zm), y'(\zm)$的表达式。
\end{Proof}

\begin{BoxExample}[待定系数法]
    设描述某LTI系统的微分方程为
    \begin{Equation}&[]
        y''(t)+a_1y'(t)+a_0y(t)=b_2f''(t)+b_1f'(t)+b_0f(t)
    \end{Equation}
    已知初值$y(0_{-}), y'(0_{-})$,并且激励是冲激函数,即$f(t)=\dirac(t)$,现求$y(0_{+}), y'(0_{+})$。
\end{BoxExample}

\begin{Proof}
    在\xrefpeq{}中代入$f(t)=\dirac(t)$,得到
    \begin{Equation}&[1]
        y''(t)+a_1y'(t)+a_0y(t)=b_2\dirac''(t)+b_1\dirac'(t)+b_0\dirac(t)
    \end{Equation}
    很明显,$y''(t)$最高包含冲激函数的二阶导数$\dirac''(t)$
    \begin{Equation}&[2]
        y''(t)=A\dirac''(t)+B\dirac'(t)+C\dirac(t)+r_2(t)
    \end{Equation}
    $y'(t)$可以表示为
    \begin{Equation}&[3]
        y'(t)=A\dirac'(t)+B\dirac(t)+r_1(t)\qquad r_1(t)=C\varepsilon(t)+\Int[-\infty][t]r_2(t)\dd{t}
    \end{Equation}
    $y(t)$可以表示为
    \begin{Equation}&[4]
        y(t)=A\dirac(t)+r_0(t)\qquad r_0(t)=B\varepsilon(t)+\Int[-\infty][t]r_1(t)\dd{t}
    \end{Equation}
    将\xrefpeq{2},\xrefpeq{3},\xrefpeq{4}相加,得到
    \begin{Split}
        A\dirac''(t)+(a_1A+B)\dirac'(t)&+(a_0A+a_1B+C)\dirac(t)\\ 
        &+[r_2(t)+a_1r_1(t)+a_0r_0(t)]=b_2\dirac''(t)+b_1\dirac'(t)+b_0\dirac(t)
    \end{Split}
    两端冲激函数的各阶导数应当对应系数相等
    \begin{Equation}
        A=b_2\qquad a_1A+B=b_1\qquad a_0A+a_1B+C=b_0
    \end{Equation}
    写成矩阵形式
    \begin{Equation}
        \begin{pmatrix}
            1&0&0\\
            a_1&1&0\\
            a_0&a_1&1\\
        \end{pmatrix}
        \begin{pmatrix}
            A\\
            B\\
            C\\
        \end{pmatrix}
        =
        \begin{pmatrix}
            b_2\\
            b_1\\
            b_0
        \end{pmatrix}
    \end{Equation}
    系数行列式$D$
    \begin{Equation}
        D=\begin{vmatrix}
            1&0&0\\
            a_1&1&0\\
            a_0&a_1&1\\
        \end{vmatrix}=1
    \end{Equation}
    系数$A$
    \begin{Equation}
        A=D_A/D=\begin{vmatrix}
            b_2&0&0\\
            b_1&1&0\\
            b_0&a_1&1\\
        \end{vmatrix}=b_2
    \end{Equation}
    系数$B$
    \begin{Equation}
        B=D_B/D=\begin{vmatrix}
            1&b_2&0\\
            a_1&b_1&0\\
            a_0&b_0&1\\
        \end{vmatrix}=b_1-a_1b_2
    \end{Equation}
    系数$C$
    \begin{Equation}
        C=D_C/D\begin{vmatrix}
            1&0&b_2\\
            a_1&1&b_1\\
            a_0&a_1&b_0\\
        \end{vmatrix}=b_0-b_1a_1-b_2(a_0-a_1^2)
    \end{Equation}
    至此,$A,B,C$这三个待定系数都是已知的了。

    根据\xrefpeq{2}
    \begin{Equation}
        y''(t)=A\dirac''(t)+B\dirac'(t)+C\dirac(t)+r_2(t)
    \end{Equation}
    两端在$[\zm,\zp]$的区间上积分,注意到$\dirac'(t),\dirac''(t)$的积分都是零
    \begin{Equation}
        y'(\zp)-y'(\zm)=C
    \end{Equation}
    根据\xrefpeq{3}
    \begin{Equation}
        y'(t)=A\dirac'(t)+B\dirac(t)+r_1(t)
    \end{Equation}
    两端在$[\zm,\zp]$的区间上积分,注意到$\dirac'(t)$的积分是零
    \begin{Equation}
        y(\zp)-y(\zm)=B
    \end{Equation}
    这就求得了$y(\zp), y'(\zp)$相对$y(\zm), y'(\zm)$的表达式。
\end{Proof}

积分法和待定系数法是由$\zm$值计算$\zp$值的两种不同方法,积分法过程相对较为简单,但仅能适合较简单的情况,待定系数法过程相对较为复杂,但相对通用。积分法用的比较多些。
\section{LTI连续系统的分解特性}
当然,经典解法将$y(t)$拆分为齐次通解$y_\te{h}(t)$和特解$y_\te{p}(t)$的作法是正确且可行的,但是作为信号与系统,有时,我们更愿意将$y(t)$拆分为零输入响应$y_\te{zi}(t)$和零状态响应$y_\te{zs}(t)$来看。

\begin{BoxTheorem}[常系数线性微分方程的分解]
    对于常系数线性非齐次微分方程
    \begin{Equation}
        \Sum[i=0][n]a_iy^{(i)}(t)=f(t)
    \end{Equation}
    其解可以表示为两部分
    \begin{Equation}
        y(t)=y_\te{zi}(t)+y_\te{zs}(t)
    \end{Equation}
    其中,$y_\te{zi}(t)$是零输入响应,无激励,有$\zm$初值,且$\zp$初值与$\zm$初值相同
    \begin{Equation}
        \Sum[i=0][n]a_iy_\te{zi}^{(i)}(t)=0\qquad
        y_\te{zi}^{(i)}(\zm)=y^{(i)}(\zm)=y_\te{zi}^{(i)}(\zp)
    \end{Equation}
    其中,$y_\te{zs}(t)$是零状态响应,有激励,无$\zm$初值,但$\zp$初值与$\zm$初值不同
    \begin{Equation}
        \Sum[i=0][n]a_iy_\te{zs}^{(i)}(t)=f(t)\qquad
        y_\te{zs}^{(i)}(\zm)=0
    \end{Equation}
\end{BoxTheorem}
我们可能会觉得$y_\te{h}(t),y_\te{p}(t)$与这里的$y_\te{zi}(t),y_\te{zs}(t)$很像,但其实两者是很不同的
\begin{itemize}
    \item $y_\te{h}(t), y_\te{p}(t)$是在解同一个非齐次方程,只不过按照微分方程的求解方法,我们将求解过程分为了两步,$y_\te{h}(t)$给出对应齐次方程的通解,保留待定系数,$y_\te{p}(t)$给出非齐次方程的任意一个特解,而两者相加得到$y(t)$后,再用初值条件定出$y_\te{h}(t)$中的待定系数。
    \item $y_\te{zi}(t), y_\te{zs}(t)$是在解两个独立的方程
    \begin{itemize}
        \item 零输入响应$y_\te{zi}(t)$是在解一个齐次但初值非零(适用$y(t)$的初值)的方程。
        \item 零状态响应$y_\te{zs}(t)$是在解一个非齐次但初值为零的方程。
    \end{itemize}
\end{itemize}
我们之后将主要采用零输入响应$y_\te{zi}(t)$与零状态响应$y_\te{zs}(t)$的这种方法求解
\section{连续系统的冲激响应和阶跃响应}
零输入响应总是容易求解的,这是一个齐次方程,依\xref{tab:不同特征根对应的通解}求出通解后依初值定出系数即可
\begin{Equation}
    \Sum[i=0][n]a_iy_\te{zi}^{(i)}(t)=0\qquad
        y_\te{zi}^{(i)}(\zm)=y_\te{zi}^{(i)}(\zp)
\end{Equation}

零状态响应的求解就会有些麻烦,这是一个包含激励$f(t)$的非齐次方程,而非齐次方程我们只能对$f(t)$具有\xref{tab:不同激励所对应的特解}所列的特定形式进行求解,具有局限性,且过程较为复杂,需要先求相应齐次方程的通解,再写出一个非齐次方程的特解,最后还需要依据通过\xref{subsec:关于初值的一个重要讨论}中提到的积分法或待定系数法从$y_\te{zs}^{(i)}(\zm)=0$求出$y_\te{zs}^{(i)}(\zp)$的值,作为初值条件定出通解中的系数
\begin{Equation}
    \Sum[i=0][n]a_iy_\te{zs}^{(i)}(t)=f(t)\qquad
        y_\te{zs}^{(i)}(\zm)=0
\end{Equation}

因此,现在面临的问题就是,我们能否找到一种更简单的求解零状态响应的方法?而令人喜悦的是,这种方法确实存在!事实是,关于零状态响应,我们只需要讨论系统在冲激函数激励下的响应就够了,即冲激响应。而激励为任意函数$f(t)$的零状态响应均可以化归为冲激响应。

在本节,我们先来讨论如何计算冲激响应,在下一节再讨论激励为任意函数的情况。

\subsection{冲激响应}
\begin{BoxDefinition}[冲激响应]
    \uwave{冲激响应}(Impulse Response)是指激励为冲激函数$\dirac(t)$
    时的零状态响应
    \begin{Equation}
        h(t)=T[\qty{0},\dirac(t)]
    \end{Equation}
    即冲激响应是以下微分方程的解
    \begin{Equation}
        \Sum[i=0][n]a_ih^{(i)}(t)=\dirac(t)\qquad
        h^{(i)}(\zm)=0
    \end{Equation}
\end{BoxDefinition}

现在的问题是,我们如何务实的求出冲激响应
\begin{BoxFormula}[冲激响应]
    冲激响应$h(t)$可以表达为以下形式
    \begin{Equation}
        h(t)=h_0(t)\varepsilon(t)
    \end{Equation}
    其中$h_0(t)$满足以下微分方程
    \begin{Equation}
        \Sum[i=0][n]a_ih_0^{(i)}(t)=0\qquad t>0
    \end{Equation}
    其初值条件为
    \begin{Equation}
        h_0^{(n-1)}(\zp)=1\qquad
        h_0^{(i)}(\zp)=0\quad i=1,2,\cdots,n-2
    \end{Equation}
\end{BoxFormula}

\begin{Proof}
    根据\fancyref{def:冲激响应}
    \begin{Equation}&[1]
        \Sum[i=0][n]a_ih^{(i)}(t)=\dirac(t)
    \end{Equation}
    因此,$h^{(n)}(t)$包含$\dirac(t)$,$h^{(n-1)}(t)$包含$\varepsilon(t)$,$h^{(n-2)}(t)$及更低阶的导数均连续。

    运用\fancyref{exp:积分法}的思想,就\xrefpeq{1}两端作$[\zm,\zp]$的积分
    \begin{Equation}
        \Int[\zm][\zp]h^{(n)}(t)\dd{t}=\Int[\zm][\zp]\dirac(t)\dd{t}
    \end{Equation}
    容易得到
    \begin{Equation}
        h^{(n-1)}(\zp)-h^{(n-1)}(\zm)=1
    \end{Equation}
    而我们已知$h^{(i)}(\zm)=0, i=1,2,\cdots,n-1$
    \begin{Equation}
        h^{(n-1)}(\zp)=1
    \end{Equation}
    而由于比$n-1$更低阶的导数连续,因此
    \begin{Equation}
        h^{i}(\zp)=0\qquad i=1,2,\cdots,n-2
    \end{Equation}
    现在我们就求得了$h(t)$在$0^{+}$的初值条件了,那如何进一步解出$h(t)$呢?由于$h(t)$是零状态响应,而冲激出现在$t=0$处,因此$h(t)$必然可以表达为$h(t)=h_0(t)\varepsilon(t)$,而$h_0(t)$适用
    \begin{Equation}
        \Sum[i=0][n]a_ih_0^{(i)}(t)=0\qquad t>0
    \end{Equation}
    这是因为当$t>0$时有$h(t)=h_0(t)$且$\dirac(t)=0$,同时,$h_0(t)$的初值条件与$h(t)$也一致
    \begin{Equation}
        h_0^{(n-1)}(\zp)=1\qquad
        h_0^{(i)}(\zp)=0\quad i=1,2,\cdots,n-2
    \end{Equation}
    至此,我们就将冲激响应的求解,转化为了一个具有特定初值条件的齐次方程的求解。
\end{Proof}

\subsection{阶跃响应}
\begin{BoxDefinition}[阶跃响应]
    阶跃响应(Step Response)是指激励为阶跃函数$\varepsilon(t)$时的零状态响应
    \begin{Equation}
        g(t)=T\qty[{0},\varepsilon(t)]
    \end{Equation}
    即阶跃响应是以下微分方程的解
    \begin{Equation}
        \Sum[i=0][n]a_ig^{(i)}(t)=\varepsilon(t)\qquad
        g^{(i)}(\zm)=0
    \end{Equation}
\end{BoxDefinition}

阶跃响应的求解并不需要新的方法,通过\fancyref{fml:阶跃函数的导数},我们知道$\varepsilon(t)$可以视为$\dirac(t)$的积分,而依据\fancyref{ppt:LTI系统的积分特性},某个函数积分的零状态响应当等于该函数零状态响应的积分。而在这里,$\dirac(t)$的零状态响应是$h(t)$,$\dirac(t)$的积分$\varepsilon(t)$的零状态响应$g(t)$就也是$h(t)$的积分。故计算阶跃响应$g(t)$只需将冲激响应$h(t)$积分即可。

\begin{BoxFormula}[阶跃响应]
    阶跃响应$g(t)$可以表示为冲激响应$h(t)$的积分
    \begin{Equation}
        g(t)=\Int[-\infty][t]h(\tau)\dd{\tau}
    \end{Equation}
\end{BoxFormula}

\subsection{二阶系统的冲激响应和激励响应}
二阶系统是经常遇到的一类典型LTI系统,其微分方程为
\begin{Equation}[二阶系统的微分方程]
    y''(t)+2\alpha y'(t)+\omega_0^2y(t)=\omega_0^2f(t)
\end{Equation}
二阶系统的冲激响应和即阶跃响应,如\xref{tab:二阶系统的冲激响应和激励响应}所示
\begin{Tablex}[二阶系统的冲激响应和激励响应]{Xll}
<类型&冲激响应$h(t)$&阶跃响应$g(t)$\\>
过阻尼$\alpha>\omega_0$&
\xgp[3ex]{$\mal{h(t)=\frac{\omega_0^2}{\beta}\e^{-\alpha t}\sinh(\beta t)\varepsilon(t)}$}&
\xgp[3ex]{$\mal{g(t)=\frac{\omega_0^2}{\beta}\qty[\frac{\beta-\e^{-\alpha t}[\beta\cosh(\beta t)+\alpha\sinh(\beta t)]}{\alpha^2-\beta^2}]\varepsilon(t)}$}\\
\mc{3}(c){\xgp[0ex][3ex]{$\beta=\sqrt{\alpha^2-\omega_0^2}$}}\\ \hlinelig
欠阻尼$\alpha<\omega_0$&
\xgp[4ex]{$\mal{h(t)=\frac{\omega_0^2}{\beta}\e^{-\alpha t}\sin(\beta t)\varepsilon(t)}$}&
\xgp[4ex]{$\mal{g(t)=\frac{\omega_0^2}{\beta}\qty[\frac{\beta-\e^{-\alpha t}[\beta\cos(\beta t)+\alpha\sin(\beta t)]}{\alpha^2+\beta^2}]\varepsilon(t)}$}\\
\mc{3}(c){\xgp[0ex][3ex]{$\beta=\sqrt{\omega_0^2-\alpha^2}$}}\\ \hlinelig
临界阻尼$\alpha=\omega_0$&
\xgp[4ex]{$\mal{h(t)=\omega_0^2\e^{-\alpha t}t\varepsilon(t)}$}&
\xgp[4ex]{$\mal{g(t)=\qty[{1-\e^{-\alpha t}(1+\alpha t)}]\varepsilon(t)}$}\\
\end{Tablex}

现在我们来证明\xref{tab:二阶系统的冲激响应和激励响应}给出的结论。

\begin{Proof}[\xref{tab:二阶系统的冲激响应和激励响应}]
    根据\xrefeq{二阶系统的微分方程},二阶系统适用以下微分方程
    \begin{Equation}&[1]
        y''(t)+2\alpha y'(t)+\omega_0^2y(t)=\omega_0^2f(t)
    \end{Equation}
    根据\fancyref{fml:冲激响应},若要求冲激响应$h(t)$,先解$h_0(t)$
    \begin{Equation}&[2]
        h_0''(t)+2\alpha h_0'(t)+\omega_0^2h_0(t)=0
    \end{Equation}
    初值条件
    \begin{Equation}&[3]
        h_0'(\zp)=1\qquad h_0(\zp)=0
    \end{Equation}
    这里$h(t)$与$h_0(t)$的关系是
    \begin{Equation}&[4]
        h(t)=\omega_0^2h_0(t)\varepsilon(t)
    \end{Equation}
    这里多了一个$\omega_0^2$的缘故是\xrefpeq{1}右端的非齐次项是$\omega_0^2f(t)$而不是$f(t)$。\goodbreak

    \xrefpeq{2}对应的特征方程为
    \begin{Equation}&[4]
        \lambda^2+2\alpha\lambda+\omega_0^2=0
    \end{Equation}
    解得
    \begin{Equation}&[5]
        \lambda=\frac{-2\alpha\pm\sqrt{4\alpha^2-4\omega_0^2}}{2}=-\alpha\pm\sqrt{\alpha^2-\omega_0^2}
    \end{Equation}
    接下来,我们将根据特征方程根的情况,对$h_0(t)$的解的形式进行讨论。

    \paragraph{过阻尼情形}
    若$\alpha>\omega_0$,此时$\Delta>0$,具有两个不等的实根,记$\beta=\sqrt{\alpha^2-\omega_0^2}$则有
    \begin{Equation}&[6]
        \lambda=-\alpha\pm\beta
    \end{Equation}
    依据\xref{tab:不同特征根对应的通解},此时的通解为
    \begin{Equation}&[7]
        h_0(t)=C_1\e^{-(\alpha-\beta)t}+C_2\e^{-(\alpha+\beta)t}
    \end{Equation}
    由$h_0(\zp)=0$容易得到
    \begin{Equation}&[8]
        C_1+C_2=0
    \end{Equation}
    由$h_0'(\zp)=1$容易得到
    \begin{Equation}&[9]
        -(\alpha-\beta)C_1-(\alpha+\beta)C_2=1
    \end{Equation}
    将\xrefpeq{8}改写为$C_2=-C_1$代入\xrefpeq{9}
    \begin{Equation}&[10]
        -(\alpha-\beta)C_1+(\alpha+\beta)C_1=1
    \end{Equation}
    即
    \begin{Equation}&[11]
        2\beta C_1=1\qquad C_1=\frac{1}{2\beta}\qquad C_2=-\frac{1}{2\beta}
    \end{Equation} 
    将\xrefpeq{11}代入\xrefpeq{7}
    \begin{Equation}&[12]
        h_0(t)=\frac{1}{2\beta}\qty[\e^{-(\alpha-\beta)t}-\e^{-(\alpha+\beta)t}]
    \end{Equation}
    或者
    \begin{Equation}&[13]
        h_0(t)=\frac{1}{2\beta}\qty[\e^{-\alpha t}\qty(\e^{\beta t}-\e^{-\beta t})]
    \end{Equation}
    改用双曲正弦的表示
    \begin{Equation}&[14]
        h_0(t)=\frac{1}{\beta}\qty[\e^{-\alpha t}\sinh\beta t]
    \end{Equation}
    将\xrefpeq{14}代入\xrefpeq{4}
    \begin{Equation}&[15]
        h(t)=\frac{\omega_0^2}{\beta}\e^{-\alpha t}\sinh(\beta t)\varepsilon(t)
    \end{Equation}
    至此,我们求得了过阻尼情形下的冲激响应。

    \paragraph{欠阻尼情形}
    若$\alpha<\omega_0$,此时$\Delta<0$,具有两个共轭的复根,记$\beta=\sqrt{\omega_0^2-\alpha^2}$,则有
    \begin{Equation}&[16]
        \lambda=-\alpha\pm\j\beta
    \end{Equation}
    依据\xref{tab:不同特征根对应的通解},此时的通解为
    \begin{Equation}&[17]
        h_0(t)=[C_1\cos\beta t+C_2\sin\beta t]\e^{-\alpha t}
    \end{Equation}
    由$h_0(\zp)=0$容易得到
    \begin{Equation}&[18]
        C_1=0
    \end{Equation}
    此时
    \begin{Equation}&[19]
        \qquad\qquad\qquad
        h_0(t)=C_2\sin\beta t\e^{-\alpha t}\qquad
        h_0'(t)=C_2[\beta\e^{-\alpha t}\cos\beta t-\alpha\e^{-\alpha t}\sin\beta t]
        \qquad\qquad\qquad
    \end{Equation}
    由$h_0'(\zp)=1$容易得到
    \begin{Equation}&[20]
        C_2\beta=1\qquad C_2=\frac{1}{\beta}
    \end{Equation}
    这样一来
    \begin{Equation}&[21]
        h_0(t)=\frac{1}{\beta}\e^{-\alpha t}\sin\beta t
    \end{Equation}
    将\xrefpeq{21}代入\xrefpeq{4}
    \begin{Equation}&[22]
        h(t)=\frac{\omega_0^2}{\beta}\e^{-\alpha t}\sin(\beta t)\varepsilon(t)
    \end{Equation}
    至此,我们求得了欠阻尼情形下的冲激响应。

    \paragraph{临界阻尼情形}
    若$\alpha=\omega_0$,此时$\Delta=0$,具有两个相等的实根,即
    \begin{Equation}&[23]
        \lambda=-\alpha
    \end{Equation}
    依据\xref{tab:不同特征根对应的通解},此时的通解为
    \begin{Equation}&[24]
        h_0(t)=[C_1t+C_2]\e^{-\alpha t}
    \end{Equation}
    由$h_0(\zp)$容易得到
    \begin{Equation}&[25]
        C_2=0
    \end{Equation}
    由$h_0(\zm)$容易得到
    \begin{Equation}&[26]
        C_1=1
    \end{Equation}
    这样一来
    \begin{Equation}&[27]
        h_0(t)=t\e^{-\alpha t}
    \end{Equation}
    将\xrefpeq{27}代入\xrefpeq{4}
    \begin{Equation}&[28]
        h(t)=\omega_0^2\e^{-\alpha t}t\varepsilon(t)
    \end{Equation}
    至此,我们就求得了临界阻尼情形下的冲激响应。

    \paragraph{计算阶跃响应}
    根据\fancyref{fml:阶跃响应},阶跃响应$g(t)$是冲激响应$f(t)$的积分
    \begin{Equation}
        g(t)=\Int[-\infty][t]h(\tau)\dd{\tau}
    \end{Equation}
    代入\xrefpeq{15}, \xrefpeq{22}, \xrefpeq{28},运用Mathematica软件,得到积分为
    \begin{Align}[6pt]
        g(t)&=\Int[-\infty][t]\frac{\omega_0^2}{\beta}\e^{-\alpha \tau}\sinh(\beta\tau)\varepsilon(t)\dd{\tau}=\frac{\omega_0^2}{\beta}\qty[\frac{\beta-\e^{-\alpha t}[\beta\cosh(\beta t)+\alpha\sinh(\beta t)]}{\alpha^2-\beta^2}]\varepsilon(t)
        && \alpha>\omega_0\\
        g(t)&=\Int[-\infty][t]\frac{\omega_0^2}{\beta}\e^{-\alpha\tau}\sin(\beta\tau)\varepsilon(\tau)\dd{\tau}=\frac{\omega_0^2}{\beta}\qty[\frac{\beta-\e^{-\alpha t}[\beta\cos(\beta t)+\alpha\sin(\beta t)]}{\alpha^2+\beta^2}]\varepsilon(t)
        && \alpha<\omega_0\\
        g(t)&=\Int[-\infty][t]\omega_0^2\e^{-\alpha \tau}\tau\varepsilon(\tau)\dd{\tau}=[1-\e^{-\alpha t}(1+\alpha t)]\varepsilon(t)
        && \alpha=\omega_0
    \end{Align}
    这样就求得了三种情况下的阶跃响应$g(t)$。
\end{Proof}

二阶系统的冲激响应和阶跃响应的如\xref{fig:二阶系统的冲激响应和阶跃响应}所示

\begin{Figure}[二阶系统的冲激响应和阶跃响应]
    \begin{FigureSub}[二阶系统的冲激响应]
        \includegraphics[width=0.48\linewidth]{build/Chapter02B_01a.fig.pdf}
    \end{FigureSub}
    \begin{FigureSub}[二阶系统的阶跃响应]
        \includegraphics[width=0.48\linewidth]{build/Chapter02B_01b.fig.pdf}
    \end{FigureSub}
\end{Figure}

由此可见,临界阻尼$\alpha=\omega_0$是系统响应是否发生振荡的临界点(\xref{fig:二阶系统的冲激响应和阶跃响应}中取$\omega=1$)。