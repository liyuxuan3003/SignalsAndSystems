\section{零状态响应与卷积和}
本节的思路与\xref{sec:零状态响应与卷积积分}是对应的
\begin{itemize}
    \item 连续系统中,任意激励的零状态响应是冲激响应与激励的卷积积分。
    \item 离散系统中,任意激励的零状态响应是单位序列响应与激励的卷积和。
\end{itemize}

\subsection{离散系统任意激励的零状态响应}
\begin{BoxFormula}[离散系统任意激励的零状态响应]*
    设激励$f(k)$的零状态响应为$y_\te{zs}(k)$,即
    \begin{Equation}
        \Sum[i=0][n]a_{n-i}y_\te{zs}(k-i)=f(k)\qquad
        y_\te{zs}(-i)=0\qquad i=1,2,\cdots,n
    \end{Equation}
    若$h(k)$为该系统的单位序列响应,即
    \begin{Equation}
        y_\te{zs}(k)=\Sum[i=-\infty][\infty]f(i)h(k-i)
    \end{Equation}
    该运算称为卷积和,亦记为
    \begin{Equation}
        y_\te{zs}(k)=f(k)*h(k)
    \end{Equation}
\end{BoxFormula}

\begin{Proof}
    类似于连续情形,任何序列$f(k)$都可以通过单位序列$\dirac(k)$表示
    \begin{Equation}
        f(k)=\Sum[i=-\infty][\infty]f(i)\dirac(i-k)
    \end{Equation}
    而根据\fancyref{def:单位序列},很明显$\dirac(i-k)=\dirac(k-i)$
    \begin{Equation}
        f(k)=\Sum[i=-\infty][\infty]f(i)\dirac(k-i)
    \end{Equation}
    将$f(k)$和$\dirac(k-i)$分别换为其零状态响应$y_\te{zs}(k), h(k-i)$
    \begin{Equation}*
        y_\te{zs}(k)=\Sum[i=-\infty][\infty]f(i)h(k-i)\qedhere
    \end{Equation}
\end{Proof}

\subsection{卷积和}
\begin{BoxDefinition}[卷积和]
    \uwave{卷积和}(Convolution Sum)定义为
    \begin{Equation}
        f_1(k)*f_2(k)=\Sum[i=-\infty][\infty]f_1(i)f_2(k-i)
    \end{Equation}
\end{BoxDefinition}

容易证明,卷积和与卷积积分一样,满足交换律和结合律等性质。