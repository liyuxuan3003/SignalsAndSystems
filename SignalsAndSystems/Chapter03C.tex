\section{离散系统的单位序列响应和阶跃序列响应}
在连续系统中,我们定义有冲激函数$\dirac(t)$和阶跃函数$\varepsilon(t)$的。这些概念可以推广到离散情形。

\subsection{单位序列和阶跃序列}
\begin{BoxDefinition}[单位序列]
    \uwave{单位序列}(Unit Sequence)定义为
    \begin{Equation}
        \dirac(k)=
        \begin{cases}
            1,&k=0\\
            0,&k\neq 0
        \end{cases}
    \end{Equation}
\end{BoxDefinition}

\begin{BoxDefinition}[阶跃序列]
    \uwave{阶跃序列}(Step Sequence)定义为
    \begin{Equation}
        \varepsilon(k)=
        \begin{cases}
            1,&k\geq 0\\
            0,&k<0
        \end{cases}
    \end{Equation}
\end{BoxDefinition}

单位序列和阶跃序列$\dirac(k), \varepsilon(k)$的图像如\xref{fig:单位序列和阶跃序列}所示
\begin{Figure}[单位序列和阶跃序列]
    \begin{FigureSub}[单位序列]
        \includegraphics{build/Chapter03C_01a.fig.pdf}
    \end{FigureSub}
    \hspace{0.25cm}
    \begin{FigureSub}[阶跃序列]
        \includegraphics{build/Chapter03C_01b.fig.pdf}
    \end{FigureSub}
\end{Figure}
单位序列和阶跃序列$\dirac(k), \varepsilon(k)$可以视为冲激函数和阶跃函数$\dirac(t), \varepsilon(t)$的对应离散版本,但仍有一些细节值得关注。首先,冲激函数$\dirac(t)$是一个在$t=0$处脉冲宽度趋于零而高度趋于无穷大的信号,需要通过广义函数的定义,是较为复杂的,单位序列$\dirac(k)$在$k=0$处则简单的定义为$1$。其次,$\varepsilon(t)$在$t=0$的跃变处通常不作定义,$\varepsilon(k)$在$k=0$处则被定义为$1$。\goodbreak

类似于$\dirac(t)$是$\varepsilon(t)$的导数,这里$\dirac(k)$也是$\varepsilon(k)$的差分,有以下结论。\nopagebreak
\begin{BoxFormula}[阶跃序列与单位序列]
    阶跃序列的差分给出单位序列
    \begin{Equation}
        \dirac(k)=\nabla\varepsilon(k)=\varepsilon(k)-\varepsilon(k-1)
    \end{Equation}
    阶跃序列反过来可以由单位序列的求和表示
    \begin{Equation}
        \varepsilon(k)=\Sum[i=-\infty][\infty]\dirac(i)
    \end{Equation}
\end{BoxFormula}

\subsection{单位序列响应}
\begin{BoxDefinition}[单位序列响应]
    \uwave{单位序列响应}(Unit Sequence Response)是指激励为单位序列$\dirac(k)$时的零状态响应
    \begin{Equation}
        h(k)=T[\qty{0},\dirac(k)]
    \end{Equation}
    即单位序列响应是以下差分方程的解
    \begin{Equation}
        \Sum[i=0][n]a_{n-i}h(k-i)=\dirac(k)\qquad h(-i)=0\qquad i=1,2,\cdots,n
    \end{Equation}
\end{BoxDefinition}

\begin{BoxFormula}[单位序列响应]
    单位序列响应$h(k)$可以表达为以下形式
    \begin{Equation}
        h(k)=h_0(k)\varepsilon(k)
    \end{Equation}
    其中$h_0(k)$满足以下差分方程
    \begin{Equation}&[b]
        \Sum[i=0][n]a_{n-i}h_0(k-i)=0\qquad k\geq 0
    \end{Equation}
    其初值条件为
    \begin{Equation}
        h_0(0)=\frac{1}{a_n}
    \end{Equation}
    而$h_0(1), h_0(2), \cdots, h_0(n-1)$可以通过\xrefpeq{b}递推得到(在$k$取负值时$h_0(k)$为零)。
\end{BoxFormula}
离散系统中的单位序列响应与连续系统中的冲激响应,具有相同的重要地位。

\subsection{阶跃序列响应}
\begin{BoxDefinition}[阶跃序列响应]*
    阶跃序列响应(Step Sequence Response)是指激励为阶跃序列$\varepsilon(k)$时的零状态响应
    \begin{Equation}
        g(k)=T[\qty{0},\varepsilon(k)]
    \end{Equation}
    即阶跃序列响应是以下差分方程的解
    \begin{Equation}
        \Sum[i=0][n]a_{n-i}g(k-i)=\varepsilon(k)\qquad
        g(-i)=0\qquad
        i=1,2,\cdots,n
    \end{Equation}
\end{BoxDefinition}

\begin{BoxFormula}[阶跃序列响应]
    阶跃序列响应$g(k)$可以表示为单位序列响应$h(k)$的求和
    \begin{Equation}
        g(k)=\Sum[i=-\infty][k]h(i)
    \end{Equation}
\end{BoxFormula}