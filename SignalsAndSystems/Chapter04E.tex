\section{连续傅里叶级数的性质}

傅里叶级数具有一系列重要的性质,这些性质对于再概念上深入理解这样的表示是很有用的。并且,它们还有助于简化求取很多信号傅里叶级数的复杂性。在\xref{tab:连续傅里叶级数的性质}中罗列出了这些性质。

\begin{Tablex}[连续傅里叶级数的性质]{XXX}
<性质&时域&频域\\>
    线性&$Ax(t)+By(t)$&$Aa_k+Bb_k$\\
    时移&$x(t-t_0)$&$a_k\e^{-\j k\omega_0t_0}$\\
    频移&$x(t)\e^{\j M\omega_0 t}$&$a_{k-M}$\\
    共轭&$x^{*}(t)$&$a^{*}_{-k}$\\
    时间反转&$x(-t)$&$a_{-k}$\\
    时域尺度变换&$x(\alpha t), \alpha>0$&$a_k$\\
    周期卷积&$x(t)*y(t)=\Int[T]x(\tau) y(t-\tau)\dd{\tau}$&$a_kb_kT$\\
    相乘&$x(t)y(t)$&$a_k*b_k=\Sum[l=-\infty][\infty]a_lb_{k-l}$\\
    微分&$\dv*{x(t)}{t}$&$(\j k\omega_0)a_k$\\
    积分&$\Int[-\infty][t]x(t)\dd{t}$&$(\j k\omega_0)^{-1}a_k$\\
    实信号&$x(t)$为实信号&\xcell<l>[2ex][0ex]{
    $\begin{cases}
        a_k=a_{-k}^{*}\\
        \Re{a_k}=\Re{a_{-k}}\\
        \Im{a_k}=-\Im{a_{-k}}\\
        |a_k|=|a_{-k}|\\
        \arg a_k=-\arg a_{-k}
    \end{cases}$}\\
    实偶信号&$x(t)$为实偶信号&$a_k$仅有实部,且实部为偶函数\\
    实奇函数&$x(t)$为奇偶信号&$a_k$仅有虚部,且虚部为奇函数\\
    帕塞瓦尔定理&\mc{2}{$(1/T)\Int[T]|x(t)|^2\dd{t}=\Sum[k=-\infty][\infty]|a_k|^2$}\\
\end{Tablex}

傅里叶级数的大部分性质可以从傅里叶变换中推导出来,所以本节中不再对此作更多说明。