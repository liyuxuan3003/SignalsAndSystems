\section{信号的基本运算}
在系统分析中,常遇到信号的某些基本运算,这包含:加、乘、反转、平移、尺度变换。

\subsection{加法和乘法}
\begin{BoxDefinition}[信号的和]
    信号$f_1(\cdot)$与$f_2(\cdot)$之和,是指同一瞬时两信号值对应相加构成的\uwave{和信号}
    \begin{Equation}
        f(\cdot)=f_1(\cdot)+f_2(\cdot)
    \end{Equation}
\end{BoxDefinition}

\begin{BoxDefinition}[信号的积]
    信号$f_1(\cdot)$与$f_2(\cdot)$之积,是指同一瞬时两信号值对应相乘构成的\uwave{积信号}
    \begin{Equation}
        f(\cdot)=f_1(\cdot)f_2(\cdot)
    \end{Equation}
\end{BoxDefinition}

关于信号的加法和乘法,广播是一个极好的例子。广播的调音台就是信号相加的一个实例,它将背景音乐$f_1(t)$和主持人的声音$f_1(t)$混合到一起。广播调制常有调幅AM和调频FM两种方法,而所谓调幅,就是指将相对低频的音频信号$f_1(t)$加载到(乘到)一个相对高频的正弦信号$f_2(t)$上,由于是通过正弦的幅值变化传递信息的,故称调幅。这就是信号相乘的实例。

\subsection{反转}
\begin{BoxDefinition}[信号的反转]
    \uwave{反转}(Reversal)是指,对于信号$f(t)$,作如下变换
    \begin{Equation}
        f(-t)
    \end{Equation}
    其意义是将信号$f(t)$以纵坐标轴为轴反转。
\end{BoxDefinition}

\subsection{平移}
\begin{BoxDefinition}[信号的平移]
    \uwave{平移}(Shifting)是指,对于信号$f(t)$和常数$t_0>0$,作如下变换
    \begin{Equation}
        f(t-t_0)\qquad f(t+t_0)
    \end{Equation}
    其意义是将信号$f(t)$沿横坐标轴平移
    \begin{itemize}
        \item $f(t-t_0)$是将原信号$f(t)$沿$t$轴正方向(向右)平移$t_0$时间,即延迟。
        \item $f(t+t_0)$是将原信号$f(t)$沿$t$轴负方向(向左)平移$t_0$时间,即提前。
    \end{itemize}
\end{BoxDefinition}

信号平移的一个实例是雷达系统,雷达接收到的目标反射的回波信号就是其发射信号$f(t)$的延迟信号$f(t-t_0)$,根据测得的延迟时间$t_0$的值,就可以计算出目标与雷达之间的距离。

信号的反转和平移,以及稍后将介绍的尺度变换的图示,如\xref{tab:信号的基本运算}所示。

\begin{TableLong}[信号的基本运算]{cc}
<连续信号&离散信号\\>
\xcell<c>[2ex][0ex]{\includegraphics{build/Chapter01C_01a.fig.pdf}}&
\xcell<c>[2ex][0ex]{\includegraphics{build/Chapter01C_01g.fig.pdf}}\\*
原信号$f(t)$&原信号$f(k)$\\
\xcell<c>[2ex][0ex]{\includegraphics{build/Chapter01C_01b.fig.pdf}}&
\xcell<c>[2ex][0ex]{\includegraphics{build/Chapter01C_01h.fig.pdf}}\\*
反转$f(-t)$&反转$f(-k)$\\
\xcell<c>[2ex][0ex]{\includegraphics{build/Chapter01C_01c.fig.pdf}}&
\xcell<c>[2ex][0ex]{\includegraphics{build/Chapter01C_01i.fig.pdf}}\\*
向右平移$f(t-t_0)$&向右平移$f(k-k_0)$\\
\xcell<c>[2ex][0ex]{\includegraphics{build/Chapter01C_01d.fig.pdf}}&
\xcell<c>[2ex][0ex]{\includegraphics{build/Chapter01C_01j.fig.pdf}}\\*
向左平移$f(t+t_0)$&向左平移$f(k+k_0)$\\
\xcell<c>[2ex][0ex]{\includegraphics{build/Chapter01C_01e.fig.pdf}}&
\xcell<c>[2ex][0ex]{\includegraphics{build/Chapter01C_01k.fig.pdf}}\\
压缩$f(at), a>1|_{a=2.0}$&压缩$f(ak), a>1|_{a=2.0}$\\
\xcell<c>[2ex][0ex]{\includegraphics{build/Chapter01C_01f.fig.pdf}}&
\xcell<c>[2ex][0ex]{\includegraphics{build/Chapter01C_01l.fig.pdf}}\\*
展宽$f(at), a<1|_{a=0.5}$&展宽$f(ak), a<1|_{a=0.5}$\\
\end{TableLong}

\subsection{尺度变换}

\begin{BoxDefinition}[尺度变换]
    \uwave{尺度变换}(Scaling)又称为横坐标展缩,是指,对于信号$f(t)$和常数$a$,作如下变换
    \begin{Equation}
        f(at)
    \end{Equation}
    其意义是将信号$f(t)$横坐标展缩
    \begin{itemize}
        \item 若$|a|>1$,$f(at)$是将$f(t)$压缩至原来的$|a|$倍。
        \item 若$|a|<1$,$f(at)$是将$f(t)$展宽至原来的$|a|^{-1}$倍。
        \item 当$a$为负数时,还要在上述展缩的基础上叠加一个反转。
    \end{itemize}
\end{BoxDefinition}

尺度变换的一个示例是倍速播放,如$f(2t)$可以视为$2$倍速,而$f(0.5t)$可以视为$0.5$倍速。

尺度变换通常都是对连续信号而言的,离散信号通常不作展缩运算。这是因为展缩运算在离散信号下并非良定义的,由于$f(ak)$仅在$ak$为整数时才有定义,这会导致诸多问题,如\xref{tab:信号的基本运算}所示,若$a>1$且为整数,比如$2$,则$f(ak)$变为了一个只有在$k$为偶数时才有定义的序列,这是很怪异的。相反,若$a<1$,比如$0.5$,此时$f(ak)$的间隔是$1$了,但是丢失了原有信号一半的信息,确切的说,奇数点的信息。因此,我们通常不会对离散信号作尺度变换。

最后我们指出,若结合反转、平移、尺度变换,其一般形式是
\begin{Equation}
    f(at+b)
\end{Equation}
注意变换作用顺序!在$f(t)$上,先作$-b$的平移得到$f(t+b)$,再作$a$的展缩得到$f(at+b)$。