\section{傅里叶变换的示例}
在本节,我们将计算一些常用的函数的傅里叶变换,以此对傅里叶变换有更深刻的理解。

\subsection{指数函数的傅里叶变换}
这一小节,我们讨论与指数函数$\e^{-at}$有关的一些傅里叶变换。

\begin{Tablex}[单边指数函数的傅里叶变换]{|Y|Y|}
    <时域&频域\\>
        \mrx<Y>{2}[-6ex]{\includegraphics[scale=0.75]{build/Chapter05C_01a.fig.pdf}}
        &\xcell<Y>[2ex][-4ex]{\includegraphics[scale=0.75]{build/Chapter05C_01b.fig.pdf}}\\
        &\xcell<Y>{\footnotesize $\mal{\Abs[X(\j\omega)]=\frac{1}{\sqrt{a^2+\omega^2}}}$}\\
        &\xcell<Y>[1ex][-4ex]{\includegraphics[scale=0.75]{build/Chapter05C_01c.fig.pdf}}\\
        &\xcell<Y>{\footnotesize $\mal{\Arg[X(\j\omega)]=-\arctan(\frac{\omega}{a})}$}\\
        \xcell<Y>{$\mal{x(t)=\e^{-at}u(t)}$}&
        \xcell<Y>{$\mal{X(\j\omega)=\frac{1}{a+\j\omega}}$}\\
\end{Tablex}

\begin{BoxExample}[单边指数函数的傅里叶变换]
    对于单边指数函数
    \begin{Equation}
        x(t)=\e^{-at}u(t)
    \end{Equation}
    傅里叶变换为
    \begin{Equation}
        X(\j\omega)=\frac{1}{a+\j\omega}
    \end{Equation}
    傅里叶变换的模和辐角分别为
    \begin{Equation}
        \Abs[X(\j\omega)]=\frac{1}{\sqrt{a^2+\omega^2}}\qquad
        \Arg[X(\j\omega)]=\arctan(\frac{\omega}{a})
    \end{Equation}
\end{BoxExample}

\begin{Proof}
    根据\fancyref{fml:傅里叶变换}
    \begin{Equation}
        X(\j\omega)=\Int[-\infty][\infty]x(t)\e^{-\j\omega t}\dd{t}=\Int[0][\infty]\e^{-at}\e^{-\j\omega t}\dd{t}=\Int[0][\infty]\e^{-(a+\j\omega)t}\dd{t}
    \end{Equation}
    计算积分
    \begin{Equation}
        X(\j\omega)=\eval{-\frac{1}{a+\j\omega}\e^{-(a+\j\omega)t}}_0^{\infty}
    \end{Equation}
    完成求值
    \begin{Equation}*
        X(\j\omega)=\frac{1}{a+\j\omega}\qedhere
    \end{Equation}
\end{Proof}

在\xref{tab:单边指数函数的傅里叶变换}中展现了单边指数函数和其傅里叶变换的图像。应注意的是,频域$X(\j\omega)$是一个复函数,因此一般来说需要两张图刻画。\xref{tab:单边指数函数的傅里叶变换}绘制了$X(\j\omega)$的模和辐角,亦可以选择实部和虚部。

\begin{Tablex}[双边指数函数的傅里叶变换]{|Y|Y|}
    <时域&频域\\>
    \xcell<Y>[2ex][-3ex]{\includegraphics[scale=0.75]{build/Chapter05C_01d.fig.pdf}}&
    \xcell<Y>[2ex][-3ex]{\includegraphics[scale=0.75]{build/Chapter05C_01e.fig.pdf}}\\
    \xcell<Y>{$\mal{x(t)=\e^{-a|t|}}$}&
    \xcell<Y>{$\mal{X(\j\omega)=\frac{2a}{a^2+\omega^2}}$}\\
    \hlinelig
    \xcell<Y>[2ex][-3ex]{\includegraphics[scale=0.75]{build/Chapter05C_01f.fig.pdf}}&
    \xcell<Y>[2ex][-3ex]{\includegraphics[scale=0.75]{build/Chapter05C_01g.fig.pdf}}\\
    \xcell<Y>{$\mal{x(t)=\e^{-a|t|}\sgn(t)}$}&
    \xcell<Y>{$\mal{X(\j\omega)=\frac{-2\j\omega}{a^2+\omega^2}}$}\\
\end{Tablex}

\begin{BoxExample}[双边偶指数函数的傅里叶变换]
    对于双边偶指数函数
    \begin{Equation}
        x(t)=\e^{-a|t|}
    \end{Equation}
    傅里叶变换为
    \begin{Equation}
        X(\j\omega)=\frac{2a}{a^2+\omega^2}
    \end{Equation}
\end{BoxExample}

\begin{Proof}
    根据\fancyref{fml:傅里叶变换}
    \begin{Equation}
        X(\j\omega)=\Int[-\infty][\infty]x(t)\e^{-\j\omega t}\dd{t}=\Int[0][\infty]\e^{-at}\e^{-\j\omega t}\dd{t}+\Int[-\infty][0]\e^{+at}\e^{-\j\omega t}\dd{t}
    \end{Equation}
    即
    \begin{Equation}
        X(\j\omega)=\Int[0][\infty]\e^{-(a+\j\omega)t}\dd{t}+\Int[-\infty][0]\e^{(a-\j\omega)t}\dd{t}
    \end{Equation}
    计算积分
    \begin{Equation}
        X(\j\omega)=
        -
        \eval{\frac{1}{a+\j\omega}\e^{-(a+\j\omega)t}}_{0}^{\infty}+
        \eval{\frac{1}{a-\j\omega}\e^{(a-\j\omega)t}}_{-\infty}^{0}
    \end{Equation}
    完成求值
    \begin{Equation}*
        X(\j\omega)=\frac{1}{a+\j\omega}+\frac{1}{a-\j\omega}+=\frac{2a}{a^2+\omega^2}\qedhere
    \end{Equation}
\end{Proof}

\begin{BoxExample}[双边奇指数函数的傅里叶变换]
    对于双边奇指数函数
    \begin{Equation}
        x(t)=\e^{-a|t|}\sgn(t)
    \end{Equation}
    傅里叶变换为
    \begin{Equation}
        X(\j\omega)=\frac{-2\j\omega}{a^2+\omega^2}
    \end{Equation}
\end{BoxExample}

\begin{Proof}
    根据\fancyref{fml:傅里叶变换}
    \begin{Equation}
        X(\j\omega)=\Int[-\infty][\infty]x(t)\e^{-\j\omega t}\dd{t}=\Int[0][\infty]\e^{-at}\e^{-\j\omega t}\dd{t}-\Int[-\infty][0]\e^{+at}\e^{-\j\omega t}\dd{t}
    \end{Equation}
    即
    \begin{Equation}
        X(\j\omega)=\Int[0][\infty]\e^{-(a+\j\omega)t}\dd{t}-\Int[-\infty][0]\e^{(a-\j\omega)t}\dd{t}
    \end{Equation}
    计算积分
    \begin{Equation}
        X(\j\omega)=
        -\eval{\frac{1}{a+\j\omega}\e^{-(a+\j\omega)t}}_{0}^{\infty}
        -\eval{\frac{1}{a-\j\omega}\e^{(a-\j\omega)t}}_{-\infty}^{0}
    \end{Equation}
    完成求值
    \begin{Equation}*
        X(\j\omega)=\frac{1}{a+\j\omega}-\frac{1}{a-\j\omega}=\frac{-2\j\omega}{a^2+\omega^2}\qedhere
    \end{Equation}
\end{Proof}

在\xref{tab:双边指数函数的傅里叶变换}中展现了双边指数函数和其傅里叶变换的图像。略微有所不同的是,由于双边指数函数中,偶函数的频域仅包含实部,奇函数的频域仅包含虚部,我们可以仅用一张图表示频域。

\subsection{常值函数的傅里叶变换}
\begin{BoxExample}[常值函数的傅里叶变换]
    对于常值函数
    \begin{Equation}
        x(t)=1
    \end{Equation}
    傅里叶变换为
    \begin{Equation}
        X(\j\omega)=2\pi\dirac(\omega)
    \end{Equation}
    简而言之,常值函数的傅里叶变换给出狄拉克函数。
\end{BoxExample}

\begin{Proof}
    试着应用\fancyref{fml:傅里叶变换},我们会遗憾的发现该积分并不收敛
    \begin{Equation}
        X(\j\omega)=\Int[-\infty][\infty]x(t)\e^{-\j\omega t}\dd{t}=\Int[-\infty][\infty]\e^{-\j\omega t}\dd{t}
    \end{Equation}
    这就是因为$x(t)=1$并不是一个绝对可积的函数,故无法通过通常的办法完成傅里叶变换。

    这里可以将$x(t)=1$视为双边偶指数函数的极限,如\xref{tab:常值函数的傅里叶变换}所示
    \begin{Equation}
        x(t)=\Lim[a\to 0]\e^{-a|t|}=1
    \end{Equation}
    故可以预期$x(t)$的傅里叶变换$X(\j\omega)$也是双边偶指数函数的极限,依据\xref{exp:双边偶指数函数的傅里叶变换}和\xref{tab:冲激函数的其他辅助函数}
    \begin{Equation}*
        X(\j\omega)=\Lim[a\to 0]\frac{2a}{a^2+\omega^2}=2\pi\dirac(\omega)\qedhere
    \end{Equation}
\end{Proof}

\begin{Tablex}[常值函数的傅里叶变换]{|Y|Y|}
    <时域&频域\\>
    \xcell<Y>[2ex][-3ex]{\includegraphics[scale=0.75]{build/Chapter05C_01h.fig.pdf}}&
    \xcell<Y>[2ex][-3ex]{\includegraphics[scale=0.75]{build/Chapter05C_01j.fig.pdf}}\\
    \xcell<Y>{$\mal{x(t)=\Lim[a\to 0]\e^{-a|t|}}$}&
    \xcell<Y>{$\mal{X(\j\omega)=\Lim[a\to 0]\frac{2a}{a^2+\omega^2}}=2\pi\dirac(\omega)$}\\
\end{Tablex}

\subsection{符号函数的傅里叶变换}
\begin{BoxExample}[符号函数的傅里叶变换]
    对于符号函数
    \begin{Equation}
        x(t)=\sgn(t)
    \end{Equation}
    傅里叶变换为
    \begin{Equation}
        X(\j\omega)=2/j\omega
    \end{Equation}
    简而言之,符号函数的傅里叶变换给出反比例函数。
\end{BoxExample}

\begin{Proof}
    试着应用\fancyref{fml:傅里叶变换},我们会同样发现符号函数也并不收敛
    \begin{Equation}
        X(\j\omega)=\Int[-\infty][\infty]x(t)\e^{-\j\omega t}=\Int[0][\infty]\e^{-\j\omega t}\dd{t}-\Int[-\infty][0]\e^{-\j\omega t}\dd{t}
    \end{Equation}
    这里可以将$x(t)=\sgn(x)$视为双边奇指数函数的极限,如\xref{tab:符号函数的傅里叶变换}所示
    \begin{Equation}
        x(t)=\Lim[a\to 0]\e^{-\alpha|t|}\sgn(t)=1
    \end{Equation}
    故可以预期$x(t)$的傅里叶变换$X(\j\omega)$也是双边奇指数函数的极限,依据\xref{exp:双边奇指数函数的傅里叶变换},计算极限
    \begin{Equation}*
        X(\j\omega)=\Lim[a\to 0]\frac{-2\j\omega}{a^2+\omega^2}=\frac{-2\j}{\omega}=\frac{2}{\j\omega}\qedhere
    \end{Equation}
\end{Proof}

\begin{Tablex}[符号函数的傅里叶变换]{|Y|Y|}
    <时域&频域\\>
    \xcell<Y>[2ex][-3ex]{\includegraphics[scale=0.75]{build/Chapter05C_01i.fig.pdf}}&
    \xcell<Y>[2ex][-3ex]{\includegraphics[scale=0.75]{build/Chapter05C_01k.fig.pdf}}\\
    \xcell<Y>{$\mal{x(t)=\Lim[a\to 0]\e^{-a|t|}\sgn(t)}$}&
    \xcell<Y>{$\mal{X(\j\omega)=\Lim[a\to 0]\frac{-2\j\omega}{a^2+\omega^2}=\frac{2}{\j\omega}}$}\\
\end{Tablex}

很有意思的是,介于阶跃函数$u(t)$可以用常值函数$1$和符号函数$\sgn(t)$的线性组合表示
\begin{Equation}
    u(t)=\frac{1}{2}\qty[1+1\sgn(t)]
\end{Equation}
由\xref{exp:常值函数的傅里叶变换}和\xref{exp:符号函数的傅里叶变换}可得
\begin{Equation}
    u(t)\Farr\frac{1}{\j\omega}+\pi\dirac(\omega)
\end{Equation}
由此,就得到了阶跃函数$u(t)$的傅里叶变换(这里实际默认了傅里叶变换的线性性质)。

\subsection{冲激函数的傅里叶变换}
\begin{BoxExample}[冲激函数的傅里叶变换]
    对于冲激函数
    \begin{Equation}
        x(t)=\dirac(t)
    \end{Equation}
    傅里叶变换为
    \begin{Equation}
        X(\j\omega)=1
    \end{Equation}
    对于冲激函数的导数
    \begin{Equation}
        x(t)=\dirac^{(n)}(t)
    \end{Equation}
    傅里叶变换为
    \begin{Equation}
        X(\j\omega)=(\j\omega)^n
    \end{Equation}
    简而言之,冲激函数及其导数的傅里叶变换给出幂函数,特别的,冲激函数给出常值。
\end{BoxExample}\goodbreak

\begin{Proof}
    根据\fancyref{fml:傅里叶变换}和\fancyref{fml:冲激函数的导数}
    \begin{Equation}*
        X(\j\omega)=\Int[-\infty][\infty]\dirac^{(n)}(t)\e^{-\j\omega t}\dd{t}=(-1)^n\eval{\dv[n]{t}\e^{-\j\omega t}}_{t=0}=(-1)^n(-\j\omega)^n=(\j\omega)^n\qedhere
    \end{Equation}
\end{Proof}
