\chapter{周期信号的傅里叶级数}
在\xref{chap:连续系统的时域分析}和\xref{chap:离散系统的时域分析}中,我们引入了卷积积分和卷积和的概念来分析LTI系统,这事实上是将信号表示为一系列冲激函数$\dirac(t)$或单位序列$\dirac(k)$的线性组合。而在本章和接下来两章,我们将讨论信号的另一种表示,仍然是表示为一系列“基本信号”的线性组合,但这时,我们所用到的“基本信号”由冲激信号变为了复指数信号,这样所得到的表示,就是所谓傅里叶级数和傅里叶变换,我们将会看到,这种表示方法也能够用来构成范围相当广泛而有用的一类信号。

在这一章中,我们将集中讨论连续时间和离散时间的周期信号的傅里叶级数表示。在接下来的两章中,我们将会进一步把这种思想,由周期信号推广到非周期信号中,从而由傅里叶级数得到傅里叶变换,两章将分别讨论连续傅里叶变换和离散傅里叶变换。总之,傅里叶级数和傅里叶方法合在一起,就为分析、设计、理解信号与LTI系统提供了一种最有力和最重要的分析方法,即\uwave{傅里叶分析}(Fourier Analysis)。下面,我们先简要回顾一下傅里叶分析的历史。

\section{傅里叶分析的历史回顾}

傅里叶分析方法的建立有过一段漫长的历史,涉及到很多人的工作和不同物理现象的研究。利用“三角函数和”,即,成谐波关系的正弦和余弦函数或复指数函数的和的来描述周期性过程,至少可以追溯到古巴比伦时代,当时他们利用这一想法来预测天体运动。这一问题的近代历史始于1748年欧拉对振动弦的研究工作。\xref{fig:振动弦的标准振荡模}绘出了弦振动的前几个标准振荡模式。若用函数$f(x,t)$来表示弦在时间$t$且沿着弦的某一横向距离$x$处的垂直偏离,则对于任意固定时刻$t$来说,所有这些振荡模式均为关于$x$的正弦函数。欧拉从中得出的结论包含两点
\begin{enumerate}
    \item 若在某一时刻,振动弦的形状是这些标准振荡模的现线性组合,那么,在其后任意时刻,振动弦的形状也都是这些振荡模的线性组合,只不过振荡模间的加权系数发生了变化。
    \item 若前面时刻的加权系数(初值)已知,则后面时刻的加权系数可以直接由此求得。
\end{enumerate}
具体而言,如果一个LTI系统的输入可以表示为正弦信号的线性组合,那么该LTI系统的输出也一定能表示为这种形式,且输出信号中的加权系数直接与输入信号中对应的系数有关。

\begin{Figure}[振动弦的标准振荡模]
    \includegraphics[scale=1.05]{build/Chapter04A_01.fig.pdf}\hspace{1.5cm}
\end{Figure}

显然,除非很多有用的信号,比如方波和锯齿波,都能用三角函数或复指数函数的线性组合来表示,否则上面所讨论的性质就不会特别有用,在18世纪中期,这一点曾是激烈争论的主题
\begin{itemize}
    \item 在1753年,伯努利(D. Bernoulli)曾经声称,一根弦的实际运动都可以用标准振荡模的线性组合表示,但他并没有继续从数学上深入研究,并且当时该想法也并未被广泛接受。
    \item 在1759年,拉格朗日(J. L. Lagrange)也曾强烈批量使用正弦级数来研究振动弦运动的主张,他反对的论据是基于他自己的信念,既不可能用三角级数来表示一个具有间断点的函数(例如方波和锯齿波)。因为振动弦的波形是由拨动弦而引起的,即将弦绷紧再松开,所以拉格朗日认为三角级数的应用范围非常有限(没太明白这前后有什么关联)。
\end{itemize}
正是在这种多少有些敌意和怀疑的处境下,傅里叶(Joseph Fourier)约在半个世纪后提出了他自己的想法。热的传播和扩散现象是导致傅里叶研究成果的实际物理背景。在当时数学物理学领域中大多数前人的研究已经涉及理论力学和天体力学的背景下,这一问题本身就是十分有意义的一步。至1807年,傅里叶已经完成了一项研究,他发现在表示一个物体的温度分布时,成谐波关系的正弦函数级数是非常有用的。另外,他还断言,任何周期信号都可以用该级数来表示。虽然在这一问题其论述很有意义,但隐藏在这一问题后面的许多基本概念在那时其实已经被其他科学家们所发现,同时,傅里叶的数学证明也不是很完善,至1829年,狄利克雷(P. L. Dirichlet)才给出了周期信号可以用傅里叶级数表述的若干精确条件,换言之,傅里叶实际上并没有对傅里叶级数的数学理论做出什么贡献。然而,傅里叶确实洞察出了这个级数表示法的潜在威力,并且在很大程度上正是由于他的工作和断言,才激励和推动了傅里叶级数物体的深入研究。除此之外,傅里叶在这一问题上的研究成果比他的任何先驱者都大大前进了一步,这指的是他还得出了关于非周期信号的表示,这不再是成谐波关系的正弦信号的加权和,而是由不全成谐波的正弦信号的加权积分,即傅里叶变换(将在下一章中讨论)。

前面提及的许多应用,以及傅里叶和她的同伴们在数学物理学方面的最初研究,都集中在连续时间内的现象。而与此同时,离散时间信号与系统的傅里叶分析方法,却有着它们自己不同的历史根基,并且也有众多的应用领域。尤其是,离散时间概念和方法是数值分析这门学科的基础。在17世纪的牛顿时代,用于处理离散点集以产生数值近似的有关内插、积分、微分方面的公式就  被研究过。在18世纪和19世纪,已知一组天体观察数据序列,预测某一天体运动的稳态问题,也曾吸引着包含高斯(Gauss)在内的许多著名科学家和数学家从事时间序列调和序列的研究,从而为大量的初始工作能在离散时间信号与系统下完成提供了第二个舞台。
\section{LTI系统对复指数信号的响应}
正如我们在本章首指出的那样,在LTI系统中,将信号表示成基本信号的线性组合是很有利的,但是,这些基本信号应当如何选取呢?我们提出以下两条基本信号应具有的性质
\begin{enumerate}
    \item 这些基本信号在LTI系统上的响应应当十分简单。
    \item 这些基本信号的线性组合,能构成相当广泛的一类信号。
\end{enumerate}
傅里叶分析的很多重要价值就来自于这两点。我们知道,傅里叶分析选取的基本信号是复指数函数,取决于系统是连续时间还是离散时间,其分别可以记作$\e^{st}, z^n$,其中$s$和$z$均为复数,而复指数函数$\e^{st}, z^n$实际上就满足以上两条好的性质,具体来说,在LTI系统中,复指数函数的响应仍然是复指数函数,同时,复指数函数的线性组合可以表示任意满足狄利克雷条件的周期函数,后者是一个相当宽松的条件。在本节和下两节我们将分别具体讨论这两条性质。

那么,为什么在LTI系统中,复指数信号的响应仍然是复指数信号呢?
\begin{BoxProperty}[LTI系统对复指数信号的响应]
    LTI系统对复指数信号的响应,同样是一个复指数信号,但幅度不同。

    对于连续时间系统
    \begin{Equation}
        \e^{st}\to H(s)\e^{st}
    \end{Equation}
    对于离散时间系统
    \begin{Equation}
        z^n\to H(z)z^n
    \end{Equation}
    其中$H(s)$或$H(z)$是复振幅因子,通常是关于$s$或$z$的函数。
\end{BoxProperty}

\begin{Proof}
    \paragraph{连续时间系统的证明}
    对于任意输入$x(t)$,其响应可由卷积分积分确定,因此若$x(t)=\e^{st}$
    \begin{Equation}
        y(t)=\Int[-\infty][\infty]h(\tau)x(t-\tau)\dd{\tau}=\Int[-\infty][\infty]h(\tau)\e^{s(t-\tau)}\dd{\tau}
    \end{Equation}
    将$\e^{st}$从积分内移出来
    \begin{Equation}
        y(t)=\e^{st}\Int[-\infty][\infty]h(\tau)\e^{-s\tau}\dd{\tau}
    \end{Equation}
    若上式右端的积分收敛,记其为$H(s)$
    \begin{Equation}*
        y(t)=\e^{st}H(s)
    \end{Equation}
    \paragraph{离散时间系统的证明}
    对于任意输入$x(n)$,其响应可由卷积核确定,因此若$x(n)=z^n$
    \begin{Equation}
        y(n)=\Sum[k=-\infty][\infty]h(k)x(n-k)=\Sum[k=-\infty][\infty]h(k)z^{n-k}
    \end{Equation}
    将$z^n$从求和内移出来
    \begin{Equation}
        y(n)=z^n\Sum[k=-\infty][\infty]h(k)z^{-k}
    \end{Equation}
    若上式右端的求和收俩,记其为$H(z)$
    \begin{Equation}*
        y(n)=z^nH(z)\qedhere
    \end{Equation}
\end{Proof}

我们已经相当熟悉特征值与特征函数的概念,若函数经过某个算子的作用后仍然为该函数的常数倍,那么,该常数称为这个算子的\uwave{特征值}(Eigenvalue),该函数则称为这个算子的\uwave{特征函数}(Eigenfunction)。在这里,LTI系统即算子,而复指数信号$\e^{st},z^n$经过其作用后是幅度变化的复指信号$H(s)\e^{st}, H(z)\e^{st}$,因此,我们就可以说,复指数信号$\e^{st}, z^n$和其对应的复振幅因子$H(s), H(z)$分别是LTI系统的特征函数和特征值。当然,尽管$H(s),H(z)$看起来是函数,但是对于每一个确定的特征函数而言$s$和$z$是固定的,因此$H(s),H(z)$确实为常数。

我们如果将上述性质与LTI系统的叠加性质联合起来,就意味着以下性质
\begin{BoxProperty}[LTI系统的叠加性与复指数响应]
    若连续LTI系统的输入为复指数信号的线性组合
    \begin{Equation}
        x(t)=\Sum[k]a_k\e^{s_kt}
    \end{Equation}
    那么输出一定是
    \begin{Equation}
        y(t)=\Sum[k]a_kH(s_k)\e^{s_kt}
    \end{Equation}
    若离散LTI系统的输入为复指数信号的线性组合
    \begin{Equation}
        x(n)=\Sum[k]a_kz_k^n
    \end{Equation}
    那么输出一定是
    \begin{Equation}
        y(n)=\Sum[k]a_kH(z_k)z_k^n
    \end{Equation}
\end{BoxProperty}

换言之,如果一个LTI系统的输入能够表示复指数组合,那么系统的输出也能表示为相同复指数的线性组合。并且,在输出表达式中的每一个系数,都可以用输入中的相应系数$a_k$分别与相应特征函数$\e^{s_kt}$或$z_k^n$有关的特征值$H(s_k)$或$H(z_k)$相乘求得。欧拉在弦振动问题的研究中发现的的正是这一事实,高斯及其他学者在时间序列分析中所用的也是这一点。这也就促使傅里叶及其后的其他人考虑这样一个问题:\empx{究竟多大范围内的信号可以用复指数的线性组来表示?}在本章接下来的几节中,将先对周期信号研究这个问题,次序是先连续后离散。在接下来的两章中再将这些表达式推广到非周期信号。除此之外,值得强调的是,理论上这里出现的$s$和$z$都可以是任意复数,但傅里叶分析仅限于这些变量的特殊形式,具体而言
\begin{itemize}
    \item 连续时间下$\e^{st}$的$s$仅限于纯虚部值$s=\j\omega$,因此仅考虑$\e^{\j\omega t}$形式的复指数。
    \item 离散时间下$z^n$的$z$仅限于单位振幅值$z=\e^{\j\omega}$,因此仅考虑$\e^{\j\omega n}$形式的复指数。
\end{itemize}
假若我们取消这一限制,就将从傅里叶变换过渡到拉普拉斯变换,不过这是更后面的问题了。
\section{连续时间傅里叶级数的表示}

\subsection{连续时间傅里叶级数}\setpeq{连续时间傅里叶级数}
正如\xref{chap:信号与系统}中\xref{def:连续信号的周期},若连续信号是周期的,那对于所有的$t$,存在某个正值的$T$,有
\begin{Equation}
    x(t)=x(t+T)
\end{Equation}
在这里,我们将上述$T$的最小非零值称为\uwave{基波周期},而$\omega_0=2\pi/T$则相应称为\uwave{基波频率}。

在本章中我们所关注的复指数信号也是一个周期信号
\begin{Equation}
    x(t)=\e^{\j\omega_0t}
\end{Equation}

而与之呈\uwave{谐波关系}(Harmonically Related)的复指数信号集就是
\begin{Equation}
    \phi_k(t)=\e^{\j k\omega_0t}\qquad k=0,\pm 1,\pm 2,\cdots 
\end{Equation}
这些信号的每一个都有一个基波频率,它们都是$\omega_0$的整数倍$1\omega_0, 2\omega_0, 3\omega_0, \cdots$,因此,它们的周期相应即为$(T/1), (T/2), (T/3), \cdots$,换言之,这些信号对于$T$来说仍然都是周期的(尽管对于$k\geq 2$的信号,$T$并非基波周期),故一个由成谐波关系的复指数线性组合的信号
\begin{Equation}&[1]
    x(t)=\Sum[k=-\infty][\infty]a_k\e^{\j k\omega_0t}
\end{Equation}
仍然是关于$T$的周期信号。其中,$k=0$的项其实就是一个常数,$k=\pm 1$这两项的基波频率都等于$\omega_0$,两者合在一起称为\uwave{基波分量}(Fundamental Component)或\uwave{一次谐波分量}(First Harmonic Component),$k=\pm 2$的频率则是基波分量的两倍,两者合在一起称为\uwave{二次谐波分量}(Second Harmonic Component)。而一般来说,$k=\pm N$就相应称为$N$次谐波分量。\goodbreak

而一个周期信号表示为\xrefpeq{1}的形式,就称为其傅里叶级数表示。
\begin{BoxDefinition}[连续傅里叶级数]
    若一个频率为$\omega_0$的连续周期信号$x(t)$被表示为以下形式
    \begin{Equation}
        x(t)=\Sum[k=-\infty][\infty]a_k\e^{\j k\omega_0t}
    \end{Equation}
    就称之为为$x(t)$的\uwave{傅里叶级数}(Fourier Series)表示。
\end{BoxDefinition}

我们在微积分等课程中更熟悉的那种傅里叶级数,通常是由$\sin(k\omega_0 t)$和$\cos(k\omega_0 t)$表示的,让我们来看,那种形式是如何从\xref{def:傅里叶级数}产生的。事实上,若$x(t)$是实信号,由于$x(t)=x^{*}(t)$
\begin{Equation}
    x(t)=x^{*}(t)=\Sum[k=-\infty][\infty]a_k^{*}\e^{-\j k\omega_0 t}
\end{Equation}
以$-k$代替$k$并不妨碍结果
\begin{Equation}
    x(t)=\Sum[k=-\infty][\infty]a_{-k}^{*}\e^{\j k\omega_0 t}
\end{Equation}
而将其与\xref{def:连续傅里叶级数}的原始定义比较
\begin{Equation}
    x(t)=\Sum[k=-\infty][\infty]a_k\e^{\j k\omega_0t}
\end{Equation}
我们就可以得到
\begin{Equation}
    a_k^{*}=a_{-k}
\end{Equation}
这是一个重要的性质,即,若$x(t)$是实信号,那其傅里叶系数就满足$a_k^{*}=a_{-k}$,或者说
\begin{itemize}
    \item $a_k$的实部是一个关于$k$的偶函数(序列)。
    \item $a_k$的虚部是要给关于$k$的奇函数(序列)。
\end{itemize}
现在,为了导出傅里叶级数的另外一种形式,我们将\xref{def:连续傅里叶级数}的求和重新写成
\begin{Equation}
    x(t)=a_0+\Sum[k=1][\infty][a_k\e^{\j k\omega_0t}+a_{-k}\e^{-\j k\omega_0t}]
\end{Equation}
由于$x(t)$被假定是实信号,有$a_{k}^{*}=a_{-k}$
\begin{Equation}
    x(t)=a_0+\Sum[k=1][\infty][a_k\e^{\j k\omega_0t}+a_k^{*}\e^{-\j k\omega_0t}]
\end{Equation}
由于括号内的两项互为共轭,重新写作
\begin{Equation}
    x(t)=a_0+\Sum[k=1][\infty]2\Re{a_k\e^{\j k\omega_0t}}
\end{Equation}
若$a_k$以极坐标形式呈现$a_k=A_k\e^{j\theta_k}$,这就有
\begin{Equation}
    x(t)=a_0+2\Sum[k=1][\infty]A_k\cos(k\omega_0t+\theta_k)
\end{Equation}
若$a_k$以直接坐标形式呈现$a_k=B_k+\j C_k$,这就有
\begin{Equation}
    x(t)=a_0+2\Sum[k=1][\infty][B_k\cos(k\omega_0t)-C_k\sin(k\omega_0t)]
\end{Equation}
值得强调的是,这两种三角形式的傅里叶级数,与\xref{def:连续傅里叶级数}中复指数形式的傅里叶级数并不完全等价。三角形式仅适用于$x(t)$为实信号的情况,若$x(t)$为复信号,则只能使用复指数形式。实际上,傅里叶级数的三角形式是最普遍被采用的(也是微积分中最初接触的形式),傅里叶最初工作中使用的傅里叶级数就是三角形式的,但是,复指数形式的傅里叶级数对于我们将讨论的问题来说,却是特别方便,今后我们都将毫无例外的采用复指数形式的傅里叶级数。

\subsection{连续时间傅里叶级数的表示}
现在的问题是,若一个周期信号可以表示为傅里叶级数,那傅里叶级数的系数如何确定呢?
\begin{BoxFormula}[连续傅里叶级数的系数]
    若$x(t)$能表示为傅里叶级数的形式
    \begin{Equation}&[a]
        x(t)=\Sum[k=-\infty][\infty]a_k\e^{\j k\omega_0t}
    \end{Equation}
    那么其中的\uwave{傅里叶系数}(Fourier Series Conefficient)$a_k$就相应为(其中$T=2\pi/\omega_0$)
    \begin{Equation}&[b]
        a_k=\frac{1}{T}\Int[T]x(t)\e^{-\j k\omega_0t}\dd{t}
    \end{Equation}
    \xrefpeq{a}称为\uwave{综合}(Synthesis)公式,\xrefpeq{b}称为\uwave{分析}(Analysis)公式。
\end{BoxFormula}

\begin{Proof}
    假设一个给定的周期信号$x(t)$可以表示为傅里叶级数,根据\fancyref{def:连续傅里叶级数}
    \begin{Equation}&[1]
        x(t)=\Sum[k=-\infty][\infty]a_k\e^{\j k\omega_0 t}
    \end{Equation}
    两边各乘$\e^{-\j n\omega_0t}$,其中$n$是任意整数
    \begin{Equation}&[2]
        x(t)\e^{-\j n\omega_0t}=\Sum[k=-\infty][\infty]a_k\e^{\j k\omega_0t}\e^{-\j n\omega_0t}
    \end{Equation}
    两边从$0$到$T=2\pi/\omega_0$对$t$积分,有
    \begin{Equation}&[3]
        \Int[0][T]x(t)\e^{-\j n\omega_0t}\dd{t}=\Int[0][T]\Sum[k=-\infty][\infty]a_k\e^{\j k\omega_0t}\e^{-\j n\omega_0t}\dd{t}
    \end{Equation}
    交换积分和求和的次序
    \begin{Equation}&[4]
        \Int[0][T]x(t)\e^{-\j n\omega_0t}\dd{t}=\Sum[k=-\infty][\infty]a_k\Int[0][T]\e^{\j(k-n)\omega_0t}\dd{t}
    \end{Equation}
    \xrefpeq{4}右端的积分是很容易的,为此,利用欧拉关系可得
    \begin{Equation}&[5]
        \Int[0][T]\e^{\j(k-n)\omega_0t}\dd{t}=\Int[0][T]\cos[(k-n)\omega_0t]+\j\Int[0][T]\sin[(k-n)\omega_0t]\dd{t}
    \end{Equation}
    对于$k\neq n$,由于$\cos[(k-n)\omega_0t]$和$\sin[(k-n)\omega_0t]$都是关于$T$的周期函数,积分值为零。特别的,对于$k=n$,左端$\cos$的为常数$1$,积分后为$T$,右端$\sin$的积分为$0$,因此
    \begin{Equation}&[6]
        \Int[0][T]\e^{\j(k-n)\omega_0t}\dd{t}=
        \begin{cases}
            T,&k=n\\
            0,&k\neq n
        \end{cases}
    \end{Equation}
    将\xrefpeq{6}代回\xrefpeq{4}
    \begin{Equation}
        \Int[0][T]x(t)\e^{-\j n\omega_0t}\dd{t}=Ta_n
    \end{Equation}
    即
    \begin{Equation}
        a_n=\frac{1}{T}\Int[0][T]x(t)\e^{-\j n\omega_0t}\dd{t}
    \end{Equation}
    将$n$改记为$k$,且$[0,T]$上积分也可以换成在任意周期内积分,故
    \begin{Equation}*
        a_k=\frac{1}{T}\Int[T]x(t)\e^{-\j k\omega_0t}\dd{t}\qedhere
    \end{Equation}
\end{Proof}

现在让我们来实践一下上述结论,如\xref{tab:周期性方波的傅里叶系数}所示,考虑周期性方波,其在一个周期内的定义是
\begin{Equation}
    x(t)=
    \begin{cases}
        1,&|t|<T_1\\
        0,&T_1<|t|<T/2\\
    \end{cases}
\end{Equation}
换言之,作为偶函数定义,方波周期为$T$,方波高电平的一半为$T_1$,即占空比是$2T_1/T$。

首先,对于$k=0$,这实质就是在计算占空比
\begin{Equation}
    a_0=\frac{1}{T}\Int[-T/2][T/2]x(t)\dd{t}=\frac{1}{T}\Int[-T_1][T_1]\dd{t}=\frac{2T_1}{T}
\end{Equation}
随后,对于$k\neq 0$,计算得到
\begin{Equation}
    a_k=\frac{1}{T}\Int[-T/2][T/2]x(t)\e^{-\j k\omega_0t}\dd{t}=\frac{1}{T}\Int[-T_1][T_1]\e^{-\j k\omega_0t\dd{t}}=\eval{-\frac{1}{\j k\omega_0T}\e^{-\j k\omega_0t}}_{-T_1}^{T_1}
\end{Equation}
或重新写作
\begin{Equation}
    a_k=\frac{2}{ k\omega_0T}\qty[\frac{\e^{\j k\omega_0T_1}-\e^{-\j k\omega_0T_1}}{2\j}]
\end{Equation}
即,并考虑到$\omega_0=2\pi/T$
\begin{Equation}
    a_k=\frac{2\sin(k\omega_0T_1)}{k\omega_0T}=\frac{\sin[k\pi(2T_1/T)]}{k\pi}
\end{Equation}
综上,对于这里的方波,其傅里叶系数满足
\begin{Equation}
    a_k=
    \begin{cases}
        2T_1/T,&k=0\\
        \sin[k\pi(2T_1/T)]/k\pi,&k=0\\
    \end{cases}
\end{Equation}
特别的,如果$T=4T_1$,即占空比$2T_1/T=0.5$时,此时的方波高电平和低电平各占一半。

\begin{Table}[周期性方波的傅里叶系数]{cc}
<时域$x(t)$&频域$a_k$\\>
\xcell<c>[2ex][0ex]{\includegraphics{build/Chapter04C_01a.fig.pdf}}&
\xcell<c>[2ex][0ex]{\includegraphics{build/Chapter04C_01d.fig.pdf}}\\
$T=4T_1$&$T=4T_1$\\
\xcell<c>[2ex][0ex]{\includegraphics{build/Chapter04C_01b.fig.pdf}}&
\xcell<c>[2ex][0ex]{\includegraphics{build/Chapter04C_01e.fig.pdf}}\\
$T=8T_1$&$T=8T_1$\\
\xcell<c>[2ex][0ex]{\includegraphics{build/Chapter04C_01c.fig.pdf}}&
\xcell<c>[2ex][0ex]{\includegraphics{build/Chapter04C_01f.fig.pdf}}\\
$T=16T_1$&$T=16T_1$\\
\end{Table}

\xref{tab:周期性方波的傅里叶系数}中,分别就$T=4T_1, T=8T_1, T=16T_1$的方波$x(t)$及其傅里叶系数$a_k$进行了绘图。

需要说明的是,这里$a_k$是整数只是一个巧合,通常$a_k$是复数,需要两张图才能表示。
\section{连续时间傅里叶级数的收敛}

\subsection{狄利克雷条件}
\fancyref{fml:连续傅里叶级数的系数}中的综合和分析公式在18世纪中叶对于欧拉和拉格朗日来说都是熟悉的,然而他们两人都放弃了这条分析途径,而没有去研究这样一个问题:\empx{究竟多大一类的周期信号可以表示成这种形式}?有趣的是,欧拉和拉格朗日都反对上一节中有关方波的例子,因为方波的$x(t)$是不连续,而每个谐波分量却都是连续的,连续的东西如何组成不连续的东西呢?另一方面,傅里叶研究了同一个例子,并认为方波的傅里叶级数表示也是正确的。不过事实上,傅里叶坚持的是任何周期信号都能用傅里叶级数表示。虽然这一点并不完全正确,但是,傅里叶级数的确能用于表示相当广泛的一类周期信号,包括周期性方波信号。

我们会说,我们都已经写出周期性方波信号的傅里叶系数了,为什么又要纠结周期性方波信号能否用傅里叶级数表示?为了理解这一点,让我们回顾一下目前已经完成的工作
\begin{itemize}
    \item \xref{def:连续傅里叶级数}描述了傅里叶级数的数学形式。
    \item \xref{fml:连续傅里叶级数的系数}描述了\textbf{若一个信号能用傅里叶级数表示},则傅里叶系数应当如何计算。
\end{itemize}
我们容易注意到问题出在哪里,前面推导方波的傅里叶系数的前提是我们已知方波可以用傅里叶级数表示,但问题是,如何确定一个信号能否用傅里叶级数表示?这就是本节要研究的问题。这个问题也可以换一个提法,我们总是可以形式的写出信号$x(t)$对应的傅里叶系数,但是由这些系数构成的傅里叶级数未必仍然收敛至$x(t)$本身,如何判断傅里叶级数是否收敛?



\begin{BoxTheorem}[狄利克雷条件]
    若信号$x(t)$满足以下的\uwave{狄利克雷条件}(Dirichlet Condition)
    \begin{enumerate}
        \item 在任何周期内,$x(t)$必须\uwave{绝对可积}(Absolutely Integrable),即
        \begin{Equation}
            \Int[T]|x(t)|\dd{t}<\infty
        \end{Equation}
        \item 在任意周期内,$x(t)$仅存在有限个极大值和极小值。
        \item 在任意周期内,$x(t)$仅存在有限个不连续点,且在不连续点的两端函数为有限值。
    \end{enumerate}
    那么$x(t)$的傅里叶级数,在连续点收敛至$x(t)$本身,在间断点收敛至左右极限的均值。
\end{BoxTheorem}

\begin{Proof}
    证明从略。
\end{Proof}

在这里,我们不讨论狄利克雷条件在分析上的证明,而是重点考虑一下不满足其条件的例子。\goodbreak

不满足狄利克雷第一条件可以举例如下,如\xref{fig:不满足第一条件},$1/t$在$[0,1]$上的积分并不收敛
\begin{Equation}
    x(t)=\frac{1}{t}\qquad 0<t\leq 1
\end{Equation}

不满足狄利克雷第二条件可以举例如下,如\xref{fig:不满足第二条件},其会出现无穷振荡
\begin{Equation}
    x(t)=\sin(\frac{2\pi}{t})\qquad 0<t\leq 1
\end{Equation}
不满足狄利克雷第三条件可以举例如下,试想函数$x(t)$的周期$T=1$,在每个周期的$[0,1/2]$区间内为$1$,在$[1/2,1/4]$内为$1/2$,在$[1/4,1/8]$内为$1/4$,在$[1/8,1/16]$内为$1/8$,依次类推,即$x(t)=1/2^n, t\in[1/2^n,1/2^{n+1}]$,很明显$x(t)$每个周期内的积分不超过$1$满足第一条件,不存在极值点满足第二条件,但是不连续点的数目却是无穷多个,并不满足第三条件。

\begin{Figure}[不满足狄利克雷条件的函数]
    \begin{FigureSub}[不满足第一条件]
        \includegraphics[scale=0.85]{build/Chapter04D_02a.fig.pdf}
    \end{FigureSub}
    \hspace{0.5cm}
    \begin{FigureSub}[不满足第二条件]
        \includegraphics[scale=0.85]{build/Chapter04D_02b.fig.pdf}
    \end{FigureSub}
\end{Figure}

由此可见,不满足狄利克雷条件的信号,通常都术语比较反常的信号,在实际场合中不会出现。因此,傅里叶级数的收敛问题对于信号与系统将讨论的问题不具有特别重要的意义
\begin{itemize}
    \item 对于一个连续函数,傅里叶级数收敛且在每一点收敛至原来的信号$x(t)$。
    \item 对于一个存在有限个间断点的函数,在连续点上,傅里叶级数仍然收敛至$x(t)$,在孤立的间断点上,傅里叶级数收敛于间断点左右极限的平均值,无论原先间断点的取值或是否定义。由于只是在一些孤立点上存在差异,因此积分意义下,两个信号完全是一致的。
\end{itemize}

\subsection{吉布斯现象}
为了进一步理解对一个不连续点的周期信号,其傅里叶级数是如何收敛的,我们还是回到方波的例子。1898年。美国物理学家迈克尔逊(Albert Michelson)制作了一台谐波分析仪。该仪器可以计算任何周期信号$x(t)$的傅里叶级数截断至$\k=\pm N$后的近似,其中$N$可达到80
\begin{Equation}
    x_N(t)=\Sum[k=-N][N]a_k\e^{\j k\omega_0 t}
\end{Equation}
迈克尔逊\footnote{没错,这位迈克尔逊和“迈克尔逊干涉仪”和“迈克尔逊--莫雷零实验”中的迈克尔逊是同一个人。}用了很多函数来测试他的仪器,结果显示$x_N(t)$和$x(t)$非常一致。然而当他测试到方波信号时,他得到了一个令他吃惊的重要结果!于是他根据这一结果而怀疑他的仪器是否有不完善的地方。他将这一问题写了一封信给著名的数学物理学家吉布斯(Josiah Gibbs)。

迈克尔逊观察到的有趣现象如\xref{tab:方波的吉布斯现象}所示,在不连续点附近,部分和$x_N(t)$呈现的起伏,而且这个起伏的峰值大小似乎不随$N$的增加而下降!吉布斯证明:情况确实是这样,而且也应该是这样,并不是仪器的设计问题。具体而言,若不连续点处的跳变的高度是$1$,则部分和所呈现的峰值最大值约是$1.09$,即在间断点两侧均有$9\%$的超量,并且无论$N$取多大,这个$9\%$的超量不变。这就是\uwave{吉布斯现象}(Gibbs Phenomenon),其含义是,不连续周期信号$x(t)$的傅里叶级数的截断近似$x_N(t)$在接近不连续点时,将呈现高频起伏和$9\%$与$N$无关超量。
\begin{Table}[方波的吉布斯现象]{cc}
    <\mc{2}(c){方波的傅里叶级数的近似$x_N(t)$}\\>
    \xcell<c>[2ex][0ex]{\includegraphics{build/Chapter04D_01a.fig.pdf}}&
    \xcell<c>[2ex][0ex]{\includegraphics{build/Chapter04D_01b.fig.pdf}}\\
    \xcell<c>[2ex][0ex]{\includegraphics{build/Chapter04D_01c.fig.pdf}}&
    \xcell<c>[2ex][0ex]{\includegraphics{build/Chapter04D_01d.fig.pdf}}\\
    \xcell<c>[2ex][0ex]{\includegraphics{build/Chapter04D_01e.fig.pdf}}&
    \xcell<c>[2ex][0ex]{\includegraphics{build/Chapter04D_01f.fig.pdf}}\\
\end{Table}

关于吉布斯现象,首先要认识到的是其与狄利克雷条件之间并不矛盾。尽管存在与$N$无关超量,但是随着$N$的增大,部分和$x_N(t)$的起伏和超量会向不连续处压缩。故部分和$x_N(t)$在连续点仍然会收敛于$x(t)$。关于为何此处存在恒定$9\%$的超量,以下列出了定性解释\cite{zhihu:吉布斯}\cite{enwiki:吉布斯}
\begin{enumerate}
    \item 从部分和$x_N(t)$中$N\to\infty$依狄利克雷条件收敛于$x(t)$来看,首先,狄利克雷定理是正确的,确实收敛。但问题在于,\empx{当存在间断点时,部分和仅逐点收敛而不一致收敛}。依照$\varepsilon$--$N$定义,对于取定的误差$\varepsilon$,如果部分和$x_N(t)$对于每一个$t$的取值,都能在相同的$N$使其于原信号$x(t)$的误差小于$\varepsilon$,那就称为一致收敛,否则就是逐点收敛。该例中,越靠近间断使误差小于$\varepsilon$所需的$N$就越大,无论$N$取多大都无法同时让$x_N(t)$上的每个点与$x(t)$的误差小于$\varepsilon$。因此,无论$N$取多大$x_N(t)$在间断点附近都存在超量。
    \item 从傅里叶系数$a_k$的观点看,方波的傅里叶系数\footnote{这里指的是那种,振幅在$[-1,1]$间且占空比为$0.5$的方波,这在微积分中研究过。}是$1,(1/3),(1/5),(1/7),\cdots$,如果将其视为一个常数项级数,我们会注意到它并不绝对收敛,因为它的衰减速度很慢,和调和级数一致而调和级数并不收敛。事实上,\empx{当存在间断点时,傅里叶系数的衰减速度将很慢,且作为常数项级数并不绝对收敛}。衰减很慢就意味着,每一项都很“重要”,而取部分和$x_N(t)$就相当于截断$N$之后的项,截断了“重要”的项自然就得不到完美的结果。
\end{enumerate}
简而言之,吉布斯现象的本质:当存在间断点,部分和不一致收敛,傅里叶系数不绝对收敛。
\section{连续傅里叶级数的性质}

傅里叶级数具有一系列重要的性质,这些性质对于再概念上深入理解这样的表示是很有用的。并且,它们还有助于简化求取很多信号傅里叶级数的复杂性。在\xref{tab:连续傅里叶级数的性质}中罗列出了这些性质。

\begin{Tablex}[连续傅里叶级数的性质]{XXX}
<性质&时域&频域\\>
    线性&$Ax(t)+By(t)$&$Aa_k+Bb_k$\\
    时移&$x(t-t_0)$&$a_k\e^{-\j k\omega_0t_0}$\\
    频移&$x(t)\e^{\j M\omega_0 t}$&$a_{k-M}$\\
    共轭&$x^{*}(t)$&$a^{*}_{-k}$\\
    时间反转&$x(-t)$&$a_{-k}$\\
    时域尺度变换&$x(\alpha t), \alpha>0$&$a_k$\\
    周期卷积&$x(t)*y(t)=\Int[T]x(\tau) y(t-\tau)\dd{\tau}$&$a_kb_kT$\\
    相乘&$x(t)y(t)$&$a_k*b_k=\Sum[l=-\infty][\infty]a_lb_{k-l}$\\
    微分&$\dv*{x(t)}{t}$&$(\j k\omega_0)a_k$\\
    积分&$\Int[-\infty][t]x(t)\dd{t}$&$(\j k\omega_0)^{-1}a_k$\\
    实信号&$x(t)$为实信号&\xcell<l>[2ex][0ex]{
    $\begin{cases}
        a_k=a_{-k}^{*}\\
        \Re{a_k}=\Re{a_{-k}}\\
        \Im{a_k}=-\Im{a_{-k}}\\
        |a_k|=|a_{-k}|\\
        \arg a_k=-\arg a_{-k}
    \end{cases}$}\\
    实偶信号&$x(t)$为实偶信号&$a_k$仅有实部,且实部为偶函数\\
    实奇函数&$x(t)$为奇偶信号&$a_k$仅有虚部,且虚部为奇函数\\
    帕塞瓦尔定理&\mc{2}{$(1/T)\Int[T]|x(t)|^2\dd{t}=\Sum[k=-\infty][\infty]|a_k|^2$}\\
\end{Tablex}

傅里叶级数的大部分性质可以从傅里叶变换中推导出来,所以本节中不再对此作更多说明。
\section{离散时间傅里叶级数的表示}
本节将讨论离散时间周期信号的傅里叶级数表示,虽然这一节与\xref{sec:连续时间傅里叶级数的表示}的讨论完全以并行的方式进行,但是它们之间有一些很重要的区别,特别是
\begin{itemize}
    \item 连续时间周期信号的傅里叶级数,是无穷级数。
    \item 离散时间周期信号的傅里叶级数,是有限项级数。
\end{itemize}
因此,离散傅里叶级数并不存在连续傅里叶级数中的收敛问题(狄利克雷定理和吉布斯现象)。

\subsection{离散时间傅里叶级数}\setpeq{离散时间傅里叶级数}
正如\xref{chap:信号与系统}中\xref{def:离散信号的周期},若离散信号是周期的,那对于所有的$n$,存在某个正值的$N$,有
\begin{Equation}
    x(n)=x(n+N)
\end{Equation}
在这里,我们将上述$N$的最小正整数称为基波周期,而$\omega_0=2\pi/N$则相应称为基波频率。

离散信号最大的不同,就表现在离散复指数信号集中
\begin{Equation}
    \phi_k(n)=\e^{\j k\omega_0n}\qquad
    k=0,\pm 1,\pm 2,\cdots
\end{Equation}
离散复指数信号集中实质上只有$N$个信号是不同的,具体而言
\begin{Equation}
    \phi_k(n)=\phi_{k+rN}(n)
\end{Equation}
换言之,当$k$变化一个$N$的整数倍时,就得到了一个完全一样的序列。因为尽管频率变得更高了,但由于离散信号仅考察一些离散的整数点,在被考察的整数点上,和过去完全一样!

现在我们尝试用$\phi_k(n)$的线性组合表示更一遍的周期序列
\begin{Equation}
    x(n)=\Sum[k=0][N-1]a_k\phi_{k}(n)=\Sum[k=0][N-1]a_k\e^{\j k\omega_0 n}
\end{Equation}
但由于前述原因,实际上除了$k=0, 1, 2, \cdots, N-1$,完全也可以取$k=1, 2, 3, \cdots, N$,只要是任意$N$个$k$的连续取值即可,介于$\phi_k(n)$仅在$k$的$N$个连续取值范围内是不同的。为了表示这种一般性,特将求和限表示为$k=\<N>$的形式,代表任意$N$个连续取值,即
\begin{Equation}&[1]
    x(n)=\Sum[k=\<N>]a_k\phi_{k}(n)=\Sum[k=\<N>]a_k\e^{\j k\omega_0 n}
\end{Equation}
这就是离散信号的傅里叶级数表示。

\begin{BoxDefinition}[离散傅里叶级数]
    若一个频率为$\omega_0$的离散周期信号$x(n)$被表示为以下形式
    \begin{Equation}
        x(n)=\Sum[k=\<N>]a_k\e^{\j k\omega_0 n}
    \end{Equation}
    就称之为离散信号$x(n)$的傅里叶级数表示。
\end{BoxDefinition}

\subsection{离散时间傅里叶级数的表示}\setpeq{离散时间傅里叶级数的表示}
假设一个周期序列$x(n)$,其周期为$N$,现在想确定$x(n)$能否表示为傅里叶级数的形式?如果可以,那么这些系数$a_k$是什么?这个问题实质上就是要求得一组线性联立方程的解。

根据\fancyref{def:离散傅里叶级数}
\begin{Equation}&[1]
    x(n)=\Sum[k=\<N>]a_k\e^{\j k\omega_0n}
\end{Equation}
联立方程组如何建立?依次令$x(n)$对$n=0,1,2,\cdots,N-1$取值
\begin{Equation}&[2]
    \begin{cases}
        x(0)=\Sum[k=\<N>]a_k\\
        x(1)=\Sum[k=\<N>]a_k\e^{\j k\omega_0}\\
        x(2)=\Sum[k=\<N>]a_k\e^{\j 2k\omega_0}\\
        x(3)=\Sum[k=\<N>]a_k\e^{\j 3k\omega_0}\\
        \vdots\\
        x(N-1)=\Sum[k=\<N>]a_k\e^{\j (N-1)k\omega_0}
    \end{cases}
\end{Equation}
这样,\xrefpeq{2}就是一个具有$N$个未知系数$a_k, k=\<N>$的$N$个线性方程。从理论上可以证明,这$N$个方程是线性独立的,因此可以利用已知的$x(n)$值求得系数$a_k$。不过在这里,我们并不计划这么做,而是仍采用与连续时间情况并行的方法进行推导,得到$x(n)$的一个闭式。

\begin{BoxFormula}[离散傅里叶级数的系数]
    若$x(n)$能表示为傅里叶级数的形式
    \begin{Equation}&[a]
        x(n)=\Sum[k=\<N>]a_k\e^{\j k\omega_0 n}
    \end{Equation}
    那么其中的傅里叶系数$a_k$就相应为(其中$N=2\pi/\omega_0$)
    \begin{Equation}&[b]
        a_k=\frac{1}{N}\Sum[n=\<N>]x(n)\e^{-\j k\omega_0n}
    \end{Equation}
    \xrefpeq{a}称为综合公式,\xrefpeq{b}称为分析公式。
\end{BoxFormula}

\begin{Proof}
   假设一个给定的周期序列$x(n)$可以表示为傅里叶级数,根据\fancyref{def:离散傅里叶级数}
   \begin{Equation}&[1]
       x(n)=\Sum[k=\<N>]a_k\e^{\j k\omega_0n}
   \end{Equation}
   两边各乘$\e^{-\j r\omega_0n}$,其中$r$是任意整数
   \begin{Equation}&[2]
       x(n)\e^{-\j r\omega_0n}=\Sum[k=\<N>]a_k\e^{\j k\omega_0 n}\e^{-\j r\omega_0n}
   \end{Equation}
   两边对$n$在任意连续$N$项上求和
   \begin{Equation}&[3]
       \Sum[n=\<N>]x(n)\e^{-\j r\omega_0n}=\Sum[n=\<N>]\Sum[k=\<N>]a_k\e^{\j k\omega_0 n}\e^{-\j r\omega_0n}
   \end{Equation}
   交换右边的求和顺序
   \begin{Equation}&[4]
    \Sum[n=\<N>]x(n)\e^{-\j r\omega_0n}=\Sum[k=\<N>]a_k\Sum[n=\<N>]\\e^{\j(k-r)\omega_0n}
   \end{Equation}
   而一个显然的事实是
   \begin{Equation}&[5]
       \Sum[n=\<N>]\e^{\j k\omega_0n}=
       \begin{cases}
           N,&k=0,\pm N,\pm 2N,\cdots\\
           0,&\text{otherwise}
       \end{cases}
   \end{Equation}
   即周期复指数序列的值在一个完整周期内求和的结果为零,除非该复指数是某一常数\footnote{当$k=0$时$\e^{\j k\omega_0n}=1$为常数,而$k=\pm N,\pm 2N,\cdots$的结果同$k=0$。}。

   在\xrefpeq{4}中应用\xrefpeq{5}的结果,\xrefpeq{4}右边内层对$n$的求和为零,除非$(k-r)$为零或者是$N$的整倍数。简便起见,不妨假设$r$的取值在$k$的变化范围内,所以,此处右侧的内层求和,在$k=r$时为$N$,在$k\neq r$时为$0$,因此右侧的求和结果就演变为$Na_r$,于是有
   \begin{Equation}
       \Sum[n=\<N>]x(n)\e^{-\j r\omega_0n}=Na_r
   \end{Equation}
   即
   \begin{Equation}
       a_r=\frac{1}{N}\Sum[n=\<N>]x(n)\e^{-\j r\omega_0n}
   \end{Equation}
   将$r$改记为$k$,得到
   \begin{Equation}*
       a_k=\frac{1}{N}\Sum[n=\<N>]x(n)\e^{-\j k\omega_0n}\qedhere
   \end{Equation}
\end{Proof}

现在让我们回过来看\fancyref{fml:离散傅里叶级数的系数}
\begin{Equation}
    x(n)=\Sum[k=\<N>]a_k\phi_k(n)
\end{Equation}
若$k$依次取$0,1,2\cdots,N-1$
\begin{Equation}
    x(n)=a_0\phi_0(n)+a_1\phi_1(n)+a_2\phi_2(n)+\cdots+a_{N-1}\phi_{N-1}(n)
\end{Equation}
若$k$依次取$1,2,3,\cdots,N$
\begin{Equation}
    x(n)=a_1\phi_1(n)+a_2\phi_2(n)+a_3\phi_3(n)+\cdots+a_{N}\phi_{N}(n)
\end{Equation}
由于$\phi_0(n)=\phi_N(n)$,因此$a_0=a_N$,更一般的说
\begin{Equation}
    a_k=a_{k+N}
\end{Equation}
即$a_k$和$\phi_k(n)$一样都是以$N$为周期重复的。诚然,傅里叶级数中只需要$N$个连续的$a_k$值就可以了,但有时需要将$a_k$视作定义在全部$k$值上的序列,此时$a_k=a_{k+N}$就很重要了。

应当指出的是,对于一个离散信号$x(n)$,离散傅里叶级数总是存在的,没有收敛问题,没有吉布斯现象。这是由于这样一个事实,任何离散时间周期序列$x(n)$完全是由有限个参数,这就是在一个周期内的$N$个序列值。从这个角度看,分析公式只是把这$N$个参数变为一组等效的$N$个傅里叶系数值,综合公式则告诉我们如何利用一个有限项级数来回复原来的序列值。




\section{离散傅里叶级数的性质}\setpeq{离散傅里叶级数的性质}
离散傅里叶级数的性质与连续傅里叶级数的性质非常相似,可以对照记忆,如\xref{tab:离散傅里叶级数的性质}所示。

在\xref{tab:连续傅里叶级数的性质}中,若连续信号$x(t), y(t)$满足
\begin{Equation}
    x(t)\FSarr a_k\qquad y(t)\FSarr b_k
\end{Equation}
那么连续相乘性质指出
\begin{Equation}&[1]
    x(t)y(t)\FSarr h_k=a_k*b_k=\Sum[l=-\infty][\infty]a_lb_{k-l}
\end{Equation}

在\xref{tab:离散傅里叶级数的性质}中,若离散信号$x(n), y(n)$满足
\begin{Equation}
    x(n)\FSarr a_k\qquad y(n)\FSarr b_k
\end{Equation}
那么离散相乘性质指出
\begin{Equation}&[2]
    x(n)y(n)\FSarr h_k=a_k*b_k=\Sum[l=\<N>]a_lb_{k-l}
\end{Equation}\goodbreak
注意到\xrefpeq{1}和\xrefpeq{2}虽然同为离散卷积,但略有不同
\begin{itemize}
    \item \xrefpeq{1}中的这种从$-\infty$至$\infty$卷积,称为\uwave{非周期卷积}(Aperiodic Convolution)。
    \item \xrefpeq{2}中的这种在周期$N$内进行的卷积,称为\uwave{周期卷积}(Periodic Convolution)。
\end{itemize}

\begin{Tablex}[离散傅里叶级数的性质]{lXX}
<性质&时域&频域\\>
    线性&$Ax(n)+By(n)$&$Aa_k+Bb_k$\\
    时移&$x(n-n_0)$&$a_k\e^{-\j k\omega_0n_0}$\\
    频移&$x(n)\e^{\j M\omega_0 n}$&$a_{k-M}$\\
    共轭&$x^{*}(n)$&$a^{*}_{-k}$\\
    时间反转&$x(-n)$&$a_{-n}$\\
    时域尺度变换&$x(n/m), m\in\Z^{+}$&$a_k/m$\\
    卷积&$x(n)*y(n)=\Sum[r=\<N>]x(r) y(n-r)$&$a_kb_kN$\\
    相乘&$x(n)y(n)$&$a_k*b_k=\Sum[l=\<N>]a_lb_{k-l}$\\
    差分&$x(n)-x(n-1)$&$(1-\e^{-\j k\omega_0})a_k$\\
    求和&$\Sum[k=-\infty][n]x(t)\dd{t}$&$(1-\e^{-\j k\omega_0})^{-1}a_k$\\
    实信号&$x(t)$为实信号&\xcell<l>[2ex][0ex]{
    $\begin{cases}
        a_k=a_{-k}^{*}\\
        \Re{a_k}=\Re{a_{-k}}\\
        \Im{a_k}=-\Im{a_{-k}}\\
        |a_k|=|a_{-k}|\\
        \arg a_k=-\arg a_{-k}
    \end{cases}$}\\
    实偶信号&$x(t)$为实偶信号&$a_k$仅有实部,且实部为偶序列\\
    实奇函数&$x(t)$为奇偶信号&$a_k$仅有虚部,且虚部为奇序列\\
    帕塞瓦尔定理&\mc{2}{$(1/N)\Sum[n=\<N>]|x(n)|^2\dd{t}=\Sum[k=\<N>]|a_k|^2$}\\
\end{Tablex}

除此之外,离散中的差分性质和连续的微分性质也是对应的,不过结论有些不同。