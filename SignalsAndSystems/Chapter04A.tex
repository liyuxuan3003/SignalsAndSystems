\section{傅里叶分析的历史回顾}

傅里叶分析方法的建立有过一段漫长的历史,涉及到很多人的工作和不同物理现象的研究。利用“三角函数和”,即,成谐波关系的正弦和余弦函数或复指数函数的和的来描述周期性过程,至少可以追溯到古巴比伦时代,当时他们利用这一想法来预测天体运动。这一问题的近代历史始于1748年欧拉对振动弦的研究工作。\xref{fig:振动弦的标准振荡模}绘出了弦振动的前几个标准振荡模式。若用函数$f(x,t)$来表示弦在时间$t$且沿着弦的某一横向距离$x$处的垂直偏离,则对于任意固定时刻$t$来说,所有这些振荡模式均为关于$x$的正弦函数。欧拉从中得出的结论包含两点
\begin{enumerate}
    \item 若在某一时刻,振动弦的形状是这些标准振荡模的现线性组合,那么,在其后任意时刻,振动弦的形状也都是这些振荡模的线性组合,只不过振荡模间的加权系数发生了变化。
    \item 若前面时刻的加权系数(初值)已知,则后面时刻的加权系数可以直接由此求得。
\end{enumerate}
具体而言,如果一个LTI系统的输入可以表示为正弦信号的线性组合,那么该LTI系统的输出也一定能表示为这种形式,且输出信号中的加权系数直接与输入信号中对应的系数有关。

\begin{Figure}[振动弦的标准振荡模]
    \includegraphics[scale=1.05]{build/Chapter04A_01.fig.pdf}\hspace{1.5cm}
\end{Figure}

显然,除非很多有用的信号,比如方波和锯齿波,都能用三角函数或复指数函数的线性组合来表示,否则上面所讨论的性质就不会特别有用,在18世纪中期,这一点曾是激烈争论的主题
\begin{itemize}
    \item 在1753年,伯努利(D. Bernoulli)曾经声称,一根弦的实际运动都可以用标准振荡模的线性组合表示,但他并没有继续从数学上深入研究,并且当时该想法也并未被广泛接受。
    \item 在1759年,拉格朗日(J. L. Lagrange)也曾强烈批量使用正弦级数来研究振动弦运动的主张,他反对的论据是基于他自己的信念,既不可能用三角级数来表示一个具有间断点的函数(例如方波和锯齿波)。因为振动弦的波形是由拨动弦而引起的,即将弦绷紧再松开,所以拉格朗日认为三角级数的应用范围非常有限(没太明白这前后有什么关联)。
\end{itemize}
正是在这种多少有些敌意和怀疑的处境下,傅里叶(Joseph Fourier)约在半个世纪后提出了他自己的想法。热的传播和扩散现象是导致傅里叶研究成果的实际物理背景。在当时数学物理学领域中大多数前人的研究已经涉及理论力学和天体力学的背景下,这一问题本身就是十分有意义的一步。至1807年,傅里叶已经完成了一项研究,他发现在表示一个物体的温度分布时,成谐波关系的正弦函数级数是非常有用的。另外,他还断言,任何周期信号都可以用该级数来表示。虽然在这一问题其论述很有意义,但隐藏在这一问题后面的许多基本概念在那时其实已经被其他科学家们所发现,同时,傅里叶的数学证明也不是很完善,至1829年,狄利克雷(P. L. Dirichlet)才给出了周期信号可以用傅里叶级数表述的若干精确条件,换言之,傅里叶实际上并没有对傅里叶级数的数学理论做出什么贡献。然而,傅里叶确实洞察出了这个级数表示法的潜在威力,并且在很大程度上正是由于他的工作和断言,才激励和推动了傅里叶级数物体的深入研究。除此之外,傅里叶在这一问题上的研究成果比他的任何先驱者都大大前进了一步,这指的是他还得出了关于非周期信号的表示,这不再是成谐波关系的正弦信号的加权和,而是由不全成谐波的正弦信号的加权积分,即傅里叶变换(将在下一章中讨论)。

前面提及的许多应用,以及傅里叶和她的同伴们在数学物理学方面的最初研究,都集中在连续时间内的现象。而与此同时,离散时间信号与系统的傅里叶分析方法,却有着它们自己不同的历史根基,并且也有众多的应用领域。尤其是,离散时间概念和方法是数值分析这门学科的基础。在17世纪的牛顿时代,用于处理离散点集以产生数值近似的有关内插、积分、微分方面的公式就  被研究过。在18世纪和19世纪,已知一组天体观察数据序列,预测某一天体运动的稳态问题,也曾吸引着包含高斯(Gauss)在内的许多著名科学家和数学家从事时间序列调和序列的研究,从而为大量的初始工作能在离散时间信号与系统下完成提供了第二个舞台。