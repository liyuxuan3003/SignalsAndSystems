\section{阶跃函数和冲激函数}
阶跃函数和冲激函数是信号与系统中,两个极为重要的函数,本节将讨论两者的定义与性质。

\subsection{阶跃函数的定义}
阶跃函数当然可以直接的用分段函数定义,但是,比较有趣的一种观点是,我们可以将阶跃函数视为一个线性缓变函数序列的极限,当后者斜率趋于无穷大时,线性缓变就趋于阶跃了。

\begin{BoxDefinition}[阶跃函数]
    选定线性缓变函数序列$\gamma_n(t)$
    \begin{Equation}
        \gamma_n(t)=
        \begin{cases}
            0,&t<-\dfrac{1}{n}\\[4mm]
            \dfrac{1}{2}+\dfrac{n}{2}t,&-\dfrac{1}{n}<t<\dfrac{1}{n}\\[4mm]
            1,&t>\dfrac{1}{n}
        \end{cases}
    \end{Equation}
    \uwave{阶跃函数}(Step Function)定义为$\gamma_n(t)$在$n\to\infty$时的极限,记为$\varepsilon(t)$
    \begin{Equation}
        \varepsilon(t)=\Lim[n][\infty]\gamma_n(t)
    \end{Equation}
    阶跃函数亦可以表示为
    \begin{Equation}
        \varepsilon(t)=
        \begin{cases}
            0,&t<0\\
            1,&t>0
        \end{cases}
    \end{Equation}
\end{BoxDefinition}\goodbreak

阶跃函数的图像,如\xref{fig:阶跃函数}所示,注意到随着$n$增大,$\gamma_n(t)$斜率趋于无穷,变为阶跃。

\begin{Figure}[阶跃函数]
    \begin{FigureSub}[线性缓变函数$\gamma_n(t)$;线性缓变]
        \includegraphics{build/Chapter01D_01a.fig.pdf}
    \end{FigureSub}
    \hspace{0.25cm}
    \begin{FigureSub}[阶跃函数$\varepsilon(t)$;阶跃]
        \includegraphics{build/Chapter01D_01c.fig.pdf}
    \end{FigureSub}
\end{Figure}

\subsection{冲激函数}
冲激函数即$\dirac(t)$,在数学物理方法中我们已经很熟悉了,它可以视为矩形脉冲的极限。

\begin{BoxDefinition}[冲激函数]
    选定矩形脉冲函数序列$p_n(t)$
    \begin{Equation}
        p_n(t)=
        \begin{cases}
            0,&t<-\dfrac{1}{n}\\[4mm]
            \dfrac{n}{2},&-\dfrac{1}{n}<t<\dfrac{1}{n}\\[4mm]
            0,&t>\dfrac{1}{n}
        \end{cases}
    \end{Equation}
    \uwave{冲激函数}(Impulse Function)定义为$p_n(x)$在$n\to\infty$时的极限,记为$\dirac(t)$
    \begin{Equation}
        \dirac(t)=\Lim[n][\infty]p_n(t)
    \end{Equation}
    冲激函数亦可以表示为
    \begin{Equation}
        \dirac(t)=
        \begin{cases}
            0,&t\neq 0\\
            \infty,&t=0
        \end{cases}
    \end{Equation}
    并要求
    \begin{Equation}
        \Int[-\infty][\infty]\dirac(t)\dd{t}=1
    \end{Equation}
\end{BoxDefinition}

冲激函数的图像,如\xref{fig:冲激函数}所示,随着$n$增大,脉冲宽度越来越窄,脉冲高度越来越高,但其脉冲面积始终为$1$,并不随$n$的增大而变化。因此,当$n\to\infty$时,从理论上来说,我们就得到了一个无限窄无限高,但面积为$1$的脉冲,这就是冲激函数$\dirac(t)$的基本想法。然而很明显的是,冲激函数是无法正常作图的,为了凸显冲激函数的特质,在冲激处,以一个键头进行标识,以键头的高度代表冲激脉冲的强度(即脉冲的面积/积分),这里$\dirac(t)$在$0$处的强度为$1$。

冲激函数的这种特性,使其可以用于表示许多“物理量在空间坐标或时间坐标上集中于一点”的物理现象,例如,质量集中于一点的质点的密度函数,电荷集中于一点的点电荷的电荷密度函数,作用时间趋于零的冲击力等,当然,上述宽度趋于零的电脉冲也是一个极好的例子。

\begin{Figure}[冲激函数]
    \begin{FigureSub}[矩形脉冲函数$p_n(t)$;矩形脉冲]
        \includegraphics{build/Chapter01D_01b.fig.pdf}
    \end{FigureSub}
    \hspace{0.25cm}
    \begin{FigureSub}[冲激函数$\dirac(t)$;冲激]
        \includegraphics{build/Chapter01D_01d.fig.pdf}
    \end{FigureSub}
\end{Figure}

然而,尽管我们以函数序列极限的方式定义了冲激函数,但很明显的是,冲激函数$\dirac(t)$并不符合函数的定义,例如其在$t=0$处为无穷大,不是一个实数值。但另外一方面,冲激函数的意义非常清晰,而且在实践中有许多有价值的应用。因此,数学家一致在寻求这类奇异函数的严格定义,在1945至1950年间,施瓦兹(L.Schwartz)发表了论文和专著,建立了广义函数理论,从而使得奇异函数能在数学上被严格定义,为研究奇异函数的数学性质奠定了基础。

接下来,我们介绍广义函数的初步概念和冲激函数$\dirac(x)$的的严格定义和性质。\cite{wiki:狄拉克函数}\cite{wiki:分布}

\subsection{广义函数的引入}
普通函数,如$y=f(x)$是将一维实数空间的实数$x$经过$f$所规定的运算,映射为一维实数空间的实数$y$,普通函数的概念可以推广。推广方向在于,函数的定义域不再要求是实数空间。

问题是,定义域不是实数空间,还可以是什么空间呢?实数空间就是将实数集中的每个实数对应一个空间点,类似的,如果将某类函数集(连续函数集、可微函数集)中的每个函数视为空间中的一个点,这类函数的全体就构成了一个函数空间(连续函数空间、可微函数空间)。粗浅地说,\empx{广义函数就是定义域为函数空间的函数},其代表了一个从函数集到实数集的映射。

\begin{BoxDefinition}[广义函数]
    \uwave{广义函数}(Generalized Function)$g(t)$是对检验函数空间中的每个\uwave{检验函数}(Testing Function)$\varphi(t)$赋予一个数值$N=N[g(t),\varphi(t)]$的映射,通常广义函数$g(t)$可以记为
    \begin{Equation}
        \Int[-\infty][\infty]g(t)\varphi(t)\dd{t}=N[g(t),\varphi(t)]
    \end{Equation}
    若$g(t)$是普通可积函数,上式可以视作积分运算,否则,上式只能作为一种记号。\footnote[2]{换言之,作为解析式不可知的广义函数,我们正是通过这种方式“定义”其积分。}
\end{BoxDefinition}\goodbreak

广义函数的性质,完全取决于所赋予的$N[g(t),\varphi(t)]$的值,因此,广义函数的相等定义如下。
\begin{BoxDefinition}[广义函数的相等]
    设$g_1(t),g_2(t)$是两个广义函数,若有
    \begin{Equation}
        \qquad\qquad
        \Int[-\infty][\infty]g_1(t)\varphi(t)\dd{t}=\Int[-\infty][\infty]g_2(t)\varphi(t)\dd{t}
        \qquad\qquad
    \end{Equation}
    则称$g_1(t),g_2(t)$相等,并记为
    \begin{Equation}
        g_1(t)=g_2(t)
    \end{Equation}
\end{BoxDefinition}

类似的,也可以定义广义极限
\begin{BoxDefinition}[广义极限]
    设$g(t)$是一个广义函数,而$g_n(t)$是一个广义函数列,若有
    \begin{Equation}
        \Int[-\infty][\infty]g(t)\varphi(t)\dd{t}=\Lim[n][\infty]\Int[-\infty][\infty]g_n(t)\varphi(t)\dd{t}
    \end{Equation}
    则称$g(t)$是$g_n(t)$的广义极限,记作
    \begin{Equation}
        g(t)=\Lim[n][\infty]g(t)
    \end{Equation}
\end{BoxDefinition}

广义函数的检验函数空间理论上是可以任意选取的,但由于我们希望我们所定义的广义函数具有比较好的性质,因此检验函数空间的选取要谨慎些,常用的一个是“急降函数空间”。\nopagebreak
\begin{BoxDefinition}[急降函数空间]
    定义\uwave{急降函数空间}为满足以下性质的全体$\varphi(t)$构成的集合,记作$\Phi$
    \begin{enumerate}
        \item $\varphi(t)$是连续且具有任意阶导数的。
        \item $\varphi(t)$及其各阶导数在无限远处急速下降,即当$|t|\to\infty$时,其比$1/|t|^m$下降更快。
    \end{enumerate}
\end{BoxDefinition}

定义在急降函数空间$\Phi$上的广义函数的全体构成\uwave{缓增广义函数空间},记为$\Phi'$。其中,缓增是指当$|t|\to\infty$时其增长不比多项式快。这类广义函数之所以受到重视,是因为其具有非常良好的性质。例如,缓增广义函数空间$\Phi'$中的广义函数的极限、各阶导数、傅里叶变换等均存在且仍属于$\Phi'$。我们之后所讨论的广义函数,都是指缓增广义函数空间$\Phi'$中的广义函数。

\subsection{冲激函数的广义函数定义}
现在,让我们用广义函数的概念,再一次,以较为严格的方式定义冲激函数$\dirac(t)$。
\begin{BoxDefinition}[冲激函数的广义函数定义]
    冲激函数$\dirac(t)$在广义函数观点下定义为
    \begin{Equation}
        \Int[-\infty][\infty]\dirac(t)\varphi(t)\dd{t}=\varphi(0)
    \end{Equation}
    冲激函数$\dirac(t)$作用于检验函数$\varphi(t)$的效果是取出$\varphi(0)$,这也称为取样性质。
\end{BoxDefinition}

为什么选取这样一个定义呢?下面将证明,这样定义的$\dirac(t)$与原先的朴素定义是一致的。
\begin{BoxProperty}[冲激函数的广义极限形式]
    冲激函数$\dirac(t)$是$p_n(t)$的广义极限
    \begin{Equation}
        \dirac(t)=\Lim[n][\infty]p_n(t)
    \end{Equation}
\end{BoxProperty}

\begin{Proof}
    依照\fancyref{def:冲激函数}中$p_n(x)$的定义,其是普通函数,容易写出其广义函数形式
    \begin{Equation}
        \Int[-\infty][\infty]p_n(t)\varphi(t)\dd{t}=\frac{n}{2}\Int[-1/n][1/n]\varphi(t)\dd{t}
    \end{Equation}
    当$n\to\infty$时,在$(-1/n,1/n)$区间内有$\varphi(t)=\varphi(0)$,故
    \begin{Equation}
        \Lim[n][\infty]\Int[-\infty][\infty]p_n(t)\varphi(t)\dd{t}=
        \Lim[n][\infty]\frac{n}{2}\Int[-1/n][1/n]\varphi(t)\dd{t}=\varphi(0)
    \end{Equation}
    而对比\fancyref{def:冲激函数的广义函数定义},注意到
    \begin{Equation}
        \Lim[n][\infty]\Int[-\infty][\infty]p_n(t)\varphi(t)\dd{t}=\Int[-\infty][\infty]\dirac(t)\varphi(t)\dd{t}
    \end{Equation}
    依照\fancyref{def:广义极限},即有广义极限成立
    \begin{Equation}*
        \dirac(t)=\Lim[n][\infty]p_n(t)\qedhere
    \end{Equation}
\end{Proof}

实际上,除了$p_n(t)$外,还有许多函数序列的广义极限都是冲激函数$\dirac(t)$,这里列举一些

高斯函数(当参数$\beta\to\infty$)
\begin{Equation}
    \dirac(t)=\Lim[\beta][\infty]\beta\exp[-\pi(\beta t)^2]
\end{Equation}
取样函数(当参数$\beta\to\infty$)
\begin{Equation}
    \dirac(t)=\Lim[\beta][\infty]\frac{\sin(\beta t)}{\pi t}
\end{Equation}
双边指数函数(当参数$\beta\to 0$)
\begin{Equation}
    \dirac(t)=\Lim[\beta][0]\frac{\exp\qty\big[-|t|/\beta]}{2\beta}
\end{Equation}
以及(当参数$\beta\to 0$)
\begin{Equation}
    \dirac(t)=\Lim[\beta][0]\frac{\beta}{\pi(\beta^2+t^2)}
\end{Equation}
这些函数统称为辅助函数,它们的图像如\xref{tab:冲激函数的其他辅助函数}所示。
\begin{TableLong}[冲激函数的其他辅助函数]{lcc}
<名称&表达式&图像\\>
高斯函数&
\xcell<c>{$\mal{\dirac(t)=\Lim[\beta][\infty]\beta\exp[-\pi(\beta t)^2]}$}&
\xcell<c>[2ex][0ex]{\includegraphics[scale=0.85]{build/Chapter01D_02a.fig.pdf}}\\
取样函数&
\xcell<c>{$\mal{\dirac(t)=\Lim[\beta][\infty]\frac{\sin(\beta t)}{\pi t}}$}&
\xcell<c>[2ex][0ex]{\includegraphics[scale=0.85]{build/Chapter01D_02b.fig.pdf}}\\
双边指标函数&
\xcell<c>{$\mal{\dirac(t)=\Lim[\beta][0]\frac{\exp\qty\big[-|t|/\beta]}{2\beta}}$}&
\xcell<c>[2ex][0ex]{\includegraphics[scale=0.85]{build/Chapter01D_02c.fig.pdf}}\\
--&
\xcell<c>{$\mal{\dirac(t)=\Lim[\beta][0]\frac{\beta}{\pi(\beta^2+t^2)}}$}&
\xcell<c>[2ex][0ex]{\includegraphics[scale=0.85]{build/Chapter01D_02d.fig.pdf}}\\
\end{TableLong}

类似的,亦可将阶跃函数$\varepsilon(t)$以广义函数的方式定义,确定其定义式并不困难,介于$\varepsilon(t)$是普通的可积函数。问题在于,阶跃函数$\varepsilon(t)$原本就是普通函数,为什么我们还要将其广义函数化?我们知道,阶跃函数$\varepsilon(t)$在$t=0$处的导数是无定义的。而实际上,广义函数的除了可以用于表达冲激函数$\dirac(t)$那样连函数表达式都无法写出的函数外,对于$\varepsilon(t)$这样有解析式但不可导的函数,广义函数化后,就可以将其导函数表示为一个广义函数。故这是有意义的。

\begin{BoxDefinition}[阶跃函数的广义函数定义]
    阶跃函数$\varepsilon(t)$在广义函数观点下定义为
    \begin{Equation}
        \Int[-\infty][\infty]\varepsilon(t)\varphi(t)\dd{t}=\Int[0][\infty]\varphi(t)\dd{t}
    \end{Equation}
\end{BoxDefinition}

\begin{BoxProperty}[阶跃函数的广义极限形式]
    阶跃函数$\varepsilon(t)$是$\gamma_n(t)$的广义极限
    \begin{Equation}
        \varepsilon(t)=\Lim[n][\infty]\gamma_n(t)
    \end{Equation}
\end{BoxProperty}

\subsection{冲激函数的导数和积分}
\begin{BoxFormula}[冲激函数的导数]
    冲激函数$\dirac(t)$的导数$\dirac'(t)$应当定义为
    \begin{Equation}
        \Int[-\infty][\infty]\dirac'(t)\varphi(t)\dd{t}=-\varphi'(0)
    \end{Equation}
    更一般的,对于其$n$阶导数$\dirac^{(n)}(t)$
    \begin{Equation}
        \Int[-\infty][\infty]\dirac^{(n)}(t)\varphi(t)\dd{t}=(-1)^n\varphi^{(n)}(t)
    \end{Equation}
\end{BoxFormula}

\begin{Proof}
    作为对冲激函数导数的一个良定义,有理由认为它遵从分部积分的运算规则,故
    \begin{Equation}
        \Int[-\infty][\infty]{\dirac'(t)\varphi(t)\dd{t}}=\dirac(t)\varphi(t)|_{-\infty}^{+\infty}-\Int[-\infty][\infty]{\dirac(t)\varphi'(t)\dd{t}}
    \end{Equation}
    由于冲激函数$\dirac(t)$在$t=\pm\infty$处为零,且检测函数$\varphi(t)$是急降的,上式第一项为零。

    而上式第二项,根据\fancyref{def:冲激函数的广义函数定义}得到
    \begin{Equation}
        \Int[-\infty][\infty]{\dirac'(t)\varphi(t)\dd{t}}=-\varphi(0)
    \end{Equation}
    重复该过程,亦可得到冲激函数$\dirac(t)$的各高阶导数$\dirac^{(n)}(t)$。
\end{Proof}


\begin{BoxFormula}[阶跃函数的导数]
    阶跃函数$\varepsilon(t)$的导数$\varepsilon'(t)$应当定义为
    \begin{Equation}
        \Int[-\infty][\infty]\varepsilon'(t)\varphi(t)\dd{t}=\varphi(0)
    \end{Equation}
    即
    \begin{Equation}
        \varepsilon'(t)=\dirac(t)
    \end{Equation}
    这表明,阶跃函数$\varepsilon(t)$的导数$\varepsilon'(t)$是冲激函数$\dirac(t)$。
\end{BoxFormula}

\begin{Proof}
    应用分部积分法
    \begin{Equation}
        \Int[-\infty][\infty]\varepsilon'(t)\varphi(t)\dd{t}=\varepsilon(t)\varphi(t)|_{-\infty}^{+\infty}-\Int[-\infty][\infty]\varepsilon(t)\varphi'(t)
    \end{Equation}
    由于阶跃函数$\varepsilon(t)$在$x=\pm\infty$处有界,且检测函数$\varphi(x)$是急降的,上式第一项为零。

    而上式第二项,根据\fancyref{def:阶跃函数的广义函数定义}
    \begin{Equation}
        \Int[-\infty][\infty]\varepsilon'(t)\varphi(t)\dd{t}=-\Int[-\infty][\infty]\varphi'(t)\dd{t}=-\varphi(t)|_{0}^{\infty}
    \end{Equation}
    再次运用检测函数的急降性质
    \begin{Equation}
        \Int[-\infty][\infty]\varepsilon'(t)\varphi(t)\dd{t}=\varphi(0)
    \end{Equation}
    而对比\fancyref{def:冲激函数的广义函数定义}
    \begin{Equation}
        \Int[-\infty][\infty]\varepsilon'(t)\varphi(t)\dd{t}=\Int[-\infty][\infty]\dirac(t)\varphi(t)\dd{t}
    \end{Equation}
    因此
    \begin{Equation}*
        \varepsilon'(t)=\dirac(t)\qedhere
    \end{Equation}
\end{Proof}

\subsection{冲激函数的性质}
\begin{BoxProperty}[冲激函数的移位特性]
    冲激函数具有以下移位特性
    \begin{Equation}
        \Int[-\infty][\infty]\dirac(t-t_0)\varphi(t)\dd{t}=\varphi(t_0)
    \end{Equation}
    更一般的
    \begin{Equation}
        \Int[-\infty][\infty]\dirac^{(n)}(t-t_0)\varphi(t)\dd{t}=(-1)^n\varphi^{(n)}(t_0)
    \end{Equation}
\end{BoxProperty}

\begin{Proof}
    作变量代换,令$x=t-t_0$,这就回到了\fancyref{def:冲激函数的广义函数定义}
    \begin{Equation}
        \Int[-\infty][\infty]\dirac(t-t_0)\varphi(t)\dd{t}=\Int[-\infty][\infty]\dirac(x)\varphi(x+t_0)\dx=\varphi(t_0)
    \end{Equation}
    类似地可以证明$\dirac^{(n)}(t-t_0)$的情形。
\end{Proof}

\begin{BoxProperty}[冲激函数的展缩特性]
    冲激函数具有以下展缩特性
    \begin{Equation}
        \dirac(at)=\frac{1}{\abs{a}}\dirac(t)
    \end{Equation}
    更一般的
    \begin{Equation}
        \dirac^{(n)}(at)=\frac{1}{\abs{a}}\frac{1}{a^n}\dirac^{(n)}(t)
    \end{Equation}
\end{BoxProperty}

\begin{Proof}
    若$a>0$,令$x=at$,此时$t=\pm\infty$时$x=\pm\infty$
    \begin{Equation}
        \Int[-\infty][\infty]\dirac(at)\varphi(t)\dd{t}=
        \Int[-\infty][\infty]\dirac(x)\varphi\qty(\frac{x}{a})\frac{1}{a}\dx=\frac{1}{a}\varphi(0)
    \end{Equation}
    若$a<0$,令$x=at$,此时$t=\pm\infty$时$x=\mp\infty$
    \begin{Equation}
        \Int[-\infty][\infty]\dirac(at)\varphi(t)\dd{t}=
        -\Int[-\infty][\infty]\dirac(x)\varphi\qty(\frac{x}{a})\frac{1}{a}\dx=-\frac{1}{a}\varphi(0)
    \end{Equation}
    结合上两式,就有
    \begin{Equation}
        \Int[-\infty][\infty]\dirac(at)\varphi(t)\dd{t}=\frac{1}{|a|}\varphi(0)
    \end{Equation}
    即
    \begin{Equation}
        \dirac(at)=\frac{1}{|a|}\dirac(t)
    \end{Equation}
    类似地也可以证明$\dirac^{(n)}(at)$的情形。
\end{Proof}

作为一个推论,若在\fancyref{ppt:冲激函数的展缩特性}中取$a=-1$,则
\begin{Equation}
    \dirac^{(n)}(-t)=(-1)^n\dirac^{(n)}(t)
\end{Equation}
这就表明
\begin{itemize}
    \item 当$n$为偶数时,有$(-1)^n=+1$,此时$\dirac^{(n)}(t)$是一个偶函数。
    \item 当$n$为奇数时,有$(-1)^n=-1$,此时$\dirac^{(n)}(t)$是一个奇函数。
\end{itemize}

\begin{BoxProperty}[冲激函数与普通函数的乘积]
    冲激函数$\dirac(t-t_0)$与普通函数$f(t)$的乘积,等于其与$f(t_0)$的乘积
    \begin{Equation}
        \dirac(t-t_0)f(t)=\dirac(t-t_0)f(t_0)
    \end{Equation}
    更进一步的
    \begin{Equation}
        \dirac'(t-t_0)f(t)=\dirac'(t-t_0)f(t_0)-\dirac(t-t_0)f'(t_0)
    \end{Equation}
\end{BoxProperty}

\begin{Proof}
    冲激函数普通函数$f(t)$的乘积$\dirac(t-t_0)f(t)$也应看作广义函数,运用\xref{ppt:冲激函数的移位特性}
    \begin{Equation}
        \qquad\qquad\qquad
        \Int[-\infty][\infty][\dirac(t-t_0)f(t)]\varphi(t)\dd{t}=
        \Int[-\infty][\infty]\dirac(t-t_0)[f(t)\varphi(t)]\dd{t}=
        f(t_0)\varphi(t_0)
        \qquad\qquad\qquad
    \end{Equation}
    当然上式做最后一步计算时,$f(t)\varphi(t)$也必须是急降的检验函数,而事实上,这确实是可以满足的,因为$\varphi(t)$急降,即便$f(t)$缓增,只要$f(t)$在$t=t_0$处连续\footnote{为什么需要要求$f(t)$在$t=t_0$处连续?感觉和这里的推论没有关系?},$f(t)\varphi(t)$就是急降的。

    而另外一方面,根据\xref{ppt:冲激函数的移位特性},显然
    \begin{Equation}
        \Int[-\infty][\infty][\dirac(t-t_0)f(t_0)]\varphi(t)\dd{t}=f(t_0)\varphi(t_0)
    \end{Equation}
    因此
    \begin{Equation}
        \dirac(t)f(t)=\dirac(t)f(0)
    \end{Equation}
    而$\dirac'(t-t_0)f(t)$的证明思路是完全相似的,这里从略。
\end{Proof}

\begin{BoxProperty}[冲激函数与普通函数的复合]
    冲激函数与普通函数的复合函数$\dirac[f(t)]$,可以表达为
    \begin{Equation}
        \dirac[f(t)]=\Sum[i=1][n]\frac{1}{|f'(t_i)|}\dirac(t-t_i)
    \end{Equation}
    其中$t_i\,,\ (i=1,2,\cdots,n)$是$f(t)=0$的$n$个互不相等的实根。
\end{BoxProperty}

\begin{Proof}
    在任一单根$t_i$附近足够小的领域内,$f(t)$可以展开为$(t-t_i)$的一阶阶泰勒级数
    \begin{Equation}
        f(t)=f(t_i)+f'(t_i)(t-t_i)
    \end{Equation}
    由于$t_i$是$f(t)=0$的单根,因此$f(t_i)=0$且$f'(t_i)\neq 0$
    \begin{Equation}
        f(t)=f'(t_i)(t-t_0)
    \end{Equation}
    因此$t$取值在$t_i$附近时就有下式,这里运用了\fancyref{ppt:冲激函数的展缩特性}
    \begin{Equation}
        \dirac[f(t)]=\dirac[f'(t_i)(t-t_i)]=\frac{1}{|f'(t_i)|}\dirac(t-t_i)
    \end{Equation}
    由于$t\neq t_i$时$\dirac[f(t)]=0$,因此一般结果($t$取任意值)就是将上式对于每一$t_i$累加
    \begin{Equation}*
        \dirac[f(t)]=\Sum[i=1][n]\frac{1}{|f'(t_i)|}\dirac(t-t_i)\qedhere
    \end{Equation}
\end{Proof}