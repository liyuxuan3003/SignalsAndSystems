\section{LTI离散系统的分解特性}
类似于LIT连续系统,LTI离散系统也可以分解为零输入响应和零状态响应。
\begin{BoxTheorem}[常系数线性差分方程的分解]
    对于常系数线性非齐次差分方程
    \begin{Equation}
        \Sum[i=0][n]a_iy(k-i)=f(k)
    \end{Equation}
    其解可以表示为两部分
    \begin{Equation}
        y(k)=y_\te{zi}(k)+y_\te{zs}(k)
    \end{Equation}
    其中,$y_\te{zi}(k)$是零输入响应,无激励,有初值
    \begin{Equation}
        \Sum[i=0][n]a_iy_\te{zi}(k-i)=0\qquad
        y_\te{zi}(-i)=y(-i),~~i=1,2,\cdots,n
    \end{Equation}
    其中,$y_\te{zs}(k)$是零状态响应,有激励,无初值
    \begin{Equation}
        \Sum[i=0][n]a_iy_\te{zs}(k-i)=f(k)\qquad y_\te{zs}(-i)=0
    \end{Equation}
\end{BoxTheorem}

总的来说,尽管我们总是将两者对比着看,离散系统还是要比连续系统简单的。$n$阶微分方程的初值是$y(t)$的$0,1,2\cdots,n-1$阶导数在$\zp$处的值,将其转换为$\zm$处的值还要花费一些功夫。相比之下,$n$阶差分方程的初值是$y(k)$在$-1,-2,\cdots,-n$处的值,无$\zm,\zp$问题。

