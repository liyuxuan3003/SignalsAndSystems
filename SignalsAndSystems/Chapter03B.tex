\section{LTI离散系统的分解特性}
类似于LIT连续系统,LTI离散系统也可以分解为零输入响应和零状态响应。
\begin{BoxTheorem}[常系数线性差分方程的分解]
    对于常系数线性非齐次差分方程
    \begin{Equation}
        \Sum[i=0][n]a_{n-i}y(k-i)=f(k)
    \end{Equation}
    其解可以表示为两部分
    \begin{Equation}
        y(k)=y_\te{zi}(k)+y_\te{zs}(k)
    \end{Equation}
    其中,$y_\te{zi}(k)$是零输入响应,无激励,有初值
    \begin{Equation}
        \Sum[i=0][n]a_{n-i}y_\te{zi}(k-i)=0\qquad
        y_\te{zi}(-i)=y(-i),~~i=1,2,\cdots,n
    \end{Equation}
    其中,$y_\te{zs}(k)$是零状态响应,有激励,无初值
    \begin{Equation}
        \Sum[i=0][n]a_{n-i}y_\te{zs}(k-i)=f(k)\qquad y_\te{zs}(-i)=0
    \end{Equation}
\end{BoxTheorem}

离散系统和连续系统的初值条件也是对应的,$n$阶微分方程的初值是$y(t)$的$0,1,2\cdots,n-1$阶导数在$\zm$处的值,我们需要将其转化为$\zp$处的值。而相比之下,$n$阶差分方程的初值则是$y(k)$在$-1,-2,\cdots,-n$处的值(对应$\zm$),需将其转化为$y(k)$在$0,1,2,\cdots,n-1$处的值(对应$\zp$),这并不困难,将差分方程递推$n$次即可得到$y(0),y(1),y(2),\cdots,y(n-1)$。